% a sample file for Journal of Quantum Information and Computation (QIC) in 
% LaTex2e by inputing macro file "qic.sty" with command \usepackage{qic}, 
% all the macros have been defined in the style file, so it is no need to 
% put many macros at the beginning of the text file  

\documentclass[twoside]{article}
\usepackage{qic}
\usepackage[dvips]{graphicx}
%\usepackage{times}
\usepackage{fullpage}
\usepackage{eprint}
\usepackage{rotating}
\usepackage{eepic}
\usepackage{amsfonts}
\usepackage{algorithmic}
\usepackage{amsthm}

\theoremstyle{plain}
\newtheorem{theorem}{Theorem}

\input{Qcircuit}

\newcommand{\braket}[2]{\langle #1|#2 \rangle}
\newcommand{\normtwo}{\frac{1}{\sqrt{2}}}
\newcommand{\norm}[1]{\parallel #1 \parallel}


\textwidth=5.6truein
\textheight=8.0truein

\renewcommand{\thefootnote}{\fnsymbol{footnote}}  %use symbolic footnote

%%%%%%%%%%%%%%%%%%%%%%%%%%%%%%%%%%%%%%%%%%%%%%%%%%%%%%%%%%%%%%%%%%%%%%%%%%
%%%%%%% starting the text file 

\begin{document}
\setlength{\textheight}{8.0truein}    %FOR 2ND PAGE ONWARDS

\runninghead{Title  $\ldots$}
            {Author(s) $\ldots$}

\normalsize\textlineskip
\thispagestyle{empty}
\setcounter{page}{1}

%\copyrightheading{Vol.}{No.}{Year}{Page Nos.}
\copyrightheading{0}{0}{2003}{000--000}

\vspace*{0.88truein}

\alphfootnote

\fpage{1}

\centerline{\bf
%%%%%%%%%%%%%%%%%%%%%%%%%%%%%%%%%%%%%%%%%%%%%%%%%%%%%%%%%%%%%%%%%%%%%%%%%%
%Put in titiles here
%%%%%%%%%%%%%%%%%%%%%
Quantum compiling two-qubit gates}

\vspace*{0.37truein}
\centerline{\footnotesize
%%%%%%%%%%%%%%%%%%%%%%%%%%%%%%%%%%%%
%put authors' name and address here
%%%%%%%%%%%%%%%%%%%%%%%%%%%%%%%%%%%%
PAUL PHAM\footnote{University of Washington, Box 352350,
Seattle, WA 98195-2350 USA}}
\vspace*{0.015truein}
\centerline{\footnotesize\it Department of Computer Science & Engineering,
University of Washington,
Box 352350, Seattle, WA 98195-2350}
\vspace*{10pt}
\vspace*{0.225truein}
\publisher{(received date)}{(revised date)}

\vspace*{0.21truein}

%% \abstracts{first paragraph}{second paragraph}{third paragraph}
%% If there is only one paragraph, just keep the second and third empty 
%% like the following one 
\abstracts{
%%%%%%%%%%%%%%%%%%%%
% put abstract here
%%%%%%%%%%%%%%%%%%%%
\input{qcompile-abstract} 
}{}{}

\vspace*{10pt}

\keywords{The contents of the keywords}
\vspace*{3pt}
\communicate{to be filled by the Editorial}

\vspace*{1pt}\textlineskip    %) USE THIS MEASUREMENT WHEN THERE IS
   %) A SECTION HEADING
%\vspace*{-0.5pt}
%\noindent
%%%%%%%%%%%%%%%%%%%%%%%%%%%%%%%%
%put the text of the paper here
%%%%%%%%%%%%%%%%%%%%%%%%%%%%%%%%

\section{Introduction}

Quantum computers can achieve amazing exponential speedups over known classical
algorithms in theory. For example, Shor's factoring algorithm can break the
widely-used RSA cryptosystem, and quantum many-body systems can only be
efficiently simulated by quantum algorithms [paper citations needed].
However, this speedup can be mitigated or lost completely when translating
these algorithms into a physical experiment.
Two important considerations in this implementation process are fault-tolerance
and architecture; both of these introduce their own resource overheads and
complications, but both are ultimately necessary to build a working quantum
computer.
Fault-tolerance (including error correction) is a vast and fascinating topic
that we unfortunately do not address here but take for granted.
Architecture (including compilation) is
concerned with the physical layout and interaction of qubits (or qudits) in
order to execute algorithms efficiently. In actuality, fault-tolerance and
architecture will probably be inter-related concerns, but for now we concern 
ourselves with the more modest task of efficient compilation.

\emph{Quantum compiling} is the
approximation of a high-level quantum algorithm (usually described as a
reversible circuit) to a sequence
of low-level, universal quantum gates that depend on our hardware, the
"assembly language" of quantum computing.
At the highest-level of abstraction, all quantum algorithms on $n$ qubits
can be considered as a unitary matrix of dimension $2^n \times 2^n$ with
unit determinant, followed by measurement (and perhaps several rounds of this).
However, in experimental settings,
we can only perform some gates efficiently, and these are usually local
two-qubit or single-qubit operations.
Moreover, most of our results for fault-tolerant
quantum computing in the presence of noise stipulates that we have a finite
number of universal gates we can perform with some limited precision.

Several quantum compiling procedures are currently known. The first one,
by Solovay and Kitaev (SK),
is a central results of quantum computing which states that we
can approximate quantum gates efficiently without losing any performance
gains over classical computers  \cite{Dawson2005}. That is, we only experience polynomial
overhead in the input size (polylogarithmic overhead in the inverse error)
rather than exponential overhead.
A second procedure due to Kitaev, Shen, and Vyalyi (KSV) is more recent but
less well-known, improves
both the depth and size of the Solovay-Kitaev procedure by trading
more ancillae (space) for time (depth) using parallelism
\cite{ksv02}. There are several other quantum compiling approaches known,
most notably the work of Austin Fowler \cite{Fowler2004} which combines
compiling with fault-tolerance,
which will be mentioned in the next section.
However, the KSV procedure is unique in its combination of various theoretical
tools and warrants this dedicated comparison.
Therefore, we do not consider
these other compiling procedures.

This work contributes the calculation of physical resources needed
to run the SK and KSV quantum compilers on a single-qubit phase gate
of the form $\ket{0}\bra{0} + e^{i\phi}\ket{1}\bra{1}$, for reasons that
will soon become clear. We also contribute open source code to duplicate
these results, which is available at \texttt{http://quantum-compiler.org}.
Finally, we give a pedagogical review of KSV and its building blocks from the
perspective of implementation on an architecture allowing arbitrary concurrent
operations. This model is known as \textsc{AC} in the literature.
\cite{VanMeter2008}. Therefore, this current work can be read in combination
with Chapter 13 of \cite{ksv02} to gain a full understanding of the KSV
compiler.

The rest of this report is organized as follows.
First things first, Section \ref{sec:prelims} defines terms and parameters
so that we can discuss quantum compilers with some rigor as well as
giving asymptotic bounds for specific algorithms.
Then Section
\ref{sec:related} gives a brief history of quantum compiling.
The next two sections describe the two compiling algorithms and how
to measure their relative performance.
Section \ref{sec:sk-algo} reviews the original SK result and
Section \ref{sec:main-algo} describes the building blocks of KSV in detail
along with its
most resource-intensive modules. Section \ref{sec:methods} describe
our methods for the performance comparisons, which are given in Section
\ref{sec:results}. Finally, we make some comments about these results
and suggest future directions for extending this work. Ready? Let's go.

\section{Preliminaries}
\label{sec:prelims}

%%%%%%%%%%%%%%%%%%%%%%%%%%%%%%%%%%%%%%%%%%%%%%%%%%%%%%%%%%%%%%%%%%%%%%%%%%%%%%
\subsection{Some Special Operator Notation}

Borrowing the notation in \cite{ksv02},
we define two controlled ``meta-operators'' which takes some unitary $U$ as
a parameter. The first describes a controlled-$U$ operation where the control
is a single qubit.

\begin{displaymath}
\Lambda(U) = \ket{0}\bra{0} \otimes I + \ket{1}\bra{1} \otimes U
\end{displaymath}

The second describes a registered-$U$ operation, which
can be thought of as applying $U^p$ to some target register
controlled on a second $m$-qubit register
encoding the number $p$.

\begin{displaymath}
\Upsilon_m(U) : \ket{p} \otimes \ket{\psi} \rightarrow \ket{p}
\otimes U^p\ket{\psi}
\end{displaymath}

Note that $\Upsilon_1(U)$ and $\Lambda(U)$ are equivalent.

%%%%%%%%%%%%%%%%%%%%%%%%%%%%%%%%%%%%%%%%%%%%%%%%%%%%%%%%%%%%%%%%%%%%%%%%%%%%%%
\subsection{A Universal Set of Gates}

We use the following universal standard set of gates $\mathcal{G}$.

\begin{displaymath}
\mathcal{G} = \{ H, K, K^{\dagger}, X, Z,
\Lambda(\sigma_x), \Lambda^2(\sigma_x) \}
\end{displaymath}

These gates all operate on single qubits with the
exception of $\Lambda(\sigma_x)$ (the two-qubit CNOT gate)
and $\Lambda^2(\sigma_x)$ (the three-qubit Toffoli gate),
which can be interpreted as singly- and doubly-controlled
$X$ gates, respectively. $X$ and $Z$ are the standard Pauli matrices
$\sigma_x$ and $\sigma_z$, $H$ is the Hadamard matrix, $K$ and its
Hermitian conjugate $K^{\dagger}$ are phase gates of $i$ and $-i$,
respectively.

\begin{displaymath}
Z = 
 \left[
  \begin{array}{cc}
    1 & 0 \\
    0 & -1 \\
  \end{array} \right]
\qquad
X = 
 \left[
  \begin{array}{cc}
    0 & 1 \\
    1 & 0 \\
  \end{array} \right]
\end{displaymath}

\begin{displaymath}
K = 
 \left[
  \begin{array}{cc}
    1 & 0 \\
    0 & i \\
  \end{array} \right]
\qquad
K^{\dagger} = 
 \left[
  \begin{array}{cc}
    1 & 0 \\
    0 & -i \\
  \end{array} \right]
\end{displaymath}

Any current or future physical implementations of a quantum
computer will need to efficiently implement this set or an equivalent one.
Without proving the universality of $\mathcal{G}$, we note that there are
methods for expressing any $2^n \times 2^n$ unitary matrix as a
tensor product of single- and two-qubit gates \cite{Bremner2002}.

\subsection{Parameters}

The problem of quantum compiling is to translate
an entire circuit $C$ of $L$ gates with depth
$D$ to a new, compiled circuit $C'$ of size $L'$ and depth $D'$ which approximates
$C$ within error $\epsilon$ using some distance measure.

In our code, we use the trace measure introduced by Austin Fowler which disregards
the global phase factor. Here,
$l$ refers to the dimensionality
of our system (for $n = 2^l$ qubits).

\begin{equation}
d(U,\tilde{U}) = \sqrt{\frac{l - \norm{\mathrm{tr}(U^\dagger \tilde{U})}}{l}}
\end{equation}

We will be somewhat sloppy and use the terms ``error'', ``precision'', and
``accuracy'' interchangeably when approximating gates.
There is some overhead in the compiled circuit, so in
general $C'$ is larger (that is, $L' > L$ and $D' > D$). It's also known that
in order to approximate a circuit with $L$ gates to a total precision of
$\epsilon$
requires each gate to be approximated to a precision of
$n = O(\log(L/\epsilon)$ \cite{Lloyd1995}.
We'll denote the classical preprocessing time to
produce $C'$ as $T$.
 
Circuit depth is analogous to running time, or how long we have to wait from
feeding in inputs to getting a correct output. Relative to the circuit size,
it is a heuristic for how parallelizable our
circuit is. For example, in a single ion trap, if we had multiple lasers,
we could ``flatten'' our circuit into layers with bounded fan-in and
fan-out and operate on multiple ions in parallel.
We could also operate multiple ion traps in parallel which communicate by
teleportation.
All other things being equal, a circuit with low depth will complete
faster than one with high depth, although in practice we can only execute
fixed-width circuits.

\subsection{Quantum Coprocessor Model}

All experimental implementations of quantum computers treat them as an
auxiliary device controlled by a classical computer. This is the way
quantum computers will function for the foreseeable future, and many
quantum algorithms can actually be split into classical and quantum parts
to reflect this distinction.

For example,
SK is a completely classical procedure which is run before
the quantum algorithm to yield a deterministic set of gates. KSV
also contains classical postprocessing as part of its parallelized
phase-estimation. Since classical computers are well understood and
pretty fast, we will neglect the performance of these classical parts
if they are polynomial in time. However, we will discuss one aspect
of classical overhead later, which is the space requirement of SK.

On a related note, we mention here that most notions of a quantum algorithm
are completely reversible. Specifically, any ancilla qubits which contain
garbage values are usually uncomputed back to $\ket{0}$ so that they can be
reused later. Uncomputing does not affect the asymptotic depth or size of
a circuit. However, in practice, we may be willing to sacrifice a
negligible number of ancillae qubits (leaving them with garbage values) to reduce
circuit depth or size. We use this ``one-shot'' method in the quantum
runtime part of KSV by projectively measuring some ancillae in the phase
estimation component instead of coherently simulating
the measurement and then unmeasuring later. This allows us to offload a
significant (although still polynomial in size and depth) computation to our
classical controller, with a quadratic increase in the error probability.
This is described in more detail in Section \ref{subsec:ppe}.

\subsection{Asymptotic Circuit Bounds}

The SK and KSV algorithms compile circuits with
a size and depth which depend on the desired per-gate precision via the
parameter $n = O(\log(L/\epsilon))$. These asymptotic bounds are listed in
Table \ref{tab:asymptotics}, which actually refer to an improved version of
SK described in Theorem 8.5 of \cite{ksv02}, where 
$\nu$ is a small positive constant.
However, we describe a simpler version of SK in Section \ref{sec:sk-algo}
which has a fixed exponent of about 3.97, which is easier to understand.

We will see these asymptotic bounds reflected
later in the actual numerical results in Section \ref{sec:results}.

\begin{center}
\begin{table}
\label{tab:asymptotics}
\begin{tabular}{|c|c|c|}
\hline
   & Solovay-Kitaev & Super-Kitaev\\
\hline
$L'$ & $O(Ln^{3+\nu})$ & $O(Ln + n^2 \log n)$\\
$d'$ & $O(dn^{3+\nu})$ & $O(d \log{n} + (\log{n})^2))$\\ 
\hline
\end{tabular}
\caption{Asymptotic circuit resources for SK and KSV compilers}
\end{table}
\end{center}

\input{quals-report-related}

\input{quals-report-sk-algo}

\input{quals-report-main-algo}

\input{quals-report-methods}

\input{quals-report-results}

\section{Conclusion and Future Directions}

In summary, we have seen the expected asymptotic bounds of both the
Solovay-Kitaev and Super-Kitaev quantum compiling algorithms reflected in
numerical resource comparisons. Solovay-Kitaev requires a large classical
preprocessing overhead but produces a more tractable number of compiled
gates and zero ancillae, albeit at larger circuit depth. This seems to be
a more reasonable choice for early experiments in running algorithms on an
80-qubit ion-trap quantum computer, where physical trapping constraints
make large numbers of qubits problematic but we are potentially willing to
wait a long time for the computation to complete. On the other hand, the
situation may change in the future when
quantum computers become more mature, scalable, and parallel;
when we become more ambitious in our
algorithm input sizes; and when the performance bottleneck becomes circuit depth.
In that case, Super-Kitaev may be preferrable.
Quantum computer engineers of the future will be able to use this work to
choose the most suitable quantum compiling technique currently available
as well as compare it to future compiling algorithms.

Moreover, the development of a compiler is intertwined with the
development of the underlying architecture. Just as classical programming
languages hint at what features would be desirable to move out of software
(and compile-time)
and into hardware (and run-time), Super-Kitaev provides some suggestions
to future quantum architectures. Hardware which seeks to take advantage of
this low-depth compiling would need to have a
"phase factory" for enacting the $\Lambda(e^{i\phi})$ gate, which would
included an efficient parallelized phase estimation routine. These are
novel resources which are currently not being considered in related literature.

\section{Acknowledgements}

The author would like to gratefully acknowledge the help of
his advisors Dave Bacon and Mark Oskin,
as well as Aram Harrow for the introduction to
Super-Kitaev and related expertise.

\bibliography{qcompile}
\bibliographystyle{tocplain}

\end{document}