%%%%%%%%%%%%%%%%%%%%%%%%%%%%%%%%%%%%%%%%%%%%%%%%%%%%%%%%%%%%%%%%%%%%%%%%%%%%%%
% This is the LaTeX source for
% "A 2D Nearest-Neighbor Quantum Architecture for Factoring"
% submitted to the Reversible Computing Workshop (RC 2012)
% based on Spring-Verlag's Lecture Notes in Computer Sciences template
% typeinst.tex
% Paul Pham and Krysta Svore
% 14 March 2012
%%%%%%%%%%%%%%%%%%%%%%%%%%%%%%%%%%%%%%%%%%%%%%%%%%%%%%%%%%%%%%%%%%%%%%%%%%%%%%

%\documentclass[runningheads]{llncs}
% Suppress page numbers
%\documentclass[a4paper]{llncs}
% For arXiv, and eprint support
\documentclass{article}

\usepackage{amssymb}
\usepackage{amsthm}
%\setcounter{tocdepth}{3}
\usepackage{graphicx}
\usepackage{hyperref}
\usepackage{eprint}
\usepackage[osf]{mathpazo} % Use Palatino / Euler fonts
%\usepackage{multirow}
%\usepackage[retainorgcmds]{IEEEtrantools}

\usepackage{url}
\urldef{\mailsa}\path|{alfred.hofmann, ursula.barth, ingrid.beyer, christine.guenther,|
\urldef{\mailsb}\path|frank.holzwarth, anna.kramer, erika.siebert-cole, lncs}@springer.com|
\newcommand{\keywords}[1]{\par\addvspace\baselineskip
\noindent\keywordname\enspace\ignorespaces#1}

% Right brace for multirows in tables / arrays
\newcommand\coolrightbrace[2]{%
\left.\vphantom{\begin{matrix} #1 \end{matrix}}\right\}#2}

% To fix Qcircuit target with new Xypic
\newcommand{\targfix}{\qw {\xy {<0em,0em> \ar @{ - } +<.39em,0em>
\ar @{ - } -<.39em,0em> \ar @{ - } +
<0em,.39em> \ar @{ - }
-<0em,.39em>},<0em,0em>*{\rule{.01em}{.01em}}*+<.8em>\frm{o}
\endxy}}

% To get Roman uppercase Greek characters
\newcommand{\X}[1]{$#1$}

\input{Qcircuit}

\newcommand{\braket}[2]{\langle #1|#2 \rangle}
\newcommand{\normtwo}{\frac{1}{\sqrt{2}}}
\newcommand{\norm}[1]{\parallel #1 \parallel}
\newcommand{\email}[1]{\href{mailto:#1}{#1}}
\theoremstyle{plain} \newtheorem{lemma}{Lemma}

\begin{document}

%\mainmatter  % start of an individual contribution

% first the title is needed
\title{A 2D Nearest-Neighbor Quantum Architecture for Factoring}

% a short form should be given in case it is too long for the running head
%\titlerunning{A 2D Nearest-Neighbor Quantum Architecture for Factoring}

% the name(s) of the author(s) follow(s) next
%
% NB: Chinese authors should write their first names(s) in front of
% their surnames. This ensures that the names appear correctly in
% the running heads and the author index.
%
\author{Paul Pham\\
University of Washington\\
Quantum Theory Group\\
Box 352350, Seattle, WA 98195, USA,\\
\email{ppham@cs.washington.edu},\\
\url{http://www.cs.washington.edu/homes/ppham/}
\and
Krysta M. Svore\\
Microsoft Research\\
Quantum Architectures and Computation Group\\
One Microsoft Way, Redmond, WA 98052, USA\\
\email{ksvore@microsoft.com},\\
\url{http://research.microsoft.com/en-us/people/ksvore/}
}
% if the list of authors exceeds the space for a headline,
% an abbreviated author list is needed
%\authorrunning{P. Pham \and K.M. Svore}
% (feature abused for this document to repeat the title also on left hand pages)

\maketitle

This document responds to comments by Referee 2, which were received on
November 30, 2012. These comments are quoted and responded to below.

\section{General Comments}

\begin{quote}
The paper "A 2D Nearest-Neighbor Quantum Architecture for Factoring", 
by Pham and Svore, contains new ideas and represents a step forward 
in our understanding of how to implement arithmetic on a quantum 
computer.  The authors do two things that separate their paper from 
previous analyses of Shor's algorithm: they consider a 2D network, 
and they choose to optimize depth rather than width.  While it is 
still too early to say which physical constraints on quantum computers 
will be most restrictive, the authors explore an important new part 
of the space of algorithms.  This paper is appropriate for publication in QIC. 

There are a few issues that should be addressed before publication. 
\end{quote}

\subsection{The introduction does not state the main result}

\begin{quote}
The introduction does not state the main result.  At the very 
least, there should be a statement of the asymptotic behavior of 
the circuit.  It is even possible that most (or all) of Section 8 
should be moved earlier, including figure 11.  The reader should not 
have to read to the end to find out the punch line. 
\end{quote}

\subsection{The introduction to Section 4 is confusing.}

\begin{quote}
The introduction to Section 4 is confusing.  Is the pair $(u,v)$ a 
CSE number only when it arises from this construction?  Is $u_{n-1} = 0 $
a convention within the paper or part of the definition?  The example 
in Figure 2 might suggest that $u_i = v_i = 1$ is not permitted, when 
in fact it is.  The authors need to clarify standard definitions versus 
their conventions. A reference to pre-quantum literature on carry-save 
addition would help. 
\end{quote}

\subsection{Non-unique representations of the answer}

\begin{quote}
The authors imply that the final output of their exponentiation circuit 
is left in CSE form, which is not a unique representation of the answer. 
This could mess up the next step of Shor's algorithm, in which states with 
the same answer collapse.  The authors need to either (a) explain why this 
is not a problem, or (b) explicitly say that they convert the answer to 
standard form.  (Note that, at worst, this conversion can be done in log 
depth, so the asymptotic analysis of the algorithm is be affected.) 
\end{quote}

\section{Minor comments}

\begin{quote}
Some other minor comments, to be addressed at the authors' discretion: 
\end{quote}

\begin{quote}
\begin{itemize}
\item Page 1: "best-known" should be "best known".  ("best-known" means 
"most famous", which is not what is intended.) 
\end{itemize}
\end{quote}

KRYSTA TODO

\begin{quote}
\begin{itemize}
\item
Top of Section 2.1: Should be "following Van Meter and Itoh [Van Meter 
and Itoh (2005)]" or "following [Van Meter and Itoh (2005)]". 
\end{itemize}
\end{quote}

KRYSTA TODO

\begin{quote}
\begin{itemize}
\item
Page 2: ``where each qubit has four neighbors'' is misleading since it 
implies a torus rather than a bounded planar region, and since the authors 
are about to change four to six.  Simply ``where there is an extra...''
would suffice.
\end{itemize}
\end{quote}

KRYSTA TODO

\begin{quote}
\begin{itemize}
\item
Last sentence of Section 2: "there is no known way".  Do the authors 
mean (a) no one has found a way, (b) there is provably no way, or 
(c) there is provably no way for a particular family of circuits? The 
at-large citation to Rosenbaum is unhelpful.  One way to clarify would be 
to cite a specific result from Rosenbaum. 
\end{itemize}
\end{quote}

\begin{quote}
\begin{itemize}
\item
Bottom of page 6: "The circuit operations out-of-place and produces 
two garbage qubits".  No.  An out-of-place circuit leaves its input intact; 
this circuit overwrites $a_i$, so it operates in place.  Calling $b_i$ and 
$c_i$ "garbage" seems not quite right, since if they were not present the 
circuit would not be reversible.  One could design an out-of-place version, 
but in this case cleaning up $b_i$ and $c_i$ would be straightforward and 
there would be no garbage. 
\end{itemize}
\end{quote}

\begin{quote}
\begin{itemize}
\item
Page 7: The phrase "n-bit" appears twice in one sentence, once as a 
an adjective and once as a noun, meaning two different things. This is 
not technically ambiguous, but it is awkward. 
\end{itemize}
\end{quote}

KRYSTA TODO

\begin{quote}
\begin{itemize}
\item
Page 7, before (6): The term "signficance" has not been defined and 
seems not be to used elsewhere.  The sentence could be rewritten. 
\end{itemize}
\end{quote}

KRYSTA TODO: 
I defined the term significance in Section 4, on the Carry-Save technique,
but this appears to be removed in the submitted version.

\begin{quote}
\begin{itemize}
\item
Page 7, proof of lemma 1: extraneous "." in "Our". 
\end{itemize}
\end{quote}

KRYSTA TODO

\begin{quote}
\begin{itemize}
\item
Pages 8 and 9:  The argument that $v'''_{n+2} = 0$ is convoluted, working 
backward from the conclusion to a true statement (and misusing the word 
"implies" in the process), and contains at least one mistake.  It would be 
cleaner to work forward.  For example: since $u'_{(n)} + v'_{(n+1)} = 
u_{(n)} + v_{(n)} <= 2^{n+1}$, the bits $u'_n and v'_{n+1}$ cannot both be $1$. 
But $u''_{n+1} = v'_{n+1}$ and $v''_{n+1} = u'_n v'_n$, so $u''_{n+1}$ and 
$v''_{n+1}$ cannot both be $1$, and hence $v'''_{n+2} = 0$.
\end{itemize}
\end{quote}

\bibliography{factor2d}
\bibliographystyle{tocplain}

\end{document}
