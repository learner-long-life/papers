\section{Results}
\label{sec:results}

The resources required for our approach,
as well as other nearest-neighbor approaches,
are listed in Table \ref{tab:results},
where the asymptotic resource bounds assume some fixed constant error
probability for each round of period-finding.
%, say $\epsilon=1/4$.
% and $\delta' = 1/2$ for KSV-QPF.
%Note that the
%number of measurements are included for completeness.
%, since these are
%not counted as gates in our model but may be comparable in terms of
%execution time.
%Some table cells are blank if the entries are not relevant to the current comparison, or if the entires were not %calculated in the prior work.
We achieve an exponential
improvement in nearest-neighbor circuit depth (from quadratic to polylogarithmic)
with our approach at the cost of a polynomial increase in
circuit width. Similar depth improvements at the cost of width increases can be achieved using the modular multipliers
of other factoring implementations
by arranging them in a parallel, KSV-style modular exponentiator.
%
\begin{table*}[htb!]
\begin{center}
\begin{tabular}{|c|c|c|c|}
\hline
Implementation             & Architecture      & Depth             & Width     \\
\hline
Vedral, Barenco \& Ekert \cite{Vedral1996}   & \textsc{AC}       & $O(n^3)$      & $O(n)$ \\
Gossett \cite{Gossett1998}                   & \textsc{AC}       & $O(n \log n)$ & $O(n^2)$  \\
Beauregard \cite{Beauregard2002}                & \textsc{AC}       & $O(n^3)$      & $O(n)$ \\
Zalka \cite{Zalka1998}                     & \textsc{AC}       & $O(n^2)$      & $O(n)$     \\
Takahashi \& Kunihiro \cite{Takahashi2006}     & \textsc{AC}       & $O(n^3)$      & $O(n)$ \\
%Cleve \& Watrous           & \textsc{AC}       & $O(\log L)$   & $O(L)$ \\
\hline
Fowler, Devitt, Hollenberg \cite{Fowler2004} & \textsc{1D NTC}   & $O(n^4)$ & $O(n^3)$\\
% + 40L^3 + 58L^2 + 2L - 2$ & $32L^3 + 80 L^2 - 4L - 2$ \\
Kutin \cite{Kutin2006}                     & \textsc{1D NTC}   & $O(n^2)$ & $O(n)$\\
\hline
%$18L^2 + 12L\log_2^2 L + O(L \log L) $ & $3L + 2\log_2 L + 2$ \\
Current Work               & \textsc{2D NTC}   & $O(\log^2{n})$ & $O(n^4)$   \\
\hline
\end{tabular}
\end{center}
\caption{Asymptotic resource usage for quantum factoring of an $n$-bit number.}
\label{tab:results}
\end{table*}
