\section{Background}
\label{sec:bg}

Quantum architecture is concerned with the physical layout of
qubits and constraints on their interactions,
as well as the efficient execution, in time, space, and other resources, of
algorithms on a given architecture.
In this paper, we focus on designing a realistic nearest-neighbor circuit for running
Shor's factoring algorithm on architectural models of a physical quantum device.

%%%%%%%%%%%%%%%%%%%%%%%%%%%%%%%%%%%%%%%%%%%%%%%%%%%%%%%%%%%%%%%%%%%%%%%%%%%%%%%
\subsection{Architectural Models and\\ Circuit Resources}
\label{subsec:models}

Following Van Meter \cite{VanMeter2005},
we distinguish between a model and an architectural implementation as follows.
A \emph{model} is a set of constraints and rules for the placement and
interaction of qubits.
An \emph{architecture}, or \emph{implementation}, is a particular
spatial layout of qubits (as a graph of vertices) and their
constrained interactions (edges between the vertices),
following the constraints of a given model.

The most general model is called Abstract Concurrent (\textsc{AC})
and allows arbitrary, long-range interactions between any qubits and concurrent
operation of quantum gates.
This corresponds to a complete graph with an edge between every pair of nodes,
which is the model assumed in most quantum algorithms.

A more specialized model restricts interactions to nearest-neighbor, two-qubit,
concurrent gates (\textsc{NTC}) in a regular one-dimensional chain (1D NTC),
which is sometimes called linear nearest-neighbor (\textsc{LNN}).
This corresponds to a line graph.

To relieve movement congestion,
we can extend to a two-dimensional regular grid
(2D NTC), where each
qubit has four neighbors and there is an extra degree of freedom
in which to move data.
In this paper, we extend the \textsc{2D NTC} model in two ways.
The first extension allows arbitrary planar graphs
with bounded degree, rather than a regular square lattice.
Namely, we assume qubits lie in a plane and edges are not allowed to intersect,
so that theoretically all qubits are accessible from above
or below by control and measurement apparatus.
Whereas 2D NTC conventionally assumes each qubit
has four neighbors, we consider up to six neighbors in a roughly hexagonal
layout. The second extension we make is the realistic assumption
that classical control can
access every qubit in parallel, and we do not count these classical
resources in our implementation. We call these augmented models
\textsc{CCAC} and \textsc{CCNTC} following 
\cite{Rosenbaum2012}. The classical controller
corresponds to fast digital computers which are
available in actual experiments and are necessary for constant-depth
communication in the next section.
%, and it allows us to use constant-depth
%fanout and teleportation.

We measure the efficiency of a circuit on a particular
architecture in terms of three resources:
%, depicted in Figure \ref{fig:resources}:
\emph{circuit size} (number of non-identity gates),
\emph{circuit depth} (number of time-steps), and
\emph{circuit width} (number of qubits).
For circuit depth, a two-qubit gate takes
one time-step and absorbs any adjacent single-qubit gates.
Multiple two-qubit gates on disjoint qubits
can occur in parallel during the same timestep.

%%%%%%%%%%%%%%%%%%%%%%%%%%%%%%%%%%%%%%%%%%%%%%%%%%%%%%%%%%%%%%%%%%%%%%%%%%%%%%%
\subsection{Constant-depth Teleportation\\ and Fanout}
\label{subsec:fanout}

Two key problems in nearest-neighbor architectures deal with communication,
namely moving and copying quantum information.
How can we transport quantum information at one site to another over
arbitrarily long distances?
%A related problem is how to teleport a qubit an arbitrary distance.
% in an
%architecture through ancillae prepared in some initial state.
To solve this problem, we employ the constant-depth teleportation circuit
shown in part (a) of Figure \ref{fig:cd}, using standard quantum circuit
notation from \cite{Nielsen2000}.

\begin{figure*}[tb!]
\begin{center}
\begin{displaymath}
\begin{array}{ccc}
\Qcircuit @C=1em @R=1em {
\lstick{\ket{\psi}}	& \qw      & \qw      & \qw & \qw & \qw & \qw & \qw                                          & \qw & \qw & \multimeasureD{1}{\mbox{Bell}} & \cw & \rstick{j_1} \\
\lstick{\ket{0}}    & \gate{H} & \ctrl{1} & \qw & \qw & \qw & \qw & \qw                                          & \qw & \qw & \ghost{\mbox{Bell}}            & \cw & \rstick{k_1} \\
\lstick{\ket{0}}    & \qw      & \targfix & \qw & \qw & \qw & \qw & \qw_{X^{k_1}Z^{j_1}\ket{\psi}}               & \qw & \qw & \multimeasureD{1}{\mbox{Bell}} & \cw & \rstick{j_2} \\
\lstick{\ket{0}}    & \gate{H} & \ctrl{1} & \qw & \qw & \qw & \qw & \qw                                          & \qw & \qw & \ghost{\mbox{Bell}}            & \cw & \rstick{k_2} \\
\lstick{\ket{0}}    & \qw      & \targfix & \qw & \qw & \qw & \qw & \qw_{X^{k_2}Z^{j_2}X^{k_1}Z^{j_1}\ket{\psi}} & \qw & \qw & \multimeasureD{1}{\mbox{Bell}} & \cw & \rstick{j_3} \\
\lstick{\ket{0}}    & \gate{H} & \ctrl{1} & \qw & \qw & \qw & \qw & \qw                                          & \qw & \qw & \ghost{\mbox{Bell}}            & \cw & \rstick{k_3} \\
\lstick{\ket{0}}    & \qw      & \targfix & \qw & \qw & \qw & \qw & \qw & \qw_{X^{k_3}Z^{j_3}X^{k_2}Z^{j_2}X^{k_1}Z^{j_1}\ket{\psi}} & \qw & \qw              & \qw & \qw \\
}
& \qquad \qquad \qquad &
\Qcircuit @C=1em @R=1em {
\lstick{\ket{\psi}}	& \qw      & \qw      & \qw & \ctrl{1} & \qw & \qw      & \qw & \rstick{\ket{\ell}}\\
\lstick{\ket{0}}	& \qw      & \qw      & \qw & \targfix & \qw & \multimeasureD{1}{\mbox{Bell}} & \cw & \rstick{j_1} \\
\lstick{\ket{0}}    & \gate{H} & \ctrl{1} & \qw & \qw      & \qw & \ghost{\mbox{Bell}}            & \cw & \rstick{k_1} \\
\lstick{\ket{0}}    & \qw      & \targfix & \qw & \ctrl{1} & \qw & \qw      & \qw & \rstick{X^{k_1}Z^{j_1}\ket{\ell}}\\
\lstick{\ket{0}}	& \qw      & \qw      & \qw & \targfix & \qw & \multimeasureD{1}{\mbox{Bell}} & \cw & \rstick{j_2} \\
\lstick{\ket{0}}    & \gate{H} & \ctrl{1} & \qw & \qw      & \qw & \ghost{\mbox{Bell}}           & \cw & \rstick{k_2} \\
\lstick{\ket{0}}    & \qw      & \targfix & \qw & \ctrl{1} & \qw & \qw      & \qw & \rstick{X^{k_2}Z^{j_2}\ket{\ell}}\\
\lstick{\ket{0}}	& \qw      & \qw      & \qw & \targfix & \qw & \multimeasureD{1}{\mbox{Bell}} & \cw & \rstick{j_3} \\
\lstick{\ket{0}}    & \gate{H} & \ctrl{1} & \qw & \qw      & \qw & \ghost{\mbox{Bell}}           & \cw & \rstick{k_3} \\
\lstick{\ket{0}}    & \qw      & \targfix & \qw & \ctrl{1} & \qw & \qw      & \qw & \rstick{X^{k_3}Z^{j_3}\ket{\ell}}\\
\lstick{\ket{0}}	& \qw      & \qw      & \qw & \targfix & \qw & \qw      & \qw & \rstick{X^{k_3}Z^{j_3}\ket{\ell}}\\
}\\
& & \\
(a) & & (b)
\end{array}
\end{displaymath}
\centerline{
}
\caption{Constant-depth circuits based on \cite{Broadbent2007,Browne2009} for
(a) teleportation \protect{\cite{Rosenbaum2012}} and
(b) fanout \cite{Harrow2012}.}
\label{fig:cd}
\end{center}\end{figure*}

The second problem is copying information. Although general cloning is
impossible \cite{Nielsen2000}, we only need to perform unbounded quantum
fanout, the operation
$\ket{x,y_1,\ldots,y_n} \rightarrow \ket{x,y_1\oplus x, \ldots, y_n\oplus x}$.
This is used in our arithmetic circuits when
a single qubit needs to control (be entangled with) a large quantum register
(called a \emph{fanout rail}).
We employ a constant-depth circuit due to insight from
measurement-based quantum computing \cite{Raussendorf2003}
that relies on the creation of an
$n$-qubit cat state \cite{Browne2009}.
It requires $O(1)$-depth, $O(n)$-size, and $O(n)$-width, and is shown in
part (b) of Figure \ref{fig:cd} for the case of fanning out $\ket{\psi}$ to
four qubits.
The technique works by creating multiple small
cat states of a fixed size (in this case, three qubits) and linking them
together with Bell measurements. The qubits marked $\ket{\ell}$ are
entangled into a (slightly) larger cat state, up to Pauli corrections.
%
\begin{equation}
\normtwo X_1^{k_1}Z_1^{j_1}X_2^{k_2}Z_2^{j_2}X_{3}^{k_3}X_{4}^{k_3}Z_{3}^{j_3}Z_{4}^{j_3}
\left( \ket{0000} + \ket{1111} \right)
\end{equation}
%
The operators $X^k_i$ and $Z^j_{\ell}$ denote Pauli $X$ and $Z$ operators
on qubits $i$ and $\ell$, controlled by classical bits $k$ and $j$,
respectively. These corrections are enacted by the classical controller based on
the Bell measurement outcomes (not depicted). Unfortunately, this
``consumes'' the cat state in that there is no known way to unentangle the
source qubit from the cat state after they have been jointly measured \cite{Rosenbaum2012}.
