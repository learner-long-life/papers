\section{Quantum Modular Addition}
\label{sec:csa-mod-add}

To perform addition of two numbers $a$ and $b$ modulo $m$,
we consider the variant problem of modular addition of three numbers to
two numbers:
%
%\begin{quote}
Given three $n$-bit input numbers $a$, $b$, and $c$ and an $n$-bit modulus $m$,
compute the following:
%\begin{equation}
$(u+v) = (a+b+c)[m]$,
%\end{equation}
where $(u+v)$ is a CSE number.

In this section, we provide an alternative, pedagogical explanation of
Gossett's modular reduction \cite{Gossett1998}. Later, we contribute a mapping
to a 2D architecture,
using unbounded fanout to maintain constant-depth for adding back
modular residues. This last step is missing in Gossett's original approach.

To start, we will demonstrate the basic method of modular addition and reduction
on an $n$-bit conventional number. In general, adding two $n$-bit conventional
numbers will produce an overflow $n$-bit, which we can truncate as long as
we add back its modular residue $2^n \bmod m$. How can we guarantee that we won't
generate another overflow bit by adding back the modular residue? It turns out
we can accomplish this by allowing
a slightly larger input and output number ($n+1$ bits in this case), truncating
multiple overflow bits, and adding back their modular residues.

For an $(n+1)$-bit conventional number $x$,
we truncate its high-order bits $x_n$ and $x_{n-1}$
and
add back their \emph{modular residue}, $x_{(n-1,n)}[m]$.
%
\begin{eqnarray}
x \bmod m &=& x_{(0,n)}[m] \nonumber \\
&=& x_{(0,n-2)} + x_{(n-1,n)}[m]
\end{eqnarray}
%
Since both the truncated number $x_{(0,n-2)}$ and the modular residue
are $n$-bit numbers, their sum is an $(n+1)$-bit number as desired, equivalent
to $x[m]$.

Now we must do the same modular reduction on a CSE number $(u+v)$,
which represents an $(n+1)$-bit conventional number and has
$2n$ bits.
%This is the special case mentioned in the
%previous
%section \label{star:csa-special}, where $x$ is the result of a single
%CSA layer, not repeated CSA layers alternating with truncation.
%
%Assume for now that this modular reduction works;
%in the next section we walk through an illustrated concrete example.
%We present a more formal argument in Section \ref{subsec:mod-reduce-1}.
%
First, we truncate the three high-order bits ($v_{n}, u_{n-1}, v_{n-1}$)
of $(u+v)$, yielding an $n$-bit
conventional number with a CSE representation of $2n-3$ bits:
$\{u_0, u_1, \ldots, u_{n-2}\} \cup \{v_1, v_2, \ldots, v_{n-2}\}$.
Then we add back the three modular residues
$(v_{(n)}[m], u_{(n-1)}[m], v_{(n-1)}[m])$, and we are guaranteed not to
get more overflow bits (of significance $2^{n-1}$ or higher). This equivalence
is shown in Equation \ref{eqn:mod-reduce}.
\begin{eqnarray}
(u+v)[m] &=& \left(u_{(0,n-1)} + v_{(1,n)}\right)[m] \nonumber \\
 &=& u_{(0,n-2)} +
     v_{(1,n-2)} + \nonumber \\
 & & u_{(n-1)}[m] + 
     v_{(n-1)}[m] + \nonumber \\
 & & v_{(n)}[m]
\label{eqn:mod-reduce}
\end{eqnarray}

\begin{lemma}[Modular Reduction in Constant Depth \cite{Gossett1998}]
The modular addition of three $n$-bit numbers to two $n$-bit numbers can be
accomplished
in constant depth.
\end{lemma}

\begin{proof}
Our goal is to show how to perform modular addition while keeping our numbers
of a fixed size by treating overflow bits correctly.
First, we enlarge our registers to allow the addition of $(n+2)$-bit numbers,
while keeping our modulus of size $n$ bits.
(In Gossett's original approach, he takes the equivalent step of restricting
the modulus to be of size $(n-2)$ bits.) We accomplish the modular addition
by first performing a layer of non-modular addition, truncating the three high-order
overflow bits, and then adding back modular residues controlled on these
bits in three successive layers, where we are guaranteed that no additional
overflow bits are generated.
This is illustrated for a $3$-bit modulus, and $5$-bit registers,
in Figure \ref{fig:csa-proof}.

\begin{center}
\begin{figure*}[h!tb]
\begin{displaymath}
\renewcommand\arraystretch{1.5}
\begin{array}{ccccccl}
        & a_4 & a_3 & a_2 & a_1 & a_0 & 5\text{-bit input number } a\\
        & b_4 & b_3 & b_2 & b_1 & b_0 & 5\text{-bit input number } b\\
        & c_4 & c_3 & c_2 & c_1 & c_0 & 5\text{-bit input number } c\\
\hline
        & u_4 & u_3 & u_2 & u_1 & u_0 & \text{truncate } u_{4} \\
    v_5 & v_4 & v_3 & v_2 & v_1 &     & \text{truncate } v_{4},v_{5} \\
        &     &     & c^{v_4}_2 & c^{v_4}_1 & c^{v_4}_0 & \text{add back } 2^4 \bmod m \text{ controlled on } v_4 \\
\hline
        &      & u'_3 & u'_2 & u'_1 & u'_0 & \\ 
        & v'_4 & v'_3 & v'_2 & v'_1 &      & \\
        &      &    & c^{u_4}_2 & c^{u_4}_1 & c^{u_4}_0  & \text{add back } 2^4 \bmod m \text{ controlled on } u_4 \\
\hline
        & u''_4 & u''_3 & u''_2 & u''_1 & u''_0 & \text{the bit } u''_4 \text{ is the same as } v'_4 \\
        & v''_4 & v''_3 & v''_2 & v''_1 &       &  \\
        &       &    & c^{v_5}_2 & c^{v_5}_1 & c^{v_5}_0 & \text{add back } 2^5 \bmod m \text{ controlled on } v_5 \\
\hline
        & u'''_4 & u'''_3 & u'''_2 & u'''_1 & u'''_0 & \text{ Final CSE output with } 5 \text{ bits}\\
        & v'''_4 & v'''_3 & v'''_2 & v'''_1 &        & \text{ Final CSE output with } 5 \text{ bits}\\
\end{array}
%\begin{array}{cccccccr}
%        & a_{n+1} & a_{n} & a_{n-1} & \ldots & a_1 & a_0 & \text{input number } a\\
%        & b_{n+1} & b_{n} & b_{n-1} & \ldots & b_1 & b_0 & \text{input number } b\\
%        & c_{n+1} & c_{n} & c_{n-1} & \ldots & c_1 & c_0 & \text{input number } c\\
%\hline
%        & u_{n+1} & u_{n} & u_{n-1} & \ldots & u_1 & u_0 & \text{truncate } u_{n+1} \\
%v_{n+2} & v_{n+1} & v_{n} & v_{n-1} & \ldots & v_1 & 0   & \text{truncate } v_{n+1},v_{n+2} \\
%        &         &       & x_{n-1} & \ldots & x_1 & x_0 \\
%\hline
%        &         & u'_{n} & u'_{n-1} & \ldots & u'_1 & u'_0 & \\ 
%        & v'_{n+1} & v'_{n} & v'_{n-1} & \ldots & v'_1 & 0 &  \\
%        &         &       & x_{n-1} & \ldots & x_1 & x_0 \\
%\hline
%        & u''_{n+1} & u''_{n} & u''_{n-1} & \ldots & u''_1 & u''_0 & \\
%        & v''_{n+1} & v''_{n} & v''_{n-1} & \ldots & v''_1 & 0 &  \\
%        &         &       & y_{n-1} & \ldots & y_1 & y_0 \\
%\hline
%        & u'''_{n+1} & u'''_{n} & u'''_{n-1} & \ldots & u'''_1 & u'''_0 & \\
%        & v'''_{n+1} & v'''_{n} & v'''_{n-1} & \ldots & v'''_1 & 0 &  \\
%\hline
%\end{array}
\end{displaymath}
\caption{A schematic proof of Gossett's constant-depth modular reduction for $n=3$}
\label{fig:csa-proof}
\end{figure*}
\end{center}

We use the following notation.
The non-modular sum of the first layer is $u$ and $v$.
The CSE output of the first modular reduction layer
is $u'$ and $v'$, and the modular residue is
written as $c^{v_{n+1}}$ to mean the precomputed value $2^{n+1} \bmod m$
controlled on $v_{n+1}$.
The CSE output of the second modular reduction layer
is $u''$ and $v''$, and the modular residue is written as
$c^{u_{n+1}}$ to mean the precomputed value $2^{n+1} \bmod m$
controlled on $u_{n+1}$.
The CSE output of the third and final modular reduction layer
is $u'''$ and $v'''$, and the modular residue is written as
$c^{v_{n+2}}$ to mean the precomputed value $2^{n+2} \bmod m$
controlled on $v_{n+2}$.

We show that at no layer is an overflow $(n+2)$-bit generated, namely in the
$v$ component of any CSE output. (The $u$ component will never exceed the
size of the input numbers.) First, we know that no $v'_{n+2}$ bit
is generated after the first modular reduction layer, because we have
truncated away all $(n+1)$-bits. Second, we know that no $v''_{n+2}$ bit is
generated because we only have one $(n+1)$-bit to add, $v'_{n+1}$.
Finally, we need to show a sufficient condition for no $v'''_{n+2}$ bit being
generated in the third modular reduction layer. This bit is the majority of
$u''_{n+1}$, $v''_{n+1}$, and $c^{v_{n+2}}_{n+1} = 0$. This means we only have
to guarantee that at most one of $u''_{n+1}$ and $v''_{n+1}$ has value 1.
This is equivalent to requiring that
$u''_{(n,n+1)} + v''_{(n+1)} \le 3\cdot 2^{n+1}$, that is, the sum of these
three bits has value at most $3$. Bit $u''_{n+1}$ is copied directly from
$v'_{n+1}$ by the rules of CSA, which implies the following condition for
the second modular reduction layer:
$u'_{(n)} + v'_{(n,n+1)} \le 3\cdot 2^n$. This is true because
$u'_{(n)} + v'_{(n+1)} = u_{(n)} + v_{(n)} \le 2$ and $v'_{(n)} \le 1$.
Everywhere
we use the fact that the modular residues are restricted to $n$ bits.
Therefore, the modular sum is computed as the sum of two $(n+2)$-bit numbers
with no overflows in constant-depth.
\end{proof}

As a side note, we can perform modular reduction in one layer instead of
three by decoding the three overflow bits into one of seven different
modular residues. This can also be done in constant depth, and in this case
we only need to enlarge all our registers to $(n+1)$ bits instead of $(n+2)$
as in the proof above. However, we omit this proof here for simplicity.

To summarize,
the circuit resources for modular addition are $O(1)$ depth and $O(n)$ width.

%%%%%%%%%%%%%%%%%%%%%%%%%%%%%%%%%%%%%%%%%%%%%%%%%%%%%%%%%%%%%%%%%%%%%%%%%%%%%%%
\subsection{A Concrete Example of\\ Modular Addition}
\label{subsec:concrete}

\begin{center}
\begin{figure*}[h!bt]
\centerline{
\includegraphics[width=6.5in]{./figures/mod-add-fixed.pdf}
}
\caption{Addition and three rounds of modular reduction for a 3-bit
modulus.}
\label{fig:csa-add-4}
\end{figure*}
\end{center}

A 2D circuit for modular addition of $5$-bit numbers using
four layers of parallel CSA's is shown graphically in Figure \ref{fig:csa-add-4}
which corresponds directly to the schematic proof in Figure \ref{fig:csa-proof}.
Figure \ref{fig:csa-add-4} also represents the approximate
physical layout of the qubits as they would look if this
circuit were to be fabricated.
Here, we convert the sum of three
$5$-bit integers into the modular sum of two $5$-bit integers, with a
$3$-bit modulus $m$.
In the first layer, 
we perform 4 CSA's in parallel on the input numbers ($a,b,c$) and produce the
output numbers ($u, v$).

As described above, we truncate
the three high-order bits during the initial CSA round
(bits $u_4, v_4, v_5$) to retain a $4$-bit number.
Each of these bits serves as a control for adding its modular residue to
a running total. We can classically precompute $2^4[m]$ for the two
additions controlled on $u_4$ and $v_4$ and
$2^5[m]$ for the addition controlled on $v_5$.

In layer 2,
we use a constant-depth fanout rail (see Figure \ref{fig:cd}) to
distribute the control bit $v_4$ to its modular residue, which we denote as
%%\begin{equation}
$\ket{c^{v_4}} \equiv \ket{2^4[m]\cdot v_4}$.
%%\end{equation}
%This fanout requires constant depth;
$c^{v_4}$ has $n$ bits, which we add to the CSE results of layer 1.
The results $u_i$ and $v_{i+1}$ are teleported into layer 3. The exception is
$v'_4$ which is teleported into layer 4, since there are no other $4$-bits
to which it can be added. Wherever there are only
two bits of the same significance, we use the 2-2 adder from \ref{sec:csa}.

Layer 3
%%, shown in Figure \ref{fig:csa-add-3},
operates similarly to layer 2, except that the modular residue is controlled on
$u_4$:
%%\begin{equation}
$\ket{c^{u_4}} \equiv \ket{2^4[m] \cdot u_4}$.
%%\end{equation}
%This fanout again requires constant depth; 
$c^{u_4}$ has $3$ bits, which we
add to the CSE results of layer 2, where $u'_i$ and $v'_{i+1}$ are teleported
forward into layer 4.

Layer 4
%%, shown in Figure \ref{fig:csa-add-4},
is similar to layers 2 and 3, with the modular residue controlled on $v_5$:
%%\begin{equation}
$\ket{c^{v_5}} \equiv \ket{2^5[m] \cdot v_5}$.
%%\end{equation}
%This fanout is constant depth; 
$c^{v_5}$ has $3$ bits, which we
add to the CSE results of layer 3.
There is no overflow bit $v'''_5$, and no carry bit from $v''_4$ and $v'_4$
as argued in Lemma 1.
The final modular sum $(a+b+c)[m]$ is $u'''+v'''$.
