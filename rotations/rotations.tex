\documentclass[twocolumn,10pt]{article}

\usepackage[dvips]{graphicx}
%\usepackage{times}
\usepackage{eprint}
\usepackage{eepic}
\usepackage{amsfonts}
\usepackage{algorithmic}
\usepackage{amsthm}

\theoremstyle{plain}
\newtheorem{theorem}{Theorem}

\title{Single-qubit rotations}
\author{Arman, Poya, Paul Pham}

\input{Qcircuit}

\begin{document}

%\newcommand{\ket}[1]{|#1 \rangle}
%\newcommand{\bra}[1]{\langle #1 |}
\newcommand{\braket}[2]{\langle #1|#2 \rangle}
\newcommand{\normtwo}{\frac{1}{\sqrt{2}}}
\newcommand{\norm}[1]{\parallel #1 \parallel}

\maketitle

\section{Definitions}

In this note, we call a single-qubit rotation
any single qubit gate (a matrix in $SU(2)$) which
represents a geometric rotation (a homomorphic operation in $SO(3)$)
about the Bloch sphere $z$-axis by angle $\theta/2$. We denote the
$SU(2)$ matrix as follows, noting that the angle which appears as a matrix
element is twice the angle of the geometric rotation due to the
double covering between $SU(2)$ and $SO(3)$.

\begin{equation}
R_Z(\theta / 2) = e^{i\frac{\theta}{2}\sigma_z} = 
 \left[
  \begin{array}{cc}
    1 & 0 \\
    0 & e^{i\theta} \\
  \end{array} \right]
\end{equation}

There are two special cases of single-qubit rotations which can be done
fault-tolerantly in many quantum error-correcting codes because they are
members of the Clifford group. These are 
the Pauli operator $\sigma_Z$ is a special case of this matrix denoted
as $R_Z(\pi/2)$ and the phase gate $S$ denoted as $R_Z(\pi/4)$, which can
also be thought of as the ``square root of $Z$'' gate.

\begin{equation}
\sigma_z = R_Z(\pi / 2) = e^{i\frac{\pi}{2}\sigma_z} = 
 \left[
  \begin{array}{cc}
    1 & 0 \\
    0 & e^{i \pi} \\
  \end{array} \right]
=
 \left[
  \begin{array}{cc}
    1 & 0 \\
    0 & -1 \\
  \end{array} \right]
\end{equation}

\begin{equation}
S = R_Z(\pi / 4) = e^{i\frac{\pi}{4}\sigma_z} = 
 \left[
  \begin{array}{cc}
    1 & 0 \\
    0 & e^{i \pi/2} \\
  \end{array} \right]
=
 \left[
  \begin{array}{cc}
    1 & 0 \\
    0 & i \\
  \end{array} \right]
\end{equation}

Single-qubit rotations play a key role in quantum compiling and quantum
algorithms in general since it is only known how to d

We denote by with
the following form:

\begin{equation}
\Lambda(e^{i\phi}) = 
 \left[
  \begin{array}{cc}
    1 & 0 \\
    0 & e^{i\phi} \\
  \end{array} \right]
\end{equation}


\bibliography{rotations}
\bibliographystyle{tocplain}

\end{document}
