\section{Preliminaries}
\label{sec:prelims}

%%%%%%%%%%%%%%%%%%%%%%%%%%%%%%%%%%%%%%%%%%%%%%%%%%%%%%%%%%%%%%%%%%%%%%%%%%%%%%
\subsection{Some Special Operator Notation}

Borrowing the notation in \cite{ksv02},
we define two controlled ``meta-operators'' which takes some unitary $U$ as
a parameter. The first describes a controlled-$U$ operation where the control
is a single qubit.

\begin{displaymath}
\Lambda(U) = \ket{0}\bra{0} \otimes I + \ket{1}\bra{1} \otimes U
\end{displaymath}

The second describes a registered-$U$ operation, which
can be thought of as applying $U^p$ to some target register
controlled on a separate $m$-qubit register
encoding the number $p$.

\begin{displaymath}
\Upsilon_m(U) : \ket{p} \otimes \ket{\psi} \rightarrow \ket{p}
\otimes U^p\ket{\psi}
\end{displaymath}

Note that $\Upsilon_1(U)$ and $\Lambda(U)$ are equivalent.

%%%%%%%%%%%%%%%%%%%%%%%%%%%%%%%%%%%%%%%%%%%%%%%%%%%%%%%%%%%%%%%%%%%%%%%%%%%%%%
\subsection{A Universal Set of Gates}

We use the following universal standard set of gates $\mathcal{G}$ throughout
this paper.

\begin{displaymath}
\mathcal{G} = \{ H, K, K^{\dagger}, X, Z,
\Lambda(\sigma_x), \Lambda^2(\sigma_x) \}
\end{displaymath}

These gates all operate on single qubits with the
exception of $\Lambda(\sigma_x)$ (the two-qubit CNOT gate)
and $\Lambda^2(\sigma_x)$ (the three-qubit Toffoli gate),
which can be interpreted as singly- and doubly-controlled
$X$ gates, respectively. $X$ and $Z$ are the standard Pauli matrices
$\sigma_x$ and $\sigma_z$, $H$ is the Hadamard matrix, and $K$ and its
Hermitian conjugate $K^{\dagger}$ are phase gates of $i$ and $-i$,
respectively.

\begin{displaymath}
Z = 
 \left[
  \begin{array}{cc}
    1 & 0 \\
    0 & -1 \\
  \end{array} \right]
\qquad
X = 
 \left[
  \begin{array}{cc}
    0 & 1 \\
    1 & 0 \\
  \end{array} \right]
\end{displaymath}

\begin{displaymath}
K = 
 \left[
  \begin{array}{cc}
    1 & 0 \\
    0 & i \\
  \end{array} \right]
\qquad
K^{\dagger} = 
 \left[
  \begin{array}{cc}
    1 & 0 \\
    0 & -i \\
  \end{array} \right]
\end{displaymath}

Without proving the universality of $\mathcal{G}$, we note that there are
methods for expressing any $2^n \times 2^n$ unitary matrix as a
tensor product of single- and two-qubit gates \cite{Bremner2002}. We call
the process of decomposing elements of $SU(N)$ into elements of
$SU(2)$ and $SU(4)$ \emph{quantum circuit synthesis}
[citation to Markov]
to distinguish them as
an intermediate stage in quantum compiling, for which procedures like SK and
KSV provide the final stage of producing circuits formed from $\mathcal{G}$.

%%%%%%%%%%%%%%%%%%%%%%%%%%%%%%%%%%%%%%%%%%%%%%%%%%%%%%%%%%%%%%%%%%%%%%%%%%%%%%
\subsection{Parameters}

The problem of quantum compiling is to translate
an entire circuit $C$ of $S$ gates with depth
$D$ to a new, compiled circuit $C'$ of size $S'$ and depth $D'$ which approximates
$C$ within error $\epsilon$ using some distance measure.

%%%%%%%%%%%%%%%%%%%%%%%%%%%%%%%%%%%%%%%%%%%%%%%%%%%%%%%%%%%%%%%%%%%%%%%%%%%%%%
%%%%%%%%%%%%%%%%%%%%%%%%%%%%%%%%%%%%%%%%%%%%%%%%%%%%%%%%%%%%%%%%%%%%%%%%%%%%%%
% Figure out what you are actually measuring / calculating in your code
$S$ is measured in the number of gates from $\mathcal{G}$. $D$ is measured in
number of concurrent timesteps of ... $W$ is measured in total number of 
ancillae qubits needed to maintain a coherent state for the duration of the
computation.

In our code, we use the trace measure introduced by Austin Fowler which disregards
the global phase factor. Here,
$d$ refers to the dimensionality
of our system ($d = 2^n$ for $n$ qubits).

\begin{equation}
d(U,\tilde{U}) = \sqrt{\frac{d - \norm{\mathrm{tr}(U^\dagger \tilde{U})}}{d}}
\end{equation}

We will use the terms ``error'', ``precision'', and
``accuracy'' interchangeably when approximating gates.
There is some overhead in the compiled circuit, so in
general $C'$ is larger (that is, $S' > S$ and $D' > D$). It's also known that
in order to approximate a circuit with $S$ gates to a total precision of
$\epsilon$
requires each gate to be approximated to a precision of
$n = O(\log(S/\epsilon)$ \cite{Lloyd1995}. We denote this per-gate precision
$n$, since it serves as an independent parameter for compiling. We will clarify
where there is confusion with using $n$ to denote the number of qubits in a
register or being acted upon by a circuit.
We denote at $T$ the classical preprocessing time to
produce $C'$.
 
Circuit depth is analogous to running time, or how long we have to wait from
feeding in inputs to getting a correct output. Relative to the circuit size,
it is a heuristic for how parallelizable our
circuit is. For example, in a single ion trap, if we had multiple lasers,
we could ``flatten'' our circuit into layers with bounded fan-in and
fan-out and operate on multiple ions in parallel.
We could also operate multiple ion traps in parallel which communicate by
teleportation.
All other things being equal, a circuit with low depth will complete
faster than one with high depth, although in practice we can only execute
fixed-width circuits.

%%%%%%%%%%%%%%%%%%%%%%%%%%%%%%%%%%%%%%%%%%%%%%%%%%%%%%%%%%%%%%%%%%%%%%%%%%%%%%
\subsection{Quantum Coprocessor Model}

All experimental implementations of quantum computers treat them as an
auxiliary device controlled by a classical computer. This is the way
quantum computers will function for the foreseeable future, and many
quantum algorithms can be split into classical and quantum parts
to reflect this distinction. The existence of classical control in this
``quantum coprocessor'' model is useful not only in quantum compiling but
also allows us to perform certain quantum circuits in constant depth
[citation needed for fanout and parity].

For example,
SK is a completely classical procedure which is run before
the quantum algorithm to yield a deterministic set of gates. KSV
also contains classical side-processing as part of its parallelized
phase-estimation. We will neglect the performance of these classical parts
since they are polynomial in time or space, with one exception. We will
discuss the classical space overhead of SK in Section \ref{subsec:results-width}.

On a related note, we may choose to projectively measure some ancilla qubits
in the middle of a quantum algorithm in order to offload some (polynomial-time)
processing to the classical controller.
These ancillae then contain garbage values and
cannot be used for the remainder of the computation. From a practical point of
view, this ``one-shot'' method is worth doing if the number of such
wasted qubits is small compared to the reduction in the quantum circuit size.
We discuss our use of this technique in Section \ref{subsec:ppe} for parallelized
phase estimation.

The alternative is sometimes
called ``coherent measurements'' or ``simulating measurements'' \cite{Cleve2000}
in which some operation is controlled on the quantum value to be measured and
can be reversed later to recover ancillae at the cost of a larger quantum circuit.

%%%%%%%%%%%%%%%%%%%%%%%%%%%%%%%%%%%%%%%%%%%%%%%%%%%%%%%%%%%%%%%%%%%%%%%%%%%%%%
\subsection{Asymptotic Circuit Bounds}

The SK and KSV algorithms compile circuits with
a size and depth which depend on the desired per-gate precision $n$.
These asymptotic bounds are listed in
Table \ref{tab:asymptotics}, which actually refer to an improved version of
SK described in Theorem 8.5 of \cite{ksv02}, where 
$\nu$ is a small positive constant.
However, we describe a simpler version of SK in Section \ref{sec:sk-algo}
which has a fixed exponent of about $c=3.97$, which is easier to understand,
and does not significantly change the asymptotic relationship between SK and
KSV.

We will see these asymptotic bounds reflected
later in the actual numerical results in Section \ref{sec:results}.

\begin{center}
\begin{table}
\label{tab:asymptotics}
\begin{tabular}{|c|c|c|}
\hline
   & SK & KSV \\
\hline
$S'$ & $O(Sn^{3+\nu})$ & $O(Sn + n^2 \log n)$\\
$D'$ & $O(Dn^{3+\nu})$ & $O(D \log{n} + (\log{n})^2))$\\ 
$W'$ & $0$             & $$ \\
\hline
\end{tabular}
\caption{Asymptotic circuit resources for SK and KSV compilers}
\end{table}
\end{center}