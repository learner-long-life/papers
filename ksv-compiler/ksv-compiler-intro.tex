\section{Introduction}

Quantum computers can achieve exponential speedups over known classical
algorithms in theory. For example, Shor's factoring algorithm can break the
widely-used RSA cryptosystem, and quantum many-body systems can only be
efficiently simulated by quantum algorithms [paper citations needed].
However, this speedup can be mitigated or lost completely when implementing
these algorithms in a physical experiment.
Two important implementation considerations 
are fault-tolerance and architecture. We do not address the vast topic
of fault-tolerance (including error correction) here, and instead
concentrate on compiling, which we include in the domain of architecture.

\emph{Quantum compiling} is the
approximation of a high-level quantum algorithm (usually described as a
reversible circuit) using a sequence
of low-level, universal quantum gates that depend on our hardware, the
"assembly language" of quantum computing.
At the highest-level of abstraction, all quantum algorithms on $n$ qubits
can be considered as a unitary matrix of dimension $2^n \times 2^n$ with
unit determinant, followed by measurement (and perhaps several rounds of this).
However, in experimental settings,
we can only perform some gates efficiently, and these are usually local
(nearest-neighbor) two-qubit or single-qubit operations.
Moreover, most of our results for fault-tolerant
quantum computing in the presence of noise stipulates that we have a finite
number of universal gates we can perform with some limited precision.

Several quantum compiling procedures are currently known. The first one,
by Solovay and Kitaev (SK),
is a central results of quantum computing which states that we
can approximate quantum gates efficiently in circuit size and
depth (i.e. without losing any exponential speedup) \cite{Dawson2005}.
A second procedure due to Kitaev, Shen, and Vyalyi (KSV) is more recent but
less well-known, improves
both the depth and size of the SK procedure by trading
more ancillae (space) for time (depth) using parallelism
\cite{ksv02}. Several other quantum compiling approaches are described
in the next section. However, the KSV procedure is unique in its combination of
various theoretical tools and warrants a dedicated comparison.

This work contributes the calculation of physical resources needed
to run the SK and KSV quantum compilers on a single-qubit phase gate
of the form $\ket{0}\bra{0} + e^{i\phi}\ket{1}\bra{1}$, for reasons that
will soon become clear. We also contribute open source code to duplicate
these results, which is available at \texttt{http://quantum-compiler.org}.
Finally, we give a pedagogical review of KSV and its building blocks from the
perspective of implementation on an architecture allowing arbitrary concurrent
operations. This model is known as \textsc{AC} in the literature.
\cite{VanMeter2008}. Therefore, this current work can be read in combination
with Chapter 13 of \cite{ksv02} to gain a full understanding of the KSV
compiler.

The rest of this report is organized as follows.
First, Section \ref{sec:prelims} defines terms and parameters
so that we can discuss quantum compilers with some rigor as well as
giving asymptotic bounds for specific algorithms.
Then Section
\ref{sec:related} gives a brief history of quantum compiling.
The next two sections describe the two compiling algorithms and how
to measure their relative performance.
Section \ref{sec:sk-algo} reviews the original SK result and
Section \ref{sec:main-algo} describes the building blocks of KSV in detail
along with its
most resource-intensive modules. Section \ref{sec:methods} describe
our methods for the performance comparisons, which are given in Section
\ref{sec:results}. Finally, we make some comments about these results
and suggest future directions for extending this work.