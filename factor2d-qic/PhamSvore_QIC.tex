% a sample file for Journal of Quantum Information and Computation (QIC) in
% LaTex2e by inputing macro file "qic.sty" with command \usepackage{qic},
% all the macros have been defined in the style file, so it is no need to
% put many macros at the beginning of the text file

\documentclass[twoside]{article}
\usepackage{qic,epsfig}

%%%%%%% starting the text file

\usepackage{graphicx}% Include figure files
\usepackage{dcolumn}% Align table columns on decimal point
\usepackage{bm}% bold math
%\usepackage{amsthm} %theorem package
%\usepackage{titlesec} %title section
\usepackage{hyperref}% add hypertext capabilities
%\usepackage[mathlines]{lineno}% Enable numbering of text and display math
%\linenumbers\relax % Commence numbering lines

%\usepackage[showframe,%Uncomment any one of the following lines to test
%%scale=0.7, marginratio={1:1, 2:3}, ignoreall,% default settings
%%text={7in,10in},centering,
%%margin=1.5in,
%%total={6.5in,8.75in}, top=1.2in, left=0.9in, includefoot,
%%height=10in,a5paper,hmargin={3cm,0.8in},
%]{geometry}

%\titlespacing*{<cmd>}{<left sep>}{<before sep>}{<after sep>}[<right sep>]
\setcounter{secnumdepth}{3}

%\newtheorem*{example}{Example}
%\newtheorem*{mydef}{Definition}
%\newtheorem*{problem}{Problem}
%\newtheorem{thm}{Theorem}
%\newtheorem{proposition}{Proposition}
%\newtheorem{lemma}{Lemma}
%\newtheorem{observation}{Observation}
%\newtheorem{query}{Query}X

%\newtheorem{conjecture}{Conjecture}
%\newtheorem{corol}{Corollary}

\newcommand{\braket}[2]{\langle #1|#2 \rangle}
\newcommand{\normtwo}{\frac{1}{\sqrt{2}}}
\newcommand{\norm}[1]{\parallel #1 \parallel}
\newcommand{\card}[1]{\left| #1 \right|}
\newcommand{\ci}{\perp\!\!\!\perp}
%\DeclareMathOperator*{\argmin}{arg\,min}

\textwidth=5.6truein
\textheight=8.0truein

\renewcommand{\thefootnote}{\fnsymbol{footnote}}  %use symbolic footnote

% To fix Qcircuit target with new Xypic
\newcommand{\targfix}{\qw {\xy {<0em,0em> \ar @{ - } +<.39em,0em>
\ar @{ - } -<.39em,0em> \ar @{ - } +
<0em,.39em> \ar @{ - }
-<0em,.39em>},<0em,0em>*{\rule{.01em}{.01em}}*+<.8em>\frm{o}
\endxy}}

\input{Qcircuit}

\begin{document}
\setlength{\textheight}{8.0truein}    %FOR 2ND PAGE ONWARDS

\runninghead{A 2D Nearest-Neighbor Quantum Architecture for Factoring in Polylogarithmic Depth}
            {P. Pham and K.M. Svore}

\normalsize\textlineskip
\thispagestyle{empty}
\setcounter{page}{1}

%\copyrightheading{Vol.}{No.}{Year}{Page Nos.}
\copyrightheading{0}{0}{2003}{000--000}

\vspace*{0.88truein}

\alphfootnote

\fpage{1}

\centerline{\bf
%%%%%%%%%%%%%%%%%%%%%
%Put in titiles here
%%%%%%%%%%%%%%%%%%%%%
A 2D Nearest-Neighbor Quantum Architecture for Factoring in Polylogarithmic Depth}
\vspace*{0.37truein}
\centerline{\footnotesize
%%%%%%%%%%%%%%%%%%%%%%%%%%%%%%%%%%%%
%put authors' name and address here
%%%%%%%%%%%%%%%%%%%%%%%%%%%%%%%%%%%%
Paul Pham\footnote{ppham@cs.washington.edu}}
\vspace*{0.015truein}
\centerline{\footnotesize\it Quantum Theory Group, University of Washington, Box 352350,}
\baselineskip=10pt
\centerline{\footnotesize\it Seattle, WA 98195, USA}
\vspace*{10pt}
\centerline{\footnotesize
Krysta M. Svore\footnote{ksvore@microsoft.com}}
\vspace*{0.015truein}
\centerline{\footnotesize\it Quantum Architectures and Computation Group, Microsoft Research, One Microsoft Way,}
\baselineskip=10pt
\centerline{\footnotesize\it Redmond, WA 98052, USA}
\vspace*{0.225truein}
\publisher{(received date)}{(revised date)}

\vspace*{0.21truein}

%% \abstracts{first paragraph}{second paragraph}{third paragraph}
%% If there is only one paragraph, just keep the second and third empty
%% like the following one
\abstracts{
%%%%%%%%%%%%%%%%%%%%
% put abstract here
%%%%%%%%%%%%%%%%%%%%
We present a 2D nearest-neighbor
quantum architecture for Shor's algorithm to factor an $n$-bit number in $O(\log^3n)$ depth.
Our implementation uses
%(1)
parallel phase estimation,
%(due to Kitaev, Shen, and Vyalyi),
%(2)
constant-depth fanout and teleportation,
%(due to Harrow, Fowler, and Taylor),
and
%(3)
constant-depth carry-save modular addition.
%(due to Gossett).
%We introduce a novel 2D architectural variation on Gossett's modular arithmetic
%and interleave constant-depth fanout and teleportation circuits for
%nearest-neighbor and long-distance communication channels, and ultimately use
%our circuit within parallel phase estimation to achieve quantum factoring.
We derive upper bounds on the circuit resources of our architecture under a
new 2D model which allows a classical controller and parallel, communicating
modules.
We provide a comparison to all previous nearest-neighbor factoring
implementations.  
Our circuit results in an exponential improvement in nearest-neighbor circuit depth at the cost of a polynomial increase in circuit size and width.
}{}{}

\vspace*{10pt}

\keywords{quantum architecture, prime factorization, Shor's algorithm,
nearest-neighbor, carry-save addition}
\vspace*{3pt}
\communicate{to be filled by the Editorial}

\vspace*{1pt}\textlineskip    %) USE THIS MEASUREMENT WHEN THERE IS
   %) A SECTION HEADING
%\vspace*{-0.5pt}
%\noindent
%%%%%%%%%%%%%%%%%%%%%%%%%%%%%%%%
%put the text of the paper here
%%%%%%%%%%%%%%%%%%%%%%%%%%%%%%%%
\section{Introduction}
\label{sec:intro}

Shor's factoring algorithm is a central result in quantum computing, with an
exponential speed-up over the best known classical algorithm \cite{Shor1994}.
As the most notable example of a quantum-classical complexity separation, much
effort has been devoted to implementations of factoring on a
realistic architectural model of a quantum computer
\cite{Beauregard2002,Kutin2006,VanMeter2006,VanMeter2005,VanMeterIL2005}.
We can bridge the gap between
the theoretical algorithm and a physical implementation by describing
the layout and interactions of qubits at an intermediate,
architectural level of abstraction.
This gives us a model for measuring circuit resources and their tradeoffs.
In this work, we present a circuit implementation of prime
factorization of an $n$-bit integer
on a two-dimensional architecture that allows concurrent (parallel) two-qubit operations
between neighboring qubits, an omnipresent classical controller, and
modules which are allowed to teleport qubits to each other. We call this new
model \textsc{2D CCNTCM}.
We show that our circuit construction is asymptotically more efficient in circuit depth than previous state-of-the-art techniques for nearest-neighbor
architectures, achieving a depth of $O(\log^3 n)$, a size of
$O(n^4\log n)$, and a width of $O(n^4)$ qubits, as detailed in Table
\ref{tab:results} of Section \ref{sec:results}.

Our technique hinges on several key building blocks.
Section \ref{sec:bg} introduces quantum architectural models, circuit
resources, and constant-depth communication techniques due to
\cite{Harrow2012,Rosenbaum2012}.
Section \ref{sec:related} places our work in the context of existing
results.
In Section \ref{sec:csa}, we provide a self-contained pedagogical review
of the carry-save technique and encoding.
In Section \ref{sec:csa-mod-add} we modify and extend the carry-save technique to a 2D
modular adder,
which we then use as a basis for a modular multiplier
(Section \ref{sec:csa-mod-mult}) and a modular exponentiator
(Section \ref{sec:modexp}).
For each building block, we provide numerical upper bounds for the
required circuit resources.
Finally, we compare our asymptotic circuit resource usage
with other factoring implementations.

%\section{Background}
\label{sec:bg}

Quantum architecture is concerned with the physical layout of
qubits and constraints on their interactions,
as well as the efficient execution, in time, space, and other resources, of
algorithms on a given architecture.
In this paper, we focus on designing a realistic nearest-neighbor circuit for running
Shor's factoring algorithm on architectural models of a physical quantum device.

%%%%%%%%%%%%%%%%%%%%%%%%%%%%%%%%%%%%%%%%%%%%%%%%%%%%%%%%%%%%%%%%%%%%%%%%%%%%%%%
\subsection{Architectural Models and\\ Circuit Resources}
\label{subsec:models}

Following Van Meter \cite{VanMeter2005},
we distinguish between a model and an architectural implementation as follows.
A \emph{model} is a set of constraints and rules for the placement and
interaction of qubits.
An \emph{architecture}, or \emph{implementation}, is a particular
spatial layout of qubits (as a graph of vertices) and their
constrained interactions (edges between the vertices),
following the constraints of a given model.

The most general model is called Abstract Concurrent (\textsc{AC})
and allows arbitrary, long-range interactions between any qubits and concurrent
operation of quantum gates.
This corresponds to a complete graph with an edge between every pair of nodes,
which is the model assumed in most quantum algorithms.

A more specialized model restricts interactions to nearest-neighbor, two-qubit,
concurrent gates (\textsc{NTC}) in a regular one-dimensional chain (1D NTC),
which is sometimes called linear nearest-neighbor (\textsc{LNN}).
This corresponds to a line graph.

To relieve movement congestion,
we can extend to a two-dimensional regular grid
(2D NTC), where each
qubit has four neighbors and there is an extra degree of freedom
in which to move data.
In this paper, we extend the \textsc{2D NTC} model in two ways.
The first extension allows arbitrary planar graphs
with bounded degree, rather than a regular square lattice.
Namely, we assume qubits lie in a plane and edges are not allowed to intersect,
so that theoretically all qubits are accessible from above
or below by control and measurement apparatus.
Whereas 2D NTC conventionally assumes each qubit
has four neighbors, we consider up to six neighbors in a roughly hexagonal
layout. The second extension we make is the realistic assumption
that classical control can
access every qubit in parallel, and we do not count these classical
resources in our implementation. We call these augmented models
\textsc{CCAC} and \textsc{CCNTC} following 
\cite{Rosenbaum2012}. The classical controller
corresponds to fast digital computers which are
available in actual experiments and are necessary for constant-depth
communication in the next section.
%, and it allows us to use constant-depth
%fanout and teleportation.

We measure the efficiency of a circuit on a particular
architecture in terms of three resources:
%, depicted in Figure \ref{fig:resources}:
\emph{circuit size} (number of non-identity gates),
\emph{circuit depth} (number of time-steps), and
\emph{circuit width} (number of qubits).
For circuit depth, a two-qubit gate takes
one time-step and absorbs any adjacent single-qubit gates.
Multiple two-qubit gates on disjoint qubits
can occur in parallel during the same timestep.

%%%%%%%%%%%%%%%%%%%%%%%%%%%%%%%%%%%%%%%%%%%%%%%%%%%%%%%%%%%%%%%%%%%%%%%%%%%%%%%
\subsection{Constant-depth Teleportation\\ and Fanout}
\label{subsec:fanout}

Two key problems in nearest-neighbor architectures deal with communication,
namely moving and copying quantum information.
How can we transport quantum information at one site to another over
arbitrarily long distances?
%A related problem is how to teleport a qubit an arbitrary distance.
% in an
%architecture through ancillae prepared in some initial state.
To solve this problem, we employ the constant-depth teleportation circuit
shown in part (a) of Figure \ref{fig:cd}, using standard quantum circuit
notation from \cite{Nielsen2000}.

\begin{figure*}[tb!]
\begin{center}
\begin{displaymath}
\begin{array}{ccc}
\Qcircuit @C=1em @R=1em {
\lstick{\ket{\psi}}	& \qw      & \qw      & \qw & \qw & \qw & \qw & \qw                                          & \qw & \qw & \multimeasureD{1}{\mbox{Bell}} & \cw & \rstick{j_1} \\
\lstick{\ket{0}}    & \gate{H} & \ctrl{1} & \qw & \qw & \qw & \qw & \qw                                          & \qw & \qw & \ghost{\mbox{Bell}}            & \cw & \rstick{k_1} \\
\lstick{\ket{0}}    & \qw      & \targfix & \qw & \qw & \qw & \qw & \qw_{X^{k_1}Z^{j_1}\ket{\psi}}               & \qw & \qw & \multimeasureD{1}{\mbox{Bell}} & \cw & \rstick{j_2} \\
\lstick{\ket{0}}    & \gate{H} & \ctrl{1} & \qw & \qw & \qw & \qw & \qw                                          & \qw & \qw & \ghost{\mbox{Bell}}            & \cw & \rstick{k_2} \\
\lstick{\ket{0}}    & \qw      & \targfix & \qw & \qw & \qw & \qw & \qw_{X^{k_2}Z^{j_2}X^{k_1}Z^{j_1}\ket{\psi}} & \qw & \qw & \multimeasureD{1}{\mbox{Bell}} & \cw & \rstick{j_3} \\
\lstick{\ket{0}}    & \gate{H} & \ctrl{1} & \qw & \qw & \qw & \qw & \qw                                          & \qw & \qw & \ghost{\mbox{Bell}}            & \cw & \rstick{k_3} \\
\lstick{\ket{0}}    & \qw      & \targfix & \qw & \qw & \qw & \qw & \qw & \qw_{X^{k_3}Z^{j_3}X^{k_2}Z^{j_2}X^{k_1}Z^{j_1}\ket{\psi}} & \qw & \qw              & \qw & \qw \\
}
& \qquad \qquad \qquad &
\Qcircuit @C=1em @R=1em {
\lstick{\ket{\psi}}	& \qw      & \qw      & \qw & \ctrl{1} & \qw & \qw      & \qw & \rstick{\ket{\ell}}\\
\lstick{\ket{0}}	& \qw      & \qw      & \qw & \targfix & \qw & \multimeasureD{1}{\mbox{Bell}} & \cw & \rstick{j_1} \\
\lstick{\ket{0}}    & \gate{H} & \ctrl{1} & \qw & \qw      & \qw & \ghost{\mbox{Bell}}            & \cw & \rstick{k_1} \\
\lstick{\ket{0}}    & \qw      & \targfix & \qw & \ctrl{1} & \qw & \qw      & \qw & \rstick{X^{k_1}Z^{j_1}\ket{\ell}}\\
\lstick{\ket{0}}	& \qw      & \qw      & \qw & \targfix & \qw & \multimeasureD{1}{\mbox{Bell}} & \cw & \rstick{j_2} \\
\lstick{\ket{0}}    & \gate{H} & \ctrl{1} & \qw & \qw      & \qw & \ghost{\mbox{Bell}}           & \cw & \rstick{k_2} \\
\lstick{\ket{0}}    & \qw      & \targfix & \qw & \ctrl{1} & \qw & \qw      & \qw & \rstick{X^{k_2}Z^{j_2}\ket{\ell}}\\
\lstick{\ket{0}}	& \qw      & \qw      & \qw & \targfix & \qw & \multimeasureD{1}{\mbox{Bell}} & \cw & \rstick{j_3} \\
\lstick{\ket{0}}    & \gate{H} & \ctrl{1} & \qw & \qw      & \qw & \ghost{\mbox{Bell}}           & \cw & \rstick{k_3} \\
\lstick{\ket{0}}    & \qw      & \targfix & \qw & \ctrl{1} & \qw & \qw      & \qw & \rstick{X^{k_3}Z^{j_3}\ket{\ell}}\\
\lstick{\ket{0}}	& \qw      & \qw      & \qw & \targfix & \qw & \qw      & \qw & \rstick{X^{k_3}Z^{j_3}\ket{\ell}}\\
}\\
& & \\
(a) & & (b)
\end{array}
\end{displaymath}
\centerline{
}
\caption{Constant-depth circuits based on \cite{Broadbent2007,Browne2009} for
(a) teleportation \protect{\cite{Rosenbaum2012}} and
(b) fanout \cite{Harrow2012}.}
\label{fig:cd}
\end{center}\end{figure*}

The second problem is copying information. Although general cloning is
impossible \cite{Nielsen2000}, we only need to perform unbounded quantum
fanout, the operation
$\ket{x,y_1,\ldots,y_n} \rightarrow \ket{x,y_1\oplus x, \ldots, y_n\oplus x}$.
This is used in our arithmetic circuits when
a single qubit needs to control (be entangled with) a large quantum register
(called a \emph{fanout rail}).
We employ a constant-depth circuit due to insight from
measurement-based quantum computing \cite{Raussendorf2003}
that relies on the creation of an
$n$-qubit cat state \cite{Browne2009}.
It requires $O(1)$-depth, $O(n)$-size, and $O(n)$-width, and is shown in
part (b) of Figure \ref{fig:cd} for the case of fanning out $\ket{\psi}$ to
four qubits.
The technique works by creating multiple small
cat states of a fixed size (in this case, three qubits) and linking them
together with Bell measurements. The qubits marked $\ket{\ell}$ are
entangled into a (slightly) larger cat state, up to Pauli corrections.
%
\begin{equation}
\normtwo X_1^{k_1}Z_1^{j_1}X_2^{k_2}Z_2^{j_2}X_{3}^{k_3}X_{4}^{k_3}Z_{3}^{j_3}Z_{4}^{j_3}
\left( \ket{0000} + \ket{1111} \right)
\end{equation}
%
The operators $X^k_i$ and $Z^j_{\ell}$ denote Pauli $X$ and $Z$ operators
on qubits $i$ and $\ell$, controlled by classical bits $k$ and $j$,
respectively. These corrections are enacted by the classical controller based on
the Bell measurement outcomes (not depicted). Unfortunately, this
``consumes'' the cat state in that there is no known way to unentangle the
source qubit from the cat state after they have been jointly measured \cite{Rosenbaum2012}.

\section{Background}
\label{sec:bg}

Quantum architecture is concerned with the physical layout of
qubits and constraints on their interactions,
as well as the efficient execution, in time, space, and other resources, of
algorithms on a given architecture.
In this paper, we focus on designing a realistic nearest-neighbor circuit for running
Shor's factoring algorithm on architectural models of a physical quantum device.

%%%%%%%%%%%%%%%%%%%%%%%%%%%%%%%%%%%%%%%%%%%%%%%%%%%%%%%%%%%%%%%%%%%%%%%%%%%%%%%
\subsection{Architectural Models and Circuit Resources}
\label{subsec:models}

Following Van Meter and Itoh \cite{VanMeter2005},
we distinguish between a model and an architectural implementation as follows.
A \emph{model} is a set of constraints and rules for the placement and
interaction of qubits.
An \emph{architecture} (or interchangeably, an \emph{implementation} 
or a \emph{circuit}) is a particular
spatial layout of qubits (as a graph of vertices) and their
constrained interactions (edges between the vertices),
following the constraints of a given model. In this section, we describe
several models which try to incorporate resources of physical interest from
experimental work. We also introduce a new model,
\textsc{2D CCNTCM}, which we will use to analyze our current circuit.

The most general model is called Abstract Concurrent (\textsc{AC})
and allows arbitrary, long-range interactions between any qubits and concurrent
operation of quantum gates.
This corresponds to a complete graph with an edge between every pair of nodes,
and is the model assumed in most quantum algorithms.

A more specialized model restricts interactions to nearest-neighbor, two-qubit,
concurrent gates (\textsc{NTC}) in a regular one-dimensional chain (1D NTC),
which is sometimes called linear nearest-neighbor (\textsc{LNN}).
This corresponds to a line graph. This is a more realistic model than
\textsc{AC}, but correspondingly, circuits in this model may incur greater
resource overheads.

To relieve movement congestion,
we can consider a two-dimensional regular grid
(2D NTC), where each
qubit has four planar neighbors and 
there is an extra degree of freedom
in which to move data.
In this paper, we extend the \textsc{2D NTC} model in three ways.
The first two extensions are described in Section \ref{subsec:2dccntc},
and the third extension is described in Section \ref{subsec:2dccntcm}.

\subsection{\textsc{2D CCNTC}: Two-Dimensional Nearest-Neighbor Two-Qubit Concurrent Gates}
\label{subsec:2dccntc}

The first extension allows arbitrary planar graphs
with bounded degree, rather than a regular square lattice.
Namely, we assume qubits lie in a plane and edges are not allowed to intersect,
so that theoretically all qubits are accessible from above
or below by control and measurement apparatus.
Whereas 2D NTC conventionally assumes each qubit
has four neighbors, we consider up to six neighbors in a roughly hexagonal
layout. The second extension is the realistic assumption
that classical control (CC) can
access every qubit in parallel, and we do not count these classical
resources in our implementation since they are polynomial-time. The classical controllers
correspond to fast digital computers which are
available in actual experiments and are necessary for constant-depth
communication in the next section.


We call an AC or NTC model augmented by these two extensions
\textsc{CCAC} and \textsc{CCNTC}, respectively. Before we describe the
third extension, let us formalize our model for \textsc{2D CCNTC}, with definitions that are (asymptotically) equivalent to those in 
\cite{Rosenbaum2012}.

\begin{definition}
A 2D CCNTC architecture consists of

\begin{itemize}
\item a quantum computer $QC$ which is represented by a planar graph $(V,E)$. A
node $v \in V$ represents a qubit which is acted upon in a circuit, and an
undirected edge $(u,v) \in E$ represents 
allowed two-qubit interactions between qubits $u,v \in V$. Each node has
degree at most $6$.
\item a universal gate set $\mathcal{G} = {X, Z, H, T, T^{\dagger}, CNOT, Measure-Z}$.

\item a deterministic machine (classical controller) $CC$ that applies a sequence
of concurrent gates in each of $D$ timesteps $\{t_1, \ldots, t_D\}$.
\item In timestep $i$, $CC$ applies gates $G_i = \{g_{i,j} \in \mathcal{G} \}$.
Each $g_{i,j}$ operates in one of the following two ways:
\begin{enumerate}
\item a gate from $\mathcal{G}$ on a single qubit $v_{i,j} \in V$
\item
the gate CNOT from $\mathcal{G}$ on two qubits $v^{(1)}_{i,j}$ and $v^{(2)}_{i,j}$ where
$(v^{(1)}_{i,j}, v^{(2)}_{i,j}) \in E$
\end{enumerate}
All $g_{i,j}$ can only operate on
disjoint qubits for a given timestep $i$. We define the support of $G_i$
as $V_i$, the set of all qubits acted upon by any $g_{i.j}$.
\end{itemize}
\end{definition}

We can then define the three conventional circuit resources in this model.

\begin{description}
\item[circuit depth ($D$):] is the number of concurrent timesteps.
\item[circuit size ($S$):] is the total number of non-identity gates applies
from $\mathcal{G}$, equal to $\sum_{i=1}^D |G_i|$.
\item[circuit width ($W$):] is the total number of qubits operated upon by
any gate, including inputs, outputs, and ancillae. It is equal to $| \bigcup_{i=1}^D V_i$|.
\end{description}

We observe that the following relationship holds between the circuit resources.
The circuit size is bounded above by
the product of circuit depth and circuit width, since in the worst case,
every qubit is acted upon by a gate for every timestep of a circuit.
The circuit depth is also bounded above by the size, since in the worst case,
every gate is executed serially without any concurrency.

\begin{equation}
D \le S \le D\cdot W
\label{eqn:depth-width}
\end{equation}

The set $\mathcal{G}$ includes measurement in the $Z$ basis. In this paper we
will treat the operations in $\mathcal{G}$ as \emph{elementary gates}.
All other gates
in $\mathcal{G}$ form a universal set of unitary
gates \cite{Kitaev2002}.

We also find it useful to define a Bell basis measurement using operations
from $\mathcal{G}$. A circuit performing this measurement is shown
in Figure \ref{fig:bell-measure} and has depth $4$,
size $4$, and width $2$.

\begin{figure*}[tb!]
\begin{center}
\begin{displaymath}
\begin{array}{ccc}
\Qcircuit @C=1em @R=1em {
& \qw & \multimeasureD{1}{\mbox{Bell}} & \cw & \rstick{j} \\
& \qw & \ghost{\mbox{Bell}}            & \cw & \rstick{k}
}
& \qquad \equiv \qquad &
\Qcircuit @C=1em @R=1em {
& \qw & \ctrl{1} & \qw & \gate{H} & \qw & \meter & \cw & \rstick{j} \\
& \qw & \targfix & \qw & \qw      & \qw & \meter & \cw & \rstick{k}
}
\end{array}
\end{displaymath}
\centerline{}
\fcaption{A circuit for measurement in the Bell state basis.}
\label{fig:bell-measure}
\end{center}\end{figure*}

The third extension to our model, and the most significant, is to consider
multiple disconnected planar graphs, each of which is a 2D CCNTC
architecture. This is described in more detail in the next section.

\subsection{\textsc{2D CCNTCM}: Two-Dimensional Nearest-Neighbor Two-Qubit Concurrent Gates with Modules}
\label{subsec:2dccntcm}

A single, contiguous
2D lattice which contains an entire quantum architecture which may be prohibitively large to manufacture. In practice,
scalable experiments will probably use many
smaller quantum computers which communicate by means of shared
entanglement \cite{Monroe2012}.
We call these individual machines \emph{modules}, each of
which is a self-contained \textsc{2D CCNTC} lattice. This should not be
confused with the word ``modular'' as in ``modular arithmetic'' or as
referring to the modulus $m$ which we are trying to factor.

We treat these modules
and allowed teleportations between them as nodes and edges, respectively,
in a higher-level planar graph. The teleportations each transmit one qubit
from one module to another, from any location within the source module and
to any location within the destination module, making using of the
omnipresent classical controller. The modules can be arbitrarily far
apart and have arbitrary
connectivity with other modules.

Teleportations can occur to
intersecting modules, in contrast to concurrent gates operating on
qubits in NTC. We justify this assumption in that it is
possible to establish entanglement between multiple
quantum computers
in parallel. We call this new model \textsc{2D CCNTCM},
and we argue that is captures the essential aspects of 2D architectures
without being overly sensitive to the exact geometry of the lattices involved.
An graphic depiction of three modules in \textsc{2D CCNTCM} is shown in
Figure \ref{fig:modules}. Each module contains within it a
\textsc{2D CCNTC} lattice.

It is notable that none of the classical controllers for each
module need to communicate with each other (through classical
channels) or with the classical controller that teleports
in between modules.

\begin{figure}[btp!]
\begin{center}
\includegraphics[width=5in]{./modules.pdf}
\end{center}
\fcaption{Three modules in the \textsc{2D CCNTCM} model}
\label{fig:modules}
\end{figure}

\begin{definition}
A \textsc{2D CCNTCM} architecture consists of

\begin{itemize}
\item a quantum computer $QC$ which is represented by a planar graph $(\overline{V},\overline{E})$. A
node $\overline{v} \in \overline{V}$ represents a module, or a graph $(V,E)$
from a \textsc{2D CCNTC} architecture defined previously. It can have
unbounded degree.
An
undirected edge $(\overline{u},\overline{v}) \in \overline{E}$ represents an
allowed teleportation from any qubit in module $\overline{u}$ to
module $\overline{v}$.
\item All modules are restricted to be linear in the number of their qubits:
$|V| = O(n) \forall (V,E) \in \overline{V}$.
\item a universal gate set $\mathcal{G} = {X, Z, H, T, T^{\dagger}, CNOT}$
for the qubits \emph{within the same} modules which is the same as for \textsc{2D CCNTC},
and an additional operation $Teleport$ which only operates on qubits
\emph{in
different modules}.
\item a deterministic machine (classical controller) $\overline{CC}$ that applies a sequence
of concurrent gates in each of $D+\overline{D}$ timesteps $\{t_1, \ldots, t_{D+\overline{D}}\}$. This can be a separate classical controller
for every pair of modules; no communication between them is
necessary.
\item In timestep $i$, $\overline{CC}$ applies
gates $G_i = \{g_{i,j} : g_{i,j} \in \mathcal{G} \lor g_{i,j} = Teleport \}$.
That is, within each timestep gates operate in one of two ways.
\begin{enumerate}
\item They are exclusively from $\mathcal{G}$ operating within modules, as described
for \textsc{2D CCNTC} above. We say there are $D$ such timesteps.
\item They are exclusively $Teleport$ gates between two qubits $v^{(1)}_{i,j} \in \overline{v}_1$ and $v^{(2)}_{i,j} \in \overline{v}_2$ for
(possibly non-distinct) modules $\overline{v}_1, \overline{v}_2 \in \overline{V}$.
Again, all such qubits much be distinct within a timestep.
We say there are $\overline{D}$ such timesteps.
\end{enumerate}

Again, we define the support of $G_i$
as $V_i$, the set of all qubits acted upon by any $g_{i.j}$, which
includes all the modules.
\end{itemize}
\end{definition}

We measure the efficiency of a circuit in this new module using not just
the three conventional circuit resources, but with three novel resources
based on modules.

\begin{description}

%, depicted in Figure \ref{fig:resources}:
\item[module depth ($\overline{D}$):] the depth of consecutive teleportations between modules
\item[module size ($\overline{S}$):] the number of total qubits teleported between any two modules over all timesteps.
\item[module width ($\overline{W}$):] the number of modules whose qubits are
acted upon during any timestep.

\end{description}

%We can make an observation analogous to Equation \ref{eqn:depth-width} but
%for modules in Equation \ref{eqn:module-depth-width}.

%\begin{equation}
%\overline{D} \le \overline{S} \le \overline{D}\cdot \overline{W}
%\label{eqn:module-depth-width}
%\end{equation}

We note the following relationship between circuit width and
module width.

\begin{equation}
W = O(n\overline{W})
\label{eqn:module-width}
\end{equation}

\subsection{Circuit Resource Comparisons}

Counting gates from $\mathcal{G}$ has having unit size and unit depth
is
an overestimate compared to the model in \cite{Kutin2006}, in which a
two-qubit gate has unit size and unit depth and
absorbs the depth and size of any adjacent single-qubit gates. We intend
for this more pessimistic estimate to reflect the practical difficulties
in compiling these gates using a non-Clifford gate in a fault-tolerant way,
such as the $T$ gate or the Toffoli gate
\cite{Fowler2011}.
%However, these difficulties may be mitigated by using
%Toffoli gates directly, which can be fault-tolerantly implemented using
%magic-state distillation according to recent works \cite{Eastin2012,Jones2013a}.

In both our resource counting method and that of \cite{Fowler2004,Kutin2006}, multiple gates acting on disjoint qubits
can occur in parallel during the same timestep. For each building block,
from modular addition to modular multiplication and finally to modular
exponentiation, we provide closed form equations for the required circuit
resources as a function of $n$, the size of the modulus $m$ to be factored.
These formulae include numerical constants and provide an upper bound.
We will use the
term \emph{numerical upper bound} to distinguish these formulae from asymptotic
upper bounds.

It is possible to reduce the numerical constants with more detailed analysis,
which would be important for any physical implementation.
However, we have chosen instead to simplify the number of terms in the formulae
for the current work. We do not intend for these upper bounds to represent
the optimal or final work in this area.

The modular adder in Section \ref{sec:csa-mod-add} and its carry-save
subcomponents only occur within a single module, so we only give their
circuit resources in terms of circuit depth, circuit size, and circuit width. 
For the modular multiplier in
Section \ref{sec:csa-mod-mult} and the modular exponentiator in
Section \ref{sec:modexp}, we also give circuit resources in
terms of module depth, module size, and module width.

%%%%%%%%%%%%%%%%%%%%%%%%%%%%%%%%%%%%%%%%%%%%%%%%%%%%%%%%%%%%%%%%%%%%%%%%%%%%%%%
\subsection{Constant-depth Teleportation and Fanout}
\label{subsec:fanout}

Communication, namely the \emph{moving} and \emph{copying} of quantum information, in nearest-neighbor quantum architectures is challenging.
The first challenge of moving quantum information from one site to another over
arbitrarily long distances can be addressed by using
%A related problem is how to teleport a qubit an arbitrary distance.
% in an
%architecture through ancillae prepared in some initial state.
the constant-depth teleportation circuit
shown in Figure \ref{fig:cdt}, illustrated using standard quantum circuit
notation \cite{Nielsen2000}. This requires the circuit resources shown in
Table \ref{tab:cd-resources}. The depth includes a layer of $H$ gates; a layer of CNOTs; an interleaved layer of Bell basis measurements; and two layers of
Pauli corrections ($X$ and $Z$ for each qubit), occurring concurrently with
resetting the $\ket{j}$ and $\ket{k}$ qubits back to $\ket{0}$.
These correction layers are not shown in the circuit.

\begin{figure*}[tb!]
\begin{center}
\begin{displaymath}
%\begin{array}{ccc}
\Qcircuit @C=1em @R=1em {
\lstick{\ket{\psi}}	& \qw      & \qw      & \qw & \qw & \qw & \qw & \qw                                          & \qw & \qw & \multimeasureD{1}{\mbox{Bell}} & \cw & \rstick{j_1} \\
\lstick{\ket{0}}    & \gate{H} & \ctrl{1} & \qw & \qw & \qw & \qw & \qw                                          & \qw & \qw & \ghost{\mbox{Bell}}            & \cw & \rstick{k_1} \\
\lstick{\ket{0}}    & \qw      & \targfix & \qw & \qw & \qw & \qw & \qw_{Z^{j_1}X^{k_1}\ket{\psi}}               & \qw & \qw & \multimeasureD{1}{\mbox{Bell}} & \cw & \rstick{j_2} \\
\lstick{\ket{0}}    & \gate{H} & \ctrl{1} & \qw & \qw & \qw & \qw & \qw                                          & \qw & \qw & \ghost{\mbox{Bell}}            & \cw & \rstick{k_2} \\
\lstick{\ket{0}}    & \qw      & \targfix & \qw & \qw & \qw & \qw & \qw_{Z^{j_2}Z^{j_1}X^{k_2}X^{k_1}\ket{\psi}} & \qw & \qw & \multimeasureD{1}{\mbox{Bell}} & \cw & \rstick{j_3} \\
\lstick{\ket{0}}    & \gate{H} & \ctrl{1} & \qw & \qw & \qw & \qw & \qw                                          & \qw & \qw & \ghost{\mbox{Bell}}            & \cw & \rstick{k_3} \\
\lstick{\ket{0}}    & \qw      & \targfix & \qw & \qw & \qw & \qw & \qw & \qw_{Z^{j_1}Z^{j_2}Z^{j_3}X^{k_3}X^{k_2}X^{k_1}\ket{\psi}} & \qw & \qw              & \qw & \qw \\
}
\end{displaymath}
\centerline{}
\fcaption{Constant-depth circuit based on \protect{\cite{Broadbent2007,Browne2009}} for teleportation over $n=5$ qubits \protect{\cite{Rosenbaum2012}}.}
\label{fig:cdt}
\end{center}\end{figure*}

\begin{figure*}[tb!]
\begin{center}
\begin{displaymath}
%& \qquad \qquad \qquad &
\Qcircuit @C=1em @R=1em {
\lstick{\ket{\psi}}	& \qw      & \qw      & \qw & \qw & \qw & \multimeasureD{1}{\mbox{Bell}'} & \cw & \rstick{j_1} \\
\lstick{\ket{0}}    & \gate{H} & \ctrl{1} & \qw & \qw      & \qw & \ghost{\mbox{Bell}'}            & \cw & \rstick{k_1} \\
\lstick{\ket{0}_1}    & \qw      & \targfix & \qw & \ctrl{1} & \qw & \qw      & \qw & \rstick{Z^{j_1}X^{k_1}\ket{\ell}_1}\\
\lstick{\ket{0}}	& \qw      & \qw      & \qw & \targfix & \qw & \multimeasureD{1}{\mbox{Bell}} & \cw & \rstick{j_2} \\
\lstick{\ket{0}}    & \gate{H} & \ctrl{1} & \qw & \qw      & \qw & \ghost{\mbox{Bell}}           & \cw & \rstick{k_2} \\
\lstick{\ket{0}_2}    & \qw      & \targfix & \qw & \ctrl{1} & \qw & \qw      & \qw & \rstick{Z^{j_2}X^{k_2}X^{k_1}\ket{\ell}_2}\\
\lstick{\ket{0}}	& \qw      & \qw      & \qw & \targfix & \qw & \multimeasureD{1}{\mbox{Bell}} & \cw & \rstick{j_3} \\
\lstick{\ket{0}}    & \gate{H} & \ctrl{1} & \qw & \qw      & \qw & \ghost{\mbox{Bell}}           & \cw & \rstick{k_3} \\
\lstick{\ket{0}_3}    & \qw      & \targfix & \qw & \ctrl{1} & \qw & \qw      & \qw & \rstick{Z^{j_3}X^{k_3}X^{k_2} X^{k_1}\ket{\ell}_3}\\
\lstick{\ket{0}_4}	& \qw      & \qw      & \qw & \targfix & \qw & \qw      & \qw & \rstick{X^{k_3}X^{k_2} X^{k_1}\ket{\ell}_4}\\
}
%& & \\
%(a) & & (b)
%\end{array}
\end{displaymath}
\centerline{}
\fcaption{Constant-depth circuits based on \protect{\cite{Broadbent2007,Browne2009}} for fanout \protect{\cite{Harrow2012}} of one qubit to $n=4$ entangled copies.}
\label{fig:cdf}
\end{center}\end{figure*}

Although general cloning is
impossible \cite{Nielsen2000}, the second challenge of copying information can be addressed by performing an unbounded quantum
fanout operation:
$\ket{x,y_1,\ldots,y_n} \rightarrow \ket{x,y_1\oplus x, \ldots, y_n\oplus x}$.
This is used in our arithmetic circuits when
a single qubit needs to control (be entangled with) a large quantum register
(called a \emph{fanout rail}).
We employ a constant-depth circuit due to insight from
measurement-based quantum computing \cite{Raussendorf2003}
that relies on the creation of an
$n$-qubit cat state \cite{Browne2009}.

It requires $O(1)$-depth, $O(n)$-size, and $O(n)$-width. Approximately
two-thirds of the ancillae are reusable and can be reset to $\ket{0}$ after
being measured. Numerical upper bounds are given in Table \ref{tab:cd-resources}.
The constant-depth fanout circuit is shown in Figure \ref{fig:cdf} for the case of fanning out a given single-qubit state
$\ket{\psi} = \alpha\ket{0} + \beta\ket{1}$ to four qubits.
The technique works by creating multiple small
cat states of a fixed size (in this case, three qubits), linking them
together into a larger cat state of unbounded size with Bell basis measurements,
and finally entangling them with the source qubit to be fanned out.
The qubits marked $\ket{\ell}$ are
entangled into the larger fanned out state given in Equation \ref{eqn:cat4}.
The Pauli corrections from the cat state creation are denoted by
$X^{k_2}$, $X^{k_3}$, $Z^{j_2}$ and $Z^{j_3}$ on qubits beginning in state $\ket{0}_1$, $\ket{0}_2$,
$\ket{0}_3$, and $\ket{0}_4$. The Pauli corrections
$X^{k_1}$ and $Z^{j_1}$ are from the Bell basis measurement
entangling the cat state with the source qubit (denoted $\text{Bell}'$).
\begin{equation}
Z_1^{j_1}X_1^{k_1}Z_2^{j_2}X_2^{k_2}X_2^{k_1}Z_{3}^{j_3}X_{3}^{k_3}X_{3}^{k_2}X_{3}^{k_1}X_{4}^{k_3}X_{4}^{k_2}X_{4}^{k_1}
\left(\alpha \ket{0}_1\ket{0}_2\ket{0}_3\ket{0}_4 + \beta \ket{1}_1\ket{1}_2\ket{1}_3\ket{1}_4 \right)
\label{eqn:cat4}
\end{equation}
%
The operators $X^k_i$ and $Z^j_{h}$ denote Pauli $X$ and $Z$ operators
on qubits $i$ and $h$, controlled by classical bits $k$ and $j$,
respectively. These corrections are enacted by the classical controller based on
the Bell measurement outcomes (not depicted).
Note the cascading nature of these corrections.
There can be up to
$n-1$ of these $X$ and $Z$
corrections on the same qubit, which can be simplified by the classical
controller to a single $X$ and $Z$ operation and then applied with a circuit of
depth 2 and size 2. Also, given the symmetric nature of the cat state, there
is an alternate set of Pauli corrections which would give the same state and
is of equal size to corrections given above.

Reversing the fanout (un-fanout) in constant depth is an interesting
open problem. Doing so would allow us to improve the overall depth of our
factoring implementation to $O(\log^2 n)$ instead of $O(\log^3 n )$.
In this work it is sufficient to perform alternating rounds of
teleportation and CNOT among the $n$ fanned-out qubits in a logarithmic-depth
binary tree. The resources for this are given in
Table \ref{tab:cd-resources}.

% From Notebook #16, p. 212
% From Notebook #16, p. 66
\begin{table}
\begin{displaymath}
\begin{tabular}{|c|c|c|c|}
\hline
\text{Circuit Name} & \text{Depth} & \text{Size} & \text{Width}\\
\hline
\text{Teleportation from Figure \ref{fig:cdt}} & 7 & 3n + 4 & n+1\\
\hline
\text{Fanout from Figure \ref{fig:cdf}} & 9 & 10n - 9 & 3n-1 \\
\hline
\text{Un-fanout} & $8\log_2(2n)$ & $33n\log_2(2n) + 10\log^2_2(2n)$ & $3n-1$ \\
\hline
\end{tabular}
\end{displaymath}
\centerline{}
\tcaption{Circuit resources for teleportation, fanout, and un-fanout
(consisting of
alternating rounds of constant-depth teleportation and CNOT).}
\label{tab:cd-resources}
\end{table}

From an experimental perspective, it is physically efficient to create
a cat state in trapped ions using the M{\o}lmer-S{\o}rensen gate
\cite{Sorensen2000}\cite{Benhelm2008}. However, the fanout circuit for
the 2D CCNTCM model would still be useful or other technologies, such
as superconducting qubits on a two-dimensional lattice.

%Unfortunately, this
%``consumes'' the cat state in that there is no known way to unentangle the
%source qubit from the cat state after they have been jointly measured \cite{Rosenbaum2012}.

%\section{Related Work}
\label{sec:related}

We extend the body of work which applies classical ideas to
quantum logic. Gossett \cite{Gossett1998} uses carry-save techniques to add
numbers in constant-depth and multiply in logarithmic-depth
using a special encoding, but at a quadratic
cost in qubits (circuit width). The underlying idea of encoded adding, sometimes
called a 3-2 adder, derives from Wallace trees \cite{Wallace1964}.

Choi and Van Meter are the first to discuss 2D architectures by designing an
adder that runs in $\Theta(\sqrt{n})$-depth on \textsc{2D NTC} \cite{Choi2010}
using $O(n)$-qubits with dedicated, special-purpose areas of a physical
circuit layout.

Takahashi and Kunihiro have also discovered a linear-depth
and linear-size adder using zero ancillae \cite{Takahashi2005}, and also
an adder with variable tradeoffs between $O(n/d(n))$ ancillae and
$O(d(n))$-depth for $d(n) = \Omega(\log n)$ \cite{Takahashi2009} which has
better width but worse depth than our adder. This approach assumes unbounded
fanout, which has not been mapped to a
nearest-neighbor circuit until the current work.

Once an adder implementation is chosen, it can be extended to perform
modular reduction, modular multiplication, modular
exponentiation, and ultimately
quantum period finding (QPF), the only quantum part of the factoring algorithm.
Since Shor's algorithm is a probabilistic algorithm, requiring several rounds of 
QPF to amplify success probability, it suffices to determine the resources
required for a single round of QPF with a fixed, modest success probability.
The original approach to QPF performs controlled
modular exponentiation followed by an inverse quantum Fourier transform
(QFT) \cite{Nielsen2000}. We will call this \emph{serial QPF}.

This is the approach taken by all other factoring (QPF)
implementations on any architectural model before the current work.
For example, Beauregard \cite{Beauregard2002} uses this QPF approach
to construct
a cubic-depth quantum period-finder using only $2n+3$ qubits on AC,
by combining the ideas of Draper's transform adder \cite{Draper2000},
Vedral et al.'s modular arithmetic blocks \cite{Vedral1996}, and a
semi-classical QFT.
This approach was subsequently adapted to \textsc{1D NTC} by Fowler, Devitt,
and Hollenberg
\cite{Fowler2004} to achieve exact resource counts for an $O(n^3)$-depth
quantum period-finder. Kutin \cite{Kutin2006} later improved this using
an idea from Zalka for approximate multipliers to get a QPF circuit on
\textsc{1D NTC}
in $O(n^2)$-depth. Thus, there is only a constant overhead from
Zalka's own factoring implementation on \textsc{AC}, also in quadratic depth
\cite{Zalka1998}.
Takahashi and Tani extend their earlier $O(n)$-depth adder to a factoring
circuit in $O(n^3)$-depth but with linear width.

All these works assume qubits are expensive (width) and that
execution time (depth) is not the limiting constraint.
We compare our work primarily against Kutin's method, and we make the
alternative assumption that ancillae are cheap and that fast classical control
is available which can access all qubits in parallel. Therefore, we optimize
circuit depth at the expense of width.

Serial QPF is depth-limited by having to the perform an inverse QFT.
On an AC architecture, even when approximating the (inverse) QFT
by truncating two-qubit
$\pi/2^k$ rotations beyond
$k = O(\log n)$, the depth is $O(n \log n)$ for factoring $n$-bit numbers.
There is an alternative, parallel version of phase estimation described in
Section 13 of \cite{Kitaev2002}, which decreases depth in exchange
for increased width and additional classical post-processing.
This eliminates the need to do an inverse QFT.
We refer the reader to \cite{Kitaev2002} and \cite{Pham2011b} for details.
Our factoring scheme employs our 2D quantum arithmetic circuits and this
\emph{parallel QPF}, and we will show that it is asymptotically more
efficient than the other QPF method. We compare the circuit resources
required by our work with the serial QPF implementations above in Table
\ref{tab:results} of Section \ref{sec:results}.

Recent results by Browne, Kashefi, and Perdrix (BKP) connect the power of
measurement-based quantum computing to the quantum circuit model augmented with
unbounded fanout \cite{Browne2009}. Their model, which we adapt and call
\textsc{CCNTC}, uses the classical controller mentioned in \ref{subsec:fanout}.
They describe a constant-depth circuit for
exact factoring, improving on a constant-depth circuit for approximate factoring
by H{\o}yer and {\v S}palek \cite{Hoyer2002}.
A direction for future work is to determine how our approach compares to the
BKP result in terms of circuit size and width.

%%%%%%%%%%%%%%%%%%%%%%%%%%%%%%%%%%%%%%%%%%%%%%%%%%%%%%%%%%%%%%%%%%%%%%%%%%%%%%
\section{Related Work}
\label{sec:related}

Our work builds upon ideas in classical digital and reversible logic and their extension to quantum logic.
A circuit implementation for Shor's algorithm requires a quantum adder.
Gossett proposed a quantum algorithm for addition using classical carry-save techniques to add
in constant-depth and multiply in logarithmic-depth, with a quadratic
cost in qubits (circuit width) \cite{Gossett1998}. The techniques relies on encoded addition, sometimes
called a 3-2 adder, and derives from classical Wallace trees \cite{Wallace1964}.

Takahashi and Kunihiro discovered a linear-depth
and linear-size quantum adder using zero ancillae \cite{Takahashi2005}.
They also developed an adder with tradeoffs between $O(n/d(n))$ ancillae and
$O(d(n))$-depth for $d(n) = \Omega(\log n)$ \cite{Takahashi2009}. 
Their approach assumes unbounded fanout, which had not previously been mapped to a
nearest-neighbor circuit until our present work.

Studies of architectural constraints, namely restriction to a 2D planar layout, 
were experimentally motivated. For example, these layouts correspond
to early ion trap proposals \cite{Kielpinski2002}
and were later analyzed at the level of physical qubits and error correction in the context of Shor's algorithm \cite{Kubi09}.
Choi and Van Meter designed one of the first adders targeted to a 2D architecture 
and showed it runs in $\Theta(\sqrt{n})$-depth on \textsc{2D NTC} \cite{Choi2010}
using $O(n)$-qubits with dedicated, special-purpose areas of a physical
circuit layout.

%Once an adder implementation is chosen, it can be extended 
%To perform modular reduction, modular multiplication, 
%Modular exponentiation, and ultimately
%quantum period finding (QPF), the only quantum part of the factoring algorithm.
Modular exponentiation is a key component of quantum period-finding (QPF),
and its efficiency relies on that of its underlying adder implementation.
Since Shor's algorithm is a probabilistic algorithm, several rounds of
QPF are required to amplify success probability above a constant fraction.
It suffices to determine the resources
required for a single round of QPF with a fixed, modest success probability
(in the current work, $3/4$).

The most common approach to QPF performs controlled
modular exponentiation followed by an inverse quantum Fourier transform
(QFT) \cite{Nielsen2000}. We will call this \emph{serial QPF}, which is
used by the following implementations.
%Quantum circuits proposed for factoring on a nearest-neighbor architecture have assumed a serial QPF circuit.

Beauregard \cite{Beauregard2002}
constructs a cubic-depth quantum period-finder using only $2n+3$ qubits on AC,
by combining the ideas of Draper's transform adder \cite{Draper2000},
Vedral et al.'s modular arithmetic blocks \cite{Vedral1996}, and a
semi-classical QFT.
This approach was subsequently adapted to \textsc{1D NTC} by Fowler, Devitt,
and Hollenberg
\cite{Fowler2004} to achieve resource counts for an $O(n^3)$-depth
quantum period-finder. Kutin \cite{Kutin2006} later improved this using
an idea from Zalka for approximate multipliers to produce a QPF circuit on
\textsc{1D NTC}
in $O(n^2)$-depth. Thus, there is only a constant overhead from
Zalka's own factoring implementation on \textsc{AC}, which also has
quadratic depth \cite{Zalka1998}.
Takahashi and Kunihiro extended their earlier $O(n)$-depth adder to a factoring
circuit in $O(n^3)$-depth with linear width \cite{Takahashi2006}.
Van Meter and Itoh explore many different approaches for serial QPF,
with their lowest achievable depth being $O(n^2\log n)$ with
$O(n^2)$ on \textsc{NTC} \cite{VanMeter2005}. Cleve and Watrous
calculate a theoretical minimum circuit depth of $O(\log^3 n)$ and corresponding
circuit size of $O(n^3)$ on \textsc{AC},
using an adder which has depth $O(\log n)$ and
$O(n)$ size and width. We meet this bound and provide a concrete
architectural implementation using an adder with $O(1)$-depth and $O(n)$
size and width.

In the current work, we assume that errors do not affect the storage of qubits
during the circuit's operation. An alternate approach is taken by
Miquel \cite{Miquel1996} and Garcia-Mata \cite{GarciaMata2007}, who both
numerically simulate Shor's algorithm for factoring specific
numbers to determine its sensitivity to errors. Beckman et al. provide a
concrete factoring implementation in ion traps with $O(n^3)$ depth and size and
$O(n)$ width \cite{Beckman1996}.

In all the previous works,
it is assumed that qubits are expensive (width) and that
execution time (depth) is not the limiting constraint.
We make the alternative assumption that ancillae are cheap and that fast classical control
is available which allows access to all qubits in parallel.
Therefore, we optimize circuit depth at the expense of width.
We compare our work primarily to Kutin's method \cite{Kutin2006}.

These works also rely on serial QPF, which requires an inverse QFT.
On an AC architecture, even when approximating the (inverse) QFT by truncating two-qubit
$\pi/2^k$ rotations beyond $k = O(\log n)$, 
the depth is $O(n \log n)$ to factor an $n$-bit number.
To be implemented fault-tolerantly on a quantum device, rotations in the QFT must then be compiled into a discrete gate basis, 
which requires at least a $O(\log(1/\epsilon))$ overhead in depth to approximate a rotation with precision $\epsilon$ \cite{Harrow02, Kitaev2002}.
We would like to avoid the use of a QFT due to its compilation overhead.
However, a recent result by \cite{Jones2013} allows one to enact a
QFT using only Clifford gates and a Toffoli gate in $O(\log^2 n)$ expected depth.
This would allow us to
greatly improve the constants in our circuit resource upper bounds in Section \ref{sec:modexp}.

There is an alternative, parallel version of phase estimation 
\cite{Cleve2000,Kitaev2002}, which we call \emph{parallel QPF} (we refer the reader to Section 13 of \cite{Kitaev2002} for details), which decreases depth in exchange
for increased width and additional classical post-processing.
This eliminates the need to do an inverse QFT.
%We refer the reader to \cite{Kitaev2002} 
%and \cite{Pham2011b} 
%for details.
We develop a nearest-neighbor factoring circuit based on parallel QPF and our proposed 2D quantum arithmetic circuits.
We show that it is asymptotically more efficient than the serial QPF method. 
We compare the circuit resources required by our work with existing serial QPF implementations in Table
\ref{tab:results} of Section \ref{sec:results}.

We also note that recent results by Browne, Kashefi, and Perdrix (BKP) connect the power of
measurement-based quantum computing to the quantum circuit model augmented with
unbounded fanout \cite{Browne2009}. Their model, which we adapt and call
\textsc{CCNTC}, uses the classical controller mentioned in Section \ref{subsec:fanout}.
Using results by H{\o}yer and {\v S}palek \cite{Hoyer2002} that
unbounded quantum fanout would allow or a constant-depth factoring algorithm,
they conclude that a probabilistic polytime classical machine with access
to a constant-depth one-way quantum computer would also be able to factor.

%\section{The Constant-Depth\\ Carry-Save Technique}
\label{sec:csa}

Our 2D factoring approach rests on the central technique of the constant-depth
carry-save adder (CSA) \cite{Gossett1998}, which converts the sum of three
numbers $a$, $b$, and $c$, to the sum of two numbers $u$ and $v$:
%\begin{equation}
$a+b+c = u+v$. To explain this technique and how it achieves constant depth,
we need the following definitions.
%\end{equation}

A \emph{conventional number} $x$ can be represented in $n$ bits as
%\begin{equation}
$x = \sum_{i=0}^{n-1} 2^i x_i$,
%\end{equation}
where $x_i \in \{0,1\}$ denotes the $i$th bit of $x$, which we call
an $i$-bit.\footnote{It will be clear from the context whether we mean an
$i$-bit, which has significance $2^i$, or an $i$-bit number.}
Equivalently, $x$ can be represented as a (non-unique)
sum of two smaller conventional numbers, $u$ and $v$.
We say $(u+v)$ is a \emph{carry-save encoded}, or CSE, number.
The CSE representation itself consists of $2n-2$ individual
bits where $v_0$ is always $0$ by convention.

At the level of bits, a CSA converts the sum of three
$i$-bits into the sum of an $i$-bit (the \emph{sum} bit) and an $(i+1)$-bit
(the \emph{carry} bit):
%\begin{equation}
%\label{eqn:csa-3-2}
$a_i+b_i+c_i = u_i+v_{i+1}$.
%\end{equation}
By convention, the bit $u_i$ is the parity of the input bits
($u_i = a_i \oplus b_i \oplus c_i$) and
the bit $v_{i+1}$ is the majority of $\{a_i, b_i, c_i\}$.
See Figure \ref{fig:csa-encoding} for a concrete example, where
$(u+v)$ has $2n-2 = 8$ bits, not counting $v_0$.

%
It will also be useful to refer to a subset of the bits in a conventional
number using subscripts to indicate a range of indices.
\begin{equation}
x_{(j,k)} \equiv \sum_{i=j}^k 2^ix_i \qquad
x_{(i)} \equiv x_{(i,i)} = 2^ix_i
\end{equation}
%
Using this notation, the following identity holds.
\begin{equation}
x_{(j,k)} = x_{(j,\ell)} + x_{(\ell+1,k)} \qquad \text{ for all } j \le \ell < k
\end{equation}
%
We can express the relationship between the bits of $x$ and $(u+v)$ as follows.
%
\begin{equation}
x = x_{(0,n-1)} \equiv u+v = u_{(0,n-2)} + v_{(1,n-1)}
\end{equation}
%
Finally, we will denote taking the modular residue of a number as follows:
$x_{(j,k)}[m] \equiv x_{(j,k)} \bmod m$.

\begin{center}
\begin{figure*}[tb!]
\begin{displaymath}
x = 30 = u+v = 8 + 22 = \left\{
\begin{array}{ccccc}
    & u_3 & u_2 & u_1 & u_0 \\
v_4 & v_3 & v_2 & v_1 &    \\
\hline
x_4 & x_3 & x_2 & x_1 & x_0
\end{array}
\right\}
=
\left\{
\begin{array}{ccccc}
    & 1 & 0 & 0 & 0 \\
  1 & 0 & 1 & 1 &   \\
\hline
1 & 1 & 1 & 1 & 0
\end{array}
\right\}
\end{displaymath}
\caption{An example of carry-save encoding for the 5-bit conventional number 30.}
\label{fig:csa-encoding}
\end{figure*}
\end{center}
%

\begin{figure*}[tb!]
\begin{center}
\begin{displaymath}
\centerline{
\Qcircuit @C=2em @R=2em {
\lstick{\ket{0}} & \qw      & \qw & \qw                 & \qw & \qw                        & \targfix  & \qw & \qw_{\ket{a_i \wedge (b_i \oplus c_i)}} & \targfix  & \qw       & \qw    & \qw_{\ket{(b_i \wedge c_i) \oplus a_i \wedge (b_i \oplus c_i)}} & \qw & \qswap & \qw & \qw & \rstick{\ket{u_i}} \\
\lstick{\ket{a_i}} & \qw      & \qw & \qw                 & \qw & \qw                        & \ctrl{-1} & \qw & \qw                             & \qw       & \targfix  & \qw  & \qw_{\ket{a_i \oplus b_i \oplus c_i}} & \qw & \qswap \qwx & \qswap & \qw & \rstick{\ket{0}} \\
\lstick{\ket{b_i}} & \ctrl{1} & \qw & \targfix            & \qw & \qw_{\ket{b_i \oplus c_i}} & \ctrl{-1} & \qw & \qw                             & \qw       & \ctrl{-1} & \targfix  & \ctrl{1} & \qw & \qw & \qw \qwx & \qw & \rstick{\ket{b_i}} \\
\lstick{\ket{c_i}} & \ctrl{1} & \qw & \ctrl{-1}           & \qw & \qw                        & \qw       & \qw & \qw                             & \qw       & \qw       & \ctrl{-1} & \ctrl{1} & \qw & \qw & \qw \qwx & \qw & \rstick{\ket{c_i}} \\
\lstick{\ket{0}} & \targfix & \qw & \qw_{\ket{b_i \wedge c_i}} & \qw & \qw                        & \qw       & \qw & \qw                             & \ctrl{-4} & \qw       & \qw       & \targfix & \qw & \qw & \qswap \qwx & \qw & \rstick{\ket{v_{i+1}}} 
}
}
\end{displaymath}
\caption{Carry-save adder circuit for a single bit position $i$:
$a_i+b_i+c_i = u_i + v_{i+1}$.}
\label{fig:csa-circuit}
\end{center}\end{figure*}

Using a Toffoli gate decomposition (see p.~182 \cite{Nielsen2000}), two control
qubits and a single target qubit must be
mutually connected to each other. Given this constraint, and the
interaction of the CNOTs in Figure \ref{fig:csa-circuit}, we can
rearrange these qubits on a 2D planar grid and obtain the layout shown
in Figure \ref{fig:csa-3-2}, which satisfies our 2D NTC model.
Note that this uses more gates and one more ancilla than the equivalent
quantum full adder circuit in Figure 5 of \cite{Gossett1998}, but this
is necessary to meet our architectural constraints and does not change the
asymptotic results.
Also in Figure \ref{fig:csa-3-2}
is a variation called a 2-2 adder, which simply re-encodes two $i$-bits
into an $i$-bit and an $(i+1)$-bit, which will be useful in the next section.

\begin{figure}[b!]
\begin{center}
\includegraphics[width=3in]{./figures/csa-32-22.pdf}
\end{center}
\caption{The carry-save adder (CSA), or 3-2 adder, and carry-save 2-2 adder.}
\label{fig:csa-3-2}
\end{figure}

At the level of numbers, the sum of three $n$-bit numbers can be converted into
the sum of two $n$-bit numbers by applying a \emph{CSA layer} of
$n$ parallel, single-bit
CSA's. Since each CSA operates in constant depth, the entire layer also
operates in constant-depth, and we have achieved (non-modular) addition.

An important consideration here is the circuit width. The circuit above
operates out-of-place and produces two garbage qubits, the original inputs
$b_i$ and $c_i$. A single addition of three $n$-bit numbers requires a
$O(n)$ circuit width.

\section{The Constant-Depth Carry-Save Technique}
\label{sec:csa}

Our 2D factoring approach rests on the central technique of the constant-depth
carry-save adder (CSA) \cite{Gossett1998}, which converts the sum of three
numbers $a$, $b$, and $c$, to the sum of two numbers $u$ and $v$:
%\begin{equation}
$a+b+c = u+v$. The explanation of this technique and how it achieves constant depth requires the following definitions.
%\end{equation}

A \emph{conventional number} $x$ can be represented in $n$ bits as
%\begin{equation}
$x = \sum_{i=0}^{n-1} 2^i x_i$,
%\end{equation}
where $x_i \in \{0,1\}$ denotes the $i$-th bit of $x$, which we call
an $i$-bit and has significance $2^i$, and the $0$-th bit is the low-order bit.\footnote{It will be clear from the context whether we mean an
$i$-bit, which has significance $2^i$, or an $i$-bit number.}
Equivalently, $x$ can be represented as a (non-unique)
sum of two smaller conventional numbers, $u$ and $v$.
We say $(u+v)$ is a \emph{carry-save encoded}, or CSE, number.
The CSE representation itself consists of $2n-2$ individual
bits where $v_0$ is always $0$ by convention.

Consider a CSA operating on three bits instead of three numbers; 
then a CSA converts the sum of three
$i$-bits into the sum of an $i$-bit (the \emph{sum} bit) and an $(i+1)$-bit
(the \emph{carry} bit):
%\begin{equation}
%\label{eqn:csa-3-2}
$a_i+b_i+c_i = u_i+v_{i+1}$.
%\end{equation}
By convention, the bit $u_i$ is the parity of the input bits
($u_i = a_i \oplus b_i \oplus c_i$) and
the bit $v_{i+1}$ is the majority of $\{a_i, b_i, c_i\}$.
Figure \ref{fig:csa-encoding} gives a concrete example, where
$(u+v)$ has $2n-2 = 8$ bits, not counting $v_0$.

%
It will also be useful to refer to a subset of the bits in a conventional
number using subscripts to indicate a range of indices:
\begin{equation}
x_{(j,k)} \equiv \sum_{i=j}^k 2^ix_i \qquad
x_{(i)} \equiv x_{(i,i)} = 2^ix_i.
\end{equation}
%
Using this notation, the following identity holds:
\begin{equation}
x_{(j,k)} = x_{(j,\ell)} + x_{(\ell+1,k)}, \qquad \text{ for all } j \le \ell < k.
\end{equation}
%
We can express the relationship between the bits of $x$ and $(u+v)$ as follows:
%
\begin{equation}
x = x_{(0,n-1)} \equiv u+v = u_{(0,n-2)} + v_{(1,n-1)}.
\end{equation}
%
Finally, we denote arithmetic modulo $m$, $x_{(j,k)} \bmod m$, as
$x_{(j,k)}[m]$.

\begin{center}
\begin{figure*}[tb!]
\begin{displaymath}
x = 30 = u+v = 8 + 22 = \left\{
\begin{array}{ccccc}
    & u_3 & u_2 & u_1 & u_0 \\
v_4 & v_3 & v_2 & v_1 &    \\
\hline
x_4 & x_3 & x_2 & x_1 & x_0
\end{array}
\right\}
=
\left\{
\begin{array}{ccccc}
    & 1 & 0 & 0 & 0 \\
  1 & 0 & 1 & 1 &   \\
\hline
1 & 1 & 1 & 1 & 0
\end{array}
\right\}
\end{displaymath}
\fcaption{An example of carry-save encoding for the 5-bit conventional number 30.}
\label{fig:csa-encoding}
\end{figure*}
\end{center}
%

\begin{figure}[tb!]
\begin{center}
\begin{displaymath}
\centerline{
\Qcircuit @C=2em @R=2em {
\lstick{\ket{0}}   & \qw      & \qw & \qw                        & \qw & \qw                        & \targfix  & \qw & \qw_{\ket{a_i \wedge (b_i \oplus c_i)}} & \targfix  & \qw       & \qw       & \qw_{\ket{(b_i \wedge c_i) \oplus a_i \wedge (b_i \oplus c_i)}} & \qw & \qswap      & \qswap      & \qw & \rstick{\ket{u_i}} \\
\lstick{\ket{a_i}} & \qw      & \qw & \qw                        & \qw & \qw                        & \ctrl{-1} & \qw & \qw                                     & \qw       & \targfix  & \qw       & \qw_{\ket{a_i \oplus b_i \oplus c_i}}                           & \qw & \qw \qwx    & \qswap \qwx & \qw & \rstick{\ket{0}} \\
\lstick{\ket{b_i}} & \ctrl{1} & \qw & \targfix                   & \qw & \qw_{\ket{b_i \oplus c_i}} & \ctrl{-1} & \qw & \qw                                     & \qw       & \ctrl{-1} & \targfix  & \ctrl{1}                                                        & \qw & \qw \qwx    & \qw         & \qw & \rstick{\ket{b_i}} \\
\lstick{\ket{c_i}} & \ctrl{1} & \qw & \ctrl{-1}                  & \qw & \qw                        & \qw       & \qw & \qw                                     & \qw       & \qw       & \ctrl{-1} & \ctrl{1}                                                        & \qw & \qw \qwx    & \qw         & \qw & \rstick{\ket{c_i}} \\
\lstick{\ket{0}}   & \targfix & \qw & \qw_{\ket{b_i \wedge c_i}} & \qw & \qw                        & \qw       & \qw & \qw                                     & \ctrl{-4} & \qw       & \qw       & \targfix                                                        & \qw & \qswap \qwx & \qw         & \qw & \rstick{\ket{v_{i+1}}}}
}
\end{displaymath}
\fcaption{Carry-save adder circuit for a single bit position $i$: $a_i+b_i+c_i = u_i + v_{i+1}$.}
\label{fig:csa-circuit}
\end{center}\end{figure}

\begin{figure}
\begin{center}
\begin{displaymath}
\begin{tabular}{p{0.5in} m{0.1in} p{2in}}

\Qcircuit @C=1em @R=2.2em { 
	& \qw & \ctrl{1} & \qw & \qw \\
	& \qw & \ctrl{1} & \qw & \qw \\
	& \qw & \targfix & \qw & \qw
}

&
\qquad
=
\qquad
&

\Qcircuit @C=1em @R=.7em { 
	& \gate{T^{\dagger}} & \qw & \targfix  & \qw & \gate{T} & \qw & \targfix  & \qw & \gate{T^{\dagger}} & \qw & \targfix  & \qw & \gate{T}           & \qw & \targfix  & \qw & \qw \\ 
	& \gate{T^{\dagger}} & \qw & \qw       & \qw & \ctrl{1} & \qw & \ctrl{-1} & \qw & \ctrl{1}           & \qw & \qw       & \qw & \qw                & \qw & \ctrl{-1} & \qw & \qw \\
	& \gate{H}           & \qw & \ctrl{-2} & \qw & \targfix & \qw & \gate{T}  & \qw & \targfix           & \qw & \ctrl{-2} & \qw & \gate{T^{\dagger}} & \qw & \gate{H}  & \qw & \qw
}
\end{tabular}
\end{displaymath}
\fcaption{The depth-efficient Toffoli gate decomposition from \cite{Amy2012}.}
\label{fig:toffoli}
\end{center}
\end{figure}

Figure \ref{fig:csa-circuit} gives a circuit description of carry-save addition (CSA) for a single bit position $i$.
The resources for this circuit are given in Table \ref{tab:csa-tile-resources}, using
the resources for the Toffoli gate (in the same table) based on
\cite{Amy2012}. We note here
that a more efficient decomposition for the Toffoli is possible using a
distillation approach described in \cite{Jones2013a}.

We must lay out the circuit to satisfy a 2D NTC model.
The Toffoli gate decomposition in \cite{Amy2012}, duplicated in
Figure \ref{fig:toffoli}, requires two control
qubits and a single target qubit to be
mutually connected to each other. Given this constraint, and the
interaction of the CNOTs in Figure \ref{fig:csa-circuit}, we can
rearrange these qubits on a 2D planar grid and obtain the layout shown
in Figure \ref{fig:csa-3-2}, which satisfies our 2D NTC model.
Qubits $\ket a_i$, $\ket b_i$, and $\ket c_i$ reside at the top of Fig.~\ref{fig:csa-3-2}, while qubits $\ket{u_i}$ and $\ket{v_{i+1}}$ are initialized to $\ket 0$.
Upon completion of the circuit, qubit $\ket{a_i}$ is in state $\ket 0$, as seen from the output in Fig.~\ref{fig:csa-circuit}. 
Note that this construction uses more gates and one more ancilla than the equivalent
quantum full adder circuit in Figure 5 of \cite{Gossett1998}, however this
is necessary in order to meet our architectural constraints and does not change the
asymptotic results.
Also in Figure \ref{fig:csa-3-2}
is a variation called a 2-2 adder, which simply re-encodes two $i$-bits
into an $i$-bit and an $(i+1)$-bit, which will be useful in the next section.

\begin{figure}[b!]
\begin{center}
\includegraphics[width=3in]{./csa-32-22.pdf}
\end{center}
\fcaption{The carry-save adder (CSA), or 3-2 adder, and carry-save 2-2 adder.}
\label{fig:csa-3-2}
\end{figure}

At the level of numbers, the sum of three $n$-bit numbers can be converted into
the sum of two $n$-bit numbers by applying a \emph{CSA layer} of
$n$ parallel, single-bit
CSA circuits (Fig.~\ref{fig:csa-circuit}). Since each CSA operates in constant depth, the entire layer also
operates in constant-depth, and we have achieved (non-modular) addition.
%
%An important consideration is the circuit width. The circuit above
%requires two additional qubits to contain the output
%out-of-place and produces two garbage qubits: the original inputs
%$b_i$ and $c_i$. 
Each single addition of three $n$-bit numbers requires $O(n)$ circuit width.

%\section{Quantum Modular Addition}
\label{sec:csa-mod-add}

To perform addition of two numbers $a$ and $b$ modulo $m$,
we consider the variant problem of modular addition of three numbers to
two numbers:
%
%\begin{quote}
Given three $n$-bit input numbers $a$, $b$, and $c$ and an $n$-bit modulus $m$,
compute the following:
%\begin{equation}
$(u+v) = (a+b+c)[m]$,
%\end{equation}
where $(u+v)$ is a CSE number.

In this section, we provide an alternative, pedagogical explanation of
Gossett's modular reduction \cite{Gossett1998}. Later, we contribute a mapping
to a 2D architecture,
using unbounded fanout to maintain constant-depth for adding back
modular residues. This last step is missing in Gossett's original approach.

To start, we will demonstrate the basic method of modular addition and reduction
on an $n$-bit conventional number. In general, adding two $n$-bit conventional
numbers will produce an overflow $n$-bit, which we can truncate as long as
we add back its modular residue $2^n \bmod m$. How can we guarantee that we won't
generate another overflow bit by adding back the modular residue? It turns out
we can accomplish this by allowing
a slightly larger input and output number ($n+1$ bits in this case), truncating
multiple overflow bits, and adding back their modular residues.

For an $(n+1)$-bit conventional number $x$,
we truncate its high-order bits $x_n$ and $x_{n-1}$
and
add back their \emph{modular residue}, $x_{(n-1,n)}[m]$.
%
\begin{eqnarray}
x \bmod m &=& x_{(0,n)}[m] \nonumber \\
&=& x_{(0,n-2)} + x_{(n-1,n)}[m]
\end{eqnarray}
%
Since both the truncated number $x_{(0,n-2)}$ and the modular residue
are $n$-bit numbers, their sum is an $(n+1)$-bit number as desired, equivalent
to $x[m]$.

Now we must do the same modular reduction on a CSE number $(u+v)$,
which represents an $(n+1)$-bit conventional number and has
$2n$ bits.
%This is the special case mentioned in the
%previous
%section \label{star:csa-special}, where $x$ is the result of a single
%CSA layer, not repeated CSA layers alternating with truncation.
%
%Assume for now that this modular reduction works;
%in the next section we walk through an illustrated concrete example.
%We present a more formal argument in Section \ref{subsec:mod-reduce-1}.
%
First, we truncate the three high-order bits ($v_{n}, u_{n-1}, v_{n-1}$)
of $(u+v)$, yielding an $n$-bit
conventional number with a CSE representation of $2n-3$ bits:
$\{u_0, u_1, \ldots, u_{n-2}\} \cup \{v_1, v_2, \ldots, v_{n-2}\}$.
Then we add back the three modular residues
$(v_{(n)}[m], u_{(n-1)}[m], v_{(n-1)}[m])$, and we are guaranteed not to
get more overflow bits (of significance $2^{n-1}$ or higher). This equivalence
is shown in Equation \ref{eqn:mod-reduce}.
\begin{eqnarray}
(u+v)[m] &=& \left(u_{(0,n-1)} + v_{(1,n)}\right)[m] \nonumber \\
 &=& u_{(0,n-2)} +
     v_{(1,n-2)} + \nonumber \\
 & & u_{(n-1)}[m] + 
     v_{(n-1)}[m] + \nonumber \\
 & & v_{(n)}[m]
\label{eqn:mod-reduce}
\end{eqnarray}

\begin{lemma}[Modular Reduction in Constant Depth \cite{Gossett1998}]
The modular addition of three $n$-bit numbers to two $n$-bit numbers can be
accomplished
in constant depth.
\end{lemma}

\begin{proof}
Our goal is to show how to perform modular addition while keeping our numbers
of a fixed size by treating overflow bits correctly.
First, we enlarge our registers to allow the addition of $(n+2)$-bit numbers,
while keeping our modulus of size $n$ bits.
(In Gossett's original approach, he takes the equivalent step of restricting
the modulus to be of size $(n-2)$ bits.) We accomplish the modular addition
by first performing a layer of non-modular addition, truncating the three high-order
overflow bits, and then adding back modular residues controlled on these
bits in three successive layers, where we are guaranteed that no additional
overflow bits are generated.
This is illustrated for a $3$-bit modulus, and $5$-bit registers,
in Figure \ref{fig:csa-proof}.

\begin{center}
\begin{figure*}[h!tb]
\begin{displaymath}
\renewcommand\arraystretch{1.5}
\begin{array}{ccccccl}
        & a_4 & a_3 & a_2 & a_1 & a_0 & 5\text{-bit input number } a\\
        & b_4 & b_3 & b_2 & b_1 & b_0 & 5\text{-bit input number } b\\
        & c_4 & c_3 & c_2 & c_1 & c_0 & 5\text{-bit input number } c\\
\hline
        & u_4 & u_3 & u_2 & u_1 & u_0 & \text{truncate } u_{4} \\
    v_5 & v_4 & v_3 & v_2 & v_1 &     & \text{truncate } v_{4},v_{5} \\
        &     &     & c^{v_4}_2 & c^{v_4}_1 & c^{v_4}_0 & \text{add back } 2^4 \bmod m \text{ controlled on } v_4 \\
\hline
        &      & u'_3 & u'_2 & u'_1 & u'_0 & \\ 
        & v'_4 & v'_3 & v'_2 & v'_1 &      & \\
        &      &    & c^{u_4}_2 & c^{u_4}_1 & c^{u_4}_0  & \text{add back } 2^4 \bmod m \text{ controlled on } u_4 \\
\hline
        & u''_4 & u''_3 & u''_2 & u''_1 & u''_0 & \text{the bit } u''_4 \text{ is the same as } v'_4 \\
        & v''_4 & v''_3 & v''_2 & v''_1 &       &  \\
        &       &    & c^{v_5}_2 & c^{v_5}_1 & c^{v_5}_0 & \text{add back } 2^5 \bmod m \text{ controlled on } v_5 \\
\hline
        & u'''_4 & u'''_3 & u'''_2 & u'''_1 & u'''_0 & \text{ Final CSE output with } 5 \text{ bits}\\
        & v'''_4 & v'''_3 & v'''_2 & v'''_1 &        & \text{ Final CSE output with } 5 \text{ bits}\\
\end{array}
%\begin{array}{cccccccr}
%        & a_{n+1} & a_{n} & a_{n-1} & \ldots & a_1 & a_0 & \text{input number } a\\
%        & b_{n+1} & b_{n} & b_{n-1} & \ldots & b_1 & b_0 & \text{input number } b\\
%        & c_{n+1} & c_{n} & c_{n-1} & \ldots & c_1 & c_0 & \text{input number } c\\
%\hline
%        & u_{n+1} & u_{n} & u_{n-1} & \ldots & u_1 & u_0 & \text{truncate } u_{n+1} \\
%v_{n+2} & v_{n+1} & v_{n} & v_{n-1} & \ldots & v_1 & 0   & \text{truncate } v_{n+1},v_{n+2} \\
%        &         &       & x_{n-1} & \ldots & x_1 & x_0 \\
%\hline
%        &         & u'_{n} & u'_{n-1} & \ldots & u'_1 & u'_0 & \\ 
%        & v'_{n+1} & v'_{n} & v'_{n-1} & \ldots & v'_1 & 0 &  \\
%        &         &       & x_{n-1} & \ldots & x_1 & x_0 \\
%\hline
%        & u''_{n+1} & u''_{n} & u''_{n-1} & \ldots & u''_1 & u''_0 & \\
%        & v''_{n+1} & v''_{n} & v''_{n-1} & \ldots & v''_1 & 0 &  \\
%        &         &       & y_{n-1} & \ldots & y_1 & y_0 \\
%\hline
%        & u'''_{n+1} & u'''_{n} & u'''_{n-1} & \ldots & u'''_1 & u'''_0 & \\
%        & v'''_{n+1} & v'''_{n} & v'''_{n-1} & \ldots & v'''_1 & 0 &  \\
%\hline
%\end{array}
\end{displaymath}
\caption{A schematic proof of Gossett's constant-depth modular reduction for $n=3$}
\label{fig:csa-proof}
\end{figure*}
\end{center}

We use the following notation.
The non-modular sum of the first layer is $u$ and $v$.
The CSE output of the first modular reduction layer
is $u'$ and $v'$, and the modular residue is
written as $c^{v_{n+1}}$ to mean the precomputed value $2^{n+1} \bmod m$
controlled on $v_{n+1}$.
The CSE output of the second modular reduction layer
is $u''$ and $v''$, and the modular residue is written as
$c^{u_{n+1}}$ to mean the precomputed value $2^{n+1} \bmod m$
controlled on $u_{n+1}$.
The CSE output of the third and final modular reduction layer
is $u'''$ and $v'''$, and the modular residue is written as
$c^{v_{n+2}}$ to mean the precomputed value $2^{n+2} \bmod m$
controlled on $v_{n+2}$.

We show that at no layer is an overflow $(n+2)$-bit generated, namely in the
$v$ component of any CSE output. (The $u$ component will never exceed the
size of the input numbers.) First, we know that no $v'_{n+2}$ bit
is generated after the first modular reduction layer, because we have
truncated away all $(n+1)$-bits. Second, we know that no $v''_{n+2}$ bit is
generated because we only have one $(n+1)$-bit to add, $v'_{n+1}$.
Finally, we need to show a sufficient condition for no $v'''_{n+2}$ bit being
generated in the third modular reduction layer. This bit is the majority of
$u''_{n+1}$, $v''_{n+1}$, and $c^{v_{n+2}}_{n+1} = 0$. This means we only have
to guarantee that at most one of $u''_{n+1}$ and $v''_{n+1}$ has value 1.
This is equivalent to requiring that
$u''_{(n,n+1)} + v''_{(n+1)} \le 3\cdot 2^{n+1}$, that is, the sum of these
three bits has value at most $3$. Bit $u''_{n+1}$ is copied directly from
$v'_{n+1}$ by the rules of CSA, which implies the following condition for
the second modular reduction layer:
$u'_{(n)} + v'_{(n,n+1)} \le 3\cdot 2^n$. This is true because
$u'_{(n)} + v'_{(n+1)} = u_{(n)} + v_{(n)} \le 2$ and $v'_{(n)} \le 1$.
Everywhere
we use the fact that the modular residues are restricted to $n$ bits.
Therefore, the modular sum is computed as the sum of two $(n+2)$-bit numbers
with no overflows in constant-depth.
\end{proof}

As a side note, we can perform modular reduction in one layer instead of
three by decoding the three overflow bits into one of seven different
modular residues. This can also be done in constant depth, and in this case
we only need to enlarge all our registers to $(n+1)$ bits instead of $(n+2)$
as in the proof above. However, we omit this proof here for simplicity.

To summarize,
the circuit resources for modular addition are $O(1)$ depth and $O(n)$ width.

%%%%%%%%%%%%%%%%%%%%%%%%%%%%%%%%%%%%%%%%%%%%%%%%%%%%%%%%%%%%%%%%%%%%%%%%%%%%%%%
\subsection{A Concrete Example of\\ Modular Addition}
\label{subsec:concrete}

\begin{center}
\begin{figure*}[h!bt]
\centerline{
\includegraphics[width=6.5in]{./figures/mod-add-fixed.pdf}
}
\caption{Addition and three rounds of modular reduction for a 3-bit
modulus.}
\label{fig:csa-add-4}
\end{figure*}
\end{center}

A 2D circuit for modular addition of $5$-bit numbers using
four layers of parallel CSA's is shown graphically in Figure \ref{fig:csa-add-4}
which corresponds directly to the schematic proof in Figure \ref{fig:csa-proof}.
Figure \ref{fig:csa-add-4} also represents the approximate
physical layout of the qubits as they would look if this
circuit were to be fabricated.
Here, we convert the sum of three
$5$-bit integers into the modular sum of two $5$-bit integers, with a
$3$-bit modulus $m$.
In the first layer, 
we perform 4 CSA's in parallel on the input numbers ($a,b,c$) and produce the
output numbers ($u, v$).

As described above, we truncate
the three high-order bits during the initial CSA round
(bits $u_4, v_4, v_5$) to retain a $4$-bit number.
Each of these bits serves as a control for adding its modular residue to
a running total. We can classically precompute $2^4[m]$ for the two
additions controlled on $u_4$ and $v_4$ and
$2^5[m]$ for the addition controlled on $v_5$.

In layer 2,
we use a constant-depth fanout rail (see Figure \ref{fig:cd}) to
distribute the control bit $v_4$ to its modular residue, which we denote as
%%\begin{equation}
$\ket{c^{v_4}} \equiv \ket{2^4[m]\cdot v_4}$.
%%\end{equation}
%This fanout requires constant depth;
$c^{v_4}$ has $n$ bits, which we add to the CSE results of layer 1.
The results $u_i$ and $v_{i+1}$ are teleported into layer 3. The exception is
$v'_4$ which is teleported into layer 4, since there are no other $4$-bits
to which it can be added. Wherever there are only
two bits of the same significance, we use the 2-2 adder from \ref{sec:csa}.

Layer 3
%%, shown in Figure \ref{fig:csa-add-3},
operates similarly to layer 2, except that the modular residue is controlled on
$u_4$:
%%\begin{equation}
$\ket{c^{u_4}} \equiv \ket{2^4[m] \cdot u_4}$.
%%\end{equation}
%This fanout again requires constant depth; 
$c^{u_4}$ has $3$ bits, which we
add to the CSE results of layer 2, where $u'_i$ and $v'_{i+1}$ are teleported
forward into layer 4.

Layer 4
%%, shown in Figure \ref{fig:csa-add-4},
is similar to layers 2 and 3, with the modular residue controlled on $v_5$:
%%\begin{equation}
$\ket{c^{v_5}} \equiv \ket{2^5[m] \cdot v_5}$.
%%\end{equation}
%This fanout is constant depth; 
$c^{v_5}$ has $3$ bits, which we
add to the CSE results of layer 3.
There is no overflow bit $v'''_5$, and no carry bit from $v''_4$ and $v'_4$
as argued in Lemma 1.
The final modular sum $(a+b+c)[m]$ is $u'''+v'''$.

\section{Quantum Modular Addition}
\label{sec:csa-mod-add}

To perform addition of two numbers $a$ and $b$ modulo $m$,
we consider the variant problem of modular addition of three numbers to
two numbers:
%
%\begin{quote}
Given three $n$-bit input numbers $a$, $b$, and $c$, and an $n$-bit modulus $m$,
compute
%\begin{equation}
$(u+v) = (a+b+c)[m]$,
%\end{equation}
where $(u+v)$ is a CSE number.

In this section, we provide an alternate, pedagogical explanation of
Gossett's modular reduction \cite{Gossett1998}. Later, we contribute a mapping of this adder
to a 2D architecture,
using unbounded fanout to maintain constant-depth for adding back
modular residues. This last step is absent in Gossett's original approach.

To start, we will demonstrate the basic method of modular addition and reduction
on an $n$-bit conventional number. In general, adding two $n$-bit conventional
numbers will produce an overflow bit of significance $2^n$, which we can truncate as long as
we add back its modular residue $2^n \bmod m$. How can we guarantee that we won't
generate another overflow bit by adding back the modular residue? It turns out
we can accomplish this by allowing
a slightly larger input and output number ($n+1$ bits in this case), truncating
multiple overflow bits, and adding back their modular residues.

For an $(n+1)$-bit conventional number $x$,
we truncate its high-order bits $x_n$ and $x_{n-1}$
and
add back their modular residue $x_{(n-1,n)}[m]$:
%
\begin{eqnarray}
x \bmod m &=& x_{(0,n)}[m] \nonumber \\
&=& x_{(0,n-2)} + x_{(n-1,n)}[m].
\end{eqnarray}
%
Since both the truncated number $x_{(0,n-2)}$ and the modular residue
are $n$-bit numbers, their sum is an $(n+1)$-bit number as desired, equivalent
to $x[m]$.

Now we must do the same modular reduction on a CSE number $(u+v)$,
which in this case represents an $(n+2)$-bit conventional number and has
$2n+3$ bits.
%This is the special case mentioned in the
%previous
%section \label{star:csa-special}, where $x$ is the result of a single
%CSA layer, not repeated CSA layers alternating with truncation.
%
%Assume for now that this modular reduction works;
%in the next section we walk through an illustrated concrete example.
%We present a more formal argument in Section \ref{subsec:mod-reduce-1}.
%
First, we truncate the three high-order bits ($v_{n}, u_{n-1}, v_{n-1}$)
of $(u+v)$, yielding an $n$-bit
conventional number with a CSE representation of $2n-3$ bits:
$\{u_0, u_1, \ldots, u_{n-2}\} \cup \{v_1, v_2, \ldots, v_{n-2}\}$.
Then we add back the three modular residues
$(v_{(n)}[m], u_{(n-1)}[m], v_{(n-1)}[m])$, and we are guaranteed not to
generate additional overflow bits (of significance $2^{n-1}$ or higher). This equivalence
is shown in Eq \ref{eqn:mod-reduce}.
\begin{eqnarray}
(u+v)[m] &=& \left(u_{(0,n+1)} + v_{(1,n+2)}\right)[m] \nonumber \\
 &=& u_{(0,n)} +
     v_{(1,n)} + \nonumber \\
 & & u_{(n+1)}[m] +
     v_{(n+1)}[m] + \nonumber \\
 & & v_{(n+2)}[m]
\label{eqn:mod-reduce}
\end{eqnarray}

\begin{lemma}[Modular Reduction in Constant Depth]
The modular addition of three $n$-bit numbers to two $n$-bit numbers can be
accomplished
in constant depth with $O(n)$ width in \textsc{2D CCNTC}.
\end{lemma}

%\begin{proof}
\vspace*{12pt}
\noindent
{\bf Proof:}
Our goal is to show how to perform modular addition while keeping our numbers
of a fixed size by treating overflow bits correctly.
We map the proof of \cite{Gossett1998} to \textsc{2D CCNTC} and show that
we meet our required depth and width.
First, we enlarge our registers to allow the addition of $(n+2)$-bit numbers,
while keeping our modulus of size $n$ bits.
(In Gossett's original approach, he takes the equivalent step of restricting
the modulus to be of size $(n-2)$ bits.) We accomplish the modular addition
by first performing a layer of non-modular addition, truncating the three high-order
overflow bits, and then adding back modular residues controlled on these
bits in three successive layers, where we are guaranteed that no additional
overflow bits are generated in each layer.
This is illustrated for a $3$-bit modulus and $5$-bit registers
in Figure \ref{fig:csa-proof}.

\begin{center}
\begin{figure*}[h!tb]
\begin{displaymath}
\renewcommand\arraystretch{1.5}
\begin{array}{ccccccll}
        & a_4 & a_3 & a_2 & a_1 & a_0 & 5\text{-bit input number } a &\\
        & b_4 & b_3 & b_2 & b_1 & b_0 & 5\text{-bit input number } b & \\
        & c_4 & c_3 & c_2 & c_1 & c_0 & 5\text{-bit input number } c & \text{[Layer 1]}\\
\hline
        & u_4 & u_3 & u_2 & u_1 & u_0 & \text{truncate } u_{4} & \\
    v_5 & v_4 & v_3 & v_2 & v_1 &     & \text{truncate } v_{4},v_{5} & \\
        &     &     & c^{v_4}_2 & c^{v_4}_1 & c^{v_4}_0 & \text{add back } 2^4 \bmod m \text{ controlled on } v_4 & \text{[Layer 2]}\\
\hline
        &      & u'_3 & u'_2 & u'_1 & u'_0 & & \\
        & v'_4 & v'_3 & v'_2 & v'_1 &      & & \\
        &      &    & c^{u_4}_2 & c^{u_4}_1 & c^{u_4}_0  & \text{add back } 2^4 \bmod m \text{ controlled on } u_4 & \text{[Layer 3]}\\
\hline
        & u''_4 & u''_3 & u''_2 & u''_1 & u''_0 & \text{the bit } u''_4 \text{ is the same as } v'_4 & \\
        & v''_4 & v''_3 & v''_2 & v''_1 &       &  & \\
        &       &    & c^{v_5}_2 & c^{v_5}_1 & c^{v_5}_0 & \text{add back } 2^5 \bmod m \text{ controlled on } v_5 & \text{[Layer 4]}\\
\hline
        & u'''_4 & u'''_3 & u'''_2 & u'''_1 & u'''_0 & \text{ Final CSE output with } 5 \text{ bits} &\\
        & v'''_4 & v'''_3 & v'''_2 & v'''_1 &        & \text{ Final CSE output with } 5 \text{ bits} & \\
\end{array}
%\begin{array}{cccccccr}
%        & a_{n+1} & a_{n} & a_{n-1} & \ldots & a_1 & a_0 & \text{input number } a\\
%        & b_{n+1} & b_{n} & b_{n-1} & \ldots & b_1 & b_0 & \text{input number } b\\
%        & c_{n+1} & c_{n} & c_{n-1} & \ldots & c_1 & c_0 & \text{input number } c\\
%\hline
%        & u_{n+1} & u_{n} & u_{n-1} & \ldots & u_1 & u_0 & \text{truncate } u_{n+1} \\
%v_{n+2} & v_{n+1} & v_{n} & v_{n-1} & \ldots & v_1 & 0   & \text{truncate } v_{n+1},v_{n+2} \\
%        &         &       & x_{n-1} & \ldots & x_1 & x_0 \\
%\hline
%        &         & u'_{n} & u'_{n-1} & \ldots & u'_1 & u'_0 & \\
%        & v'_{n+1} & v'_{n} & v'_{n-1} & \ldots & v'_1 & 0 &  \\
%        &         &       & x_{n-1} & \ldots & x_1 & x_0 \\
%\hline
%        & u''_{n+1} & u''_{n} & u''_{n-1} & \ldots & u''_1 & u''_0 & \\
%        & v''_{n+1} & v''_{n} & v''_{n-1} & \ldots & v''_1 & 0 &  \\
%        &         &       & y_{n-1} & \ldots & y_1 & y_0 \\
%\hline
%        & u'''_{n+1} & u'''_{n} & u'''_{n-1} & \ldots & u'''_1 & u'''_0 & \\
%        & v'''_{n+1} & v'''_{n} & v'''_{n-1} & \ldots & v'''_1 & 0 &  \\
%\hline
%\end{array}
\end{displaymath}
\fcaption{A schematic proof of Gossett's constant-depth modular reduction for $n=3$.}
\label{fig:csa-proof}
\end{figure*}
\end{center}

We use the following notation.
The non-modular sum of the first layer is $u$ and $v$.
The CSE output of the first modular reduction layer
is $u'$ and $v'$, and the modular residue is
written as $c^{v_{n+1}}$ to mean the precomputed value $2^{n+1} \bmod m$
controlled on $v_{n+1}$.
The CSE output of the second modular reduction layer
is $u''$ and $v''$, and the modular residue is written as
$c^{u_{n+1}}$ to mean the precomputed value $2^{n+1} \bmod m$
controlled on $u_{n+1}$.
The CSE output of the third and final modular reduction layer
is $u'''$ and $v'''$, and the modular residue is written as
$c^{v_{n+2}}$ to mean the precomputed value $2^{n+2} \bmod m$
controlled on $v_{n+2}$.

We show that no layer generates an overflow $(n+2)$-bit, namely in the
$v$ component of any CSE output. (The $u$ component will never exceed the
size of the input numbers.) First, we know that no $v'_{n+2}$ bit
is generated after the first modular reduction layer, because we have
truncated away all $(n+1)$-bits. Second, we know that no $v''_{n+2}$ bit is
generated because we only have one $(n+1)$-bit to add, $v'_{n+1}$.
Finally, we need to that $v'''_{n+2} = 0$ in the third modular reduction layer. 

Since $u'_{(n)} + v'_{(n+1)} =
u_{(n)} + v_{(n)} <= 2^{n+1}$, the bits $u'_n and v'_{n+1}$ cannot both be $1$.
But $u''_{n+1} = v'_{n+1}$ and $v''_{n+1} = u'_n\land v'_n$, so $u''_{n+1}$ and
$v''_{n+1}$ cannot both be $1$, and hence $v'''_{n+2} = 0$.
%This bit is the majority of
%$u''_{n+1}$, $v''_{n+1}$, and $c^{v_{n+2}}_{n+1} = 0$. This means we only have
%to guarantee that at most one of $u''_{n+1}$ and $v''_{n+1}$ has value 1.
%This is equivalent to requiring that
%$u''_{(n,n+1)} + v''_{(n+1)} \le 3\cdot 2^{n}$, that is, the sum of these
%three bits has value at most $3$. Bit $u''_{n+1}$ is copied directly from
%$v'_{n+1}$ by the rules of CSA, which requires the following condition for
%the second modular reduction layer:
%$u'_{(n)} + v'_{(n,n+1)} \le 3\cdot 2^n$. This is true because
%$u'_{(n)} + v'_{(n+1)} = u_{(n)} + v_{(n)} \le 2$ and $v'_{(n)} \le 1$.
Everywhere
we use the fact that the modular residues are restricted to $n$ bits.
Therefore, the modular sum is computed as the sum of two $(n+2)$-bit numbers
with no overflows in constant-depth.
%\end{proof}
\square\,

As a side note, we can perform modular reduction in one layer instead of
three by decoding the three overflow bits into one of seven different
modular residues. This can also be done in constant depth, and in this case
we only need to enlarge all our registers to $(n+1)$ bits instead of $(n+2)$
as in the proof above. We omit the proof for brevity.

In the following two subsections, we give a concrete example to illustrate
the modular addition circuit as well as a numerical upper bound for the
general circuit resources.

%%%%%%%%%%%%%%%%%%%%%%%%%%%%%%%%%%%%%%%%%%%%%%%%%%%%%%%%%%%%%%%%%%%%%%%%%%%%%%%
\subsection{A Concrete Example of Modular Addition}
\label{subsec:concrete}

\begin{center}
\begin{figure*}[h!bt]
\centerline{
\includegraphics[width=6.5in]{./mod-add-fixed.pdf}
}
\fcaption{Addition and three rounds of modular reduction for a 3-bit
modulus.}
\label{fig:csa-add-4}
\end{figure*}
\end{center}

A \textsc{2D CCNTC} circuit for modular addition of $5$-bit numbers using
four layers of parallel CSA's is shown graphically in Figure \ref{fig:csa-add-4}
which corresponds directly to the schematic proof in Figure \ref{fig:csa-proof}.
Note that in Figure \ref{fig:csa-add-4}, the least significant qubits are
on the left, and in Figure \ref{fig:csa-proof}, the least significant qubits are
on the right.
Figure \ref{fig:csa-add-4} also represents the approximate
physical layout of the qubits as they would look if this
circuit were to be fabricated.
Here, we convert the sum of three
$5$-bit integers into the modular sum of two $5$-bit integers, with a
$3$-bit modulus $m$.
In the first layer,
we perform 4 CSA's in parallel on the input numbers ($a,b,c$) and produce the
output numbers ($u, v$).

As described above, we truncate
the three high-order bits during the initial CSA round
(bits $u_4, v_4, v_5$) to retain a $4$-bit number.
Each of these bits serves as a control for adding its modular residue to
a running total. We can classically precompute $2^4[m]$ for the two
additions controlled on $u_4$ and $v_4$ and
$2^5[m]$ for the addition controlled on $v_5$.

In Layer 2,
we use a constant-depth fanout rail (see Figure \ref{fig:cdf}) to
distribute the control bit $v_4$ to its modular residue, which we denote as
%%\begin{equation}
$\ket{c^{v_4}} \equiv \ket{2^4[m]\cdot v_4}$.
%%\end{equation}
%This fanout requires constant depth;
$c^{v_4}$ has $n$ bits, which we add to the CSE results of layer 1.
The results $u_i$ and $v_{i+1}$ are teleported into layer 3. The exception is
$v'_4$ which is teleported into layer 4, since there are no other $4$-bits
to which it can be added. Wherever there are only
two bits of the same significance, we use the 2-2 adder from
Section \ref{sec:csa}.

Layer 3
%%, shown in Figure \ref{fig:csa-add-3},
operates similarly to layer 2, except that the modular residue is controlled on
$u_4$:
%%\begin{equation}
$\ket{c^{u_4}} \equiv \ket{2^4[m] \cdot u_4}$.
%%\end{equation}
%This fanout again requires constant depth;
The qubit $c^{u_4}$ has $3$ bits, which we
add to the CSE results of layer 2, where $u'_i$ and $v'_{i+1}$ are teleported
forward into layer 4.

Layer 4
%%, shown in Figure \ref{fig:csa-add-4},
is similar to layers 2 and 3, with the modular residue controlled on $v_5$:
%%\begin{equation}
$\ket{c^{v_5}} \equiv \ket{2^5[m] \cdot v_5}$.
%%\end{equation}
%This fanout is constant depth;
The qubit $c^{v_5}$ has $3$ bits, which we
add to the CSE results of layer 3.
There is no overflow bit $v'''_5$, and no carry bit from $v''_4$ and $v'_4$
as argued in Lemma 1.
The final modular sum $(a+b+c)[m]$ is $u'''+v'''$.

The general circuit for adding three $n$-qubit quantum integers to
two $n$-qubit quantum integers, illustrated in Figure \ref{fig:csa-add-4}
for $n=3$, is called a \emph{CSA tile}. Each CSA tile in our architecture 
corresponds to its own module, and it will be represented by the symbol in 
Figure \ref{fig:csa-tile-symbol} for the rest of this paper.

\begin{center}
\begin{figure*}[h!bt]
\centerline{
\includegraphics[width=1.5in]{./csa-tile-symbol.pdf}
}
\fcaption{Symbol for an $n$-bit 3-to-2 modular adder, also called a CSA tile.}
\label{fig:csa-tile-symbol}
\end{figure*}
\end{center}


\subsection{Quantum Circuit Resources for Modular Addition}

We now calculate numerical upper bounds for the circuit resources of
the $(n+2)$-bit $3$-to-$2$ modular adder described in the previous section.
There are four layers of non-modular $n'$-bit $3$-to-$2$ adders, which
consists of $n'$ parallel single-bit adders whose
resources are detailed in Table \ref{tab:csa-tile-resources}. For factoring
an $n$-bit modulus, we have $n'=n+2$ in the first and fourth layers
and $n'=n+1$ in the second and third layers.

After each of the first three layers, we must move the output qubits
across the fanout rail to be the inputs of the next layer. We use
two swap gates, which have a depth and size of $6$ CNOTs each, since
the depth of teleportation is only more efficient for moving more than
two qubits. The control bit for each modular residue needs to be
teleported $0$, $4$, and $7$ qubits respectively according to the
diagram in Figure \ref{fig:csa-add-4}, before being fanned out $n$
times along the fanout rails, where the fanned out copies will end up
in the correct position to be added as inputs.

The detailed resources for a Toffoli gate and the single-bit adder that uses
them are given in Table \ref{tab:csa-tile-resources}.

The resources for the $n$-bit $3$-to-$2$ modular adder depicted in Figure
\ref{fig:csa-add-4} is more complicated due to the un-fanout procedure.
The formulae below reflect the resources needed for both computing the output
in the forward direction and also uncomputing ancillae in the backward
direction.

The circuit depth is:

\begin{equation}
356 + 8\log_2(2n+4)\text{.}
\end{equation}

The circuit size is:

\begin{equation}
33n\log_2 n + 40\log^2 n + 575n + 752\text{.}
\end{equation}

The circuit width is:

\begin{equation}
33n + 47\text{.}
\end{equation}

% From Notebook #16, pp. 68-69
\begin{table}
\begin{displaymath}
\begin{tabular}{|c|c|c|c|}
\hline
\text{Circuit Name} & \text{Depth} & \text{Size} & \text{Width} \\
\hline
\text{Toffoli gate from \cite{Amy2012}} & 8 & 15 & 3 \\
\hline
\text{Single-bit } 3\text{-to-}2 \text{ adder from Figure \ref{fig:csa-circuit}} & 33 & 55 & 5 \\
%\hline
%$n$ \text{-bit modular } 3\text{-to-}2 \text{ adder from Figure \ref{fig:csa-add-4}} & 356 & 572n + 724 & 33n+47 \\
\hline
\end{tabular}
\end{displaymath}
\centerline{}
\tcaption{Circuit resources for Toffoli and single-bit addition.}
\label{tab:csa-tile-resources}
\end{table}


%\section{Quantum Modular\\ Multiplication}
\label{sec:csa-mod-mult}

We can build upon our carry-save adder to implement quantum modular
multiplication in logarithmic depth. We start with a completely classical
problem to illustrate the principle of multiplication by repeated addition.
Then we consider modular multiplication of two quantum numbers in a serial
and a parallel fashion in
\ref{subsec:csa-mod-mult-qq}. Both of these problems use as a subroutine the
generic problem of \emph{modular multiple addition} which we define and solve
in \ref{subsec:mma}.

%%%%%%%%%%%%%%%%%%%%%%%%%%%%%%%%%%%%%%%%%%%%%%%%%%%%%%%%%%%%%%%%%%%%%%%%%%%%%%%
First we consider a completely classical problem:
given three $n$-bit classical numbers $a$, $b$, and $m$,
compute $c = ab \bmod m$, where $c$ is allowed to be in CSE.

We only have to add shifted
multiples of $a$ to itself, ``controlled'' on the bits of $b$. There are
$n$ shifted multiples of $a$, let's call them $z^{(i)}$, one for every bit of $b$:
%%\begin{equation}
$z^{(i)} = 2^i a b_i \bmod m$.
%%\end{equation}
We can parallelize the addition of $n$ numbers in a logarithmic depth
binary tree to get a total depth of $O(\log n)$.

%%%%%%%%%%%%%%%%%%%%%%%%%%%%%%%%%%%%%%%%%%%%%%%%%%%%%%%%%%%%%%%%%%%%%%%%%%%%%%
\subsection{Modular Multiplication of\\ Two Quantum Numbers}
\label{subsec:csa-mod-mult-qq}

We now consider the problem of multiplying a classical number controlled
on a quantum bit and a
\emph{quantum number},\footnote{In this paper, quantum
numbers often result by entangling a classical number in one register with a
quantum control bit. This should not be confused with the physics meaning
of a quantum number.} which is a
quantum superposition of classical numbers:
%\begin{quote}
given an $n$-qubit quantum number $\ket{x}$, a control qubit $\ket{p}$,
and two $n$-bit classical numbers $a$
and $m$,
compute $\ket{c} = \ket{xa[m]}$, where $c$ is allowed to be in CSE.
This problem occurs naturally in modular exponentiation (described in
the next section) and can be considered \emph{serial multiplication},
in that $t$ quantum numbers are multiplied in series to a single
quantum register.
%\end{quote}

We first create $n$ quantum numbers $\ket{z^{(i)}}$,
which are shifted multiples of the classical number $a$ controlled on the bits
of $x$:
%\begin{equation}
$\ket{z^{(i)}} \equiv \ket{2^i a[m] \cdot x_i }$.
%\end{equation}
How do we create these numbers, and what is the depth of the procedure?
First, note that $\ket{2^i a[m]}$ is a classical number, so we can
precompute this classically and prepare them in parallel using single-qubit
operations
on $n$ registers, each consisting of $n$ ancillae qubits. Each $n$-qubit
register will hold a future $\ket{z^{(i)}}$ value.
We then copy all
$n$ bits of $x$, $n$ times each, using an unbounded fanout operation so that
$n$ copies of each bit $\ket{x_i}$ is next to register $\ket{z^{(i)}}$.
This takes a total of $O(n^2)$ parallel CNOT operations.
We then entangle each $\ket{z^{(i)}}$ with the corresponding $x_i$.
The schematic for this is shown in Figure \ref{fig:mod-mult-create}, not
showing how we interleave these numbers into groups of three using
constant-depth teleportation. This reduces to the task of modular
multiple addition, in order to add these numbers down to a single
number modulo $m$, which is described in \ref{subsec:mma}.

\begin{figure*}[htp!]
\centerline{
\includegraphics[width=4.5in]{figures/znumbers.pdf}
}
\caption{Creating $n=4$ shifted values $\{z^{(0)},z^{(1)},z^{(2)},z^{(3)}\}$
for an input number $x$.}
\label{fig:mod-mult-create}
\end{figure*}

Finally, we tackle the most interesting problem:
%\begin{quote}
given two $n$-qubit quantum numbers $\ket{x}$ and
$\ket{y}$ and a $n$-bit classical number
$m$,
compute $\ket{c} = \ket{xy \bmod m}$,
where $\ket{c}$ is allowed to be in CSE.
This can be considered \emph{parallel multiplication} and is responsible
for our logarithmic speedup in modular exponentiation.
%\end{quote}

Instead of creating $n$ quantum numbers $\ket{z^{(i)}}$, we must create
$n^2$ numbers
$\ket{z^{i,j}}$ for all possible pairs of quantum bits $x_i$ and $y_j$,
$i,j \in \{0,\ldots,n-1\}$:
%\begin{equation}
$\ket{z^{i,j}} \equiv \ket{2^i2^j[m]\cdot x_i \cdot y_j}$.
%\end{equation}
We create these numbers using a similar procedure to the previous problem.
Adding $n^2$ quantum numbers of $n$ qubits each takes depth
$O(\log(n^2))$ which is still $O(\log n)$.
Creating $n^2\times n$-bit quantum numbers takes width $O(n^3)$. 

%%%%%%%%%%%%%%%%%%%%%%%%%%%%%%%%%%%%%%%%%%%%%%%%%%%%%%%%%%%%%%%%%%%%%%%%%%%%%%%
\subsection{Modular Multiple Addition}
\label{subsec:mma}

As a subroutine to modular multiplication, we define the operation of
repeatedly adding multiple numbers down to a single CSE number, called
\emph{modular multiple addition}.

The modular multiple addition circuit generically adds down $t\times n$-bit
conventional numbers to an $n$-bit CSE number.
%
\begin{equation}
z^{(0)} + z^{(1)} + \ldots z^{(n-1)} \equiv (u+v)[m]
\end{equation}
%
It does not matter how the
$t$ numbers are generated, as long as they are divided into groups of three
and have their bits interleaved to be the inputs of a CSA tile.
In the cases above, serial multiplication results in
$t = n$ and parallel multiplication results in $t = n^2$.
At the beginning of the circuit, all CSA tiles are
\emph{active} in that they have tile input numbers $z^{(i)}$
to multiply, and their tile outputs will affect the overall circuit output,
$u+v$.

As the circuit proceeds through a number of timesteps,
tiles will become \emph{inactive} when they
do not receive new numbers for their tile inputs; at
that point, their tile outputs can no longer affect the circuit output.
Since the CSA tile is a 3-2 adder, one can see that if there are $t$ CSA tiles
active at the beginning of a timestep, there are $\lceil 2t/3 \rceil$ active
tiles at the end of the timestep, since there are roughly two-thirds as many
input numbers left to add down to the circuit output $u+v$. One can see that
the total number of timesteps
is therefore $\lceil \log_{3/2}(t/3) \rceil + 1$.

To facilitate the below discussion, we will assign colors to each CSA tile,
which are updated during the circuit execution. Active tiles can either be
black or gray.
A \emph{black} tile will keep its two output numbers as inputs and receive
a third input number. An exception is the rightmost black tile may teleport
one of its output numbers to its left black nearest neighbor and receive
two input numbers from its right gray nearest neighbor.
A \emph{gray} tile will teleport one of its output
numbers to the nearest active tile to its left and the other output number
to the nearest active tile to its right. An exception is the rightmost gray
tile may teleport both output numbers to its left black nearest neighbor.
We can think of inactive tiles as
\emph{white} tiles in that they ``fade'' out of the circuit, and numbers
get teleported through them without stopping to be added. The symbols for
these colors are shown in Figure \ref{fig:tile-colors}.

\begin{figure*}[htb!]
\centerline{
\includegraphics[width=5.5in]{figures/csa-tile-colors.pdf}
}
\caption{From left to right, the symbols for a black, gray, and white tile,
respectively.}
\label{fig:tile-colors}
\end{figure*}


The rules for
updating tiles at the end of each timestep are as follows:

\begin{itemize}
\item \textbf{Black tiles} are always active for the next timestep, but
change colors as follows.
\begin{itemize}
\item The leftmost tile always stays black.
\item If a black tile has a gray tile as its nearest active right neighbor in
the current timestep,
it stays \textbf{black} in the next timestep.
\item If a black tile has a black tile as its nearest active neighbor either
to the right or the left, and it is not the leftmost tile,
it turns \textbf{gray} in the next timestep.
\end{itemize}
\item \textbf{Gray tiles} always turn white (inactive) in the next timestep.
\end{itemize}

The initial state of the tile colors depends on its index
$i \in \{0, 1, \ldots, q-1\}$ within $q = \lceil t/3 \rceil$ tiles.

\begin{itemize}
\item If $i \bmod 3 = 0$, then it starts out black.
\item If $i \bmod 3 = 1$, then it starts out gray.
\item If $i \bmod 3 = 2$, then it starts out black.
\end{itemize}

Given the rules above, one can see that the leftmost tile stays black
throughout the entire circuit, and holds the final output number $(u+v)$ at
the end.

Each timestep of the circuit consists of the following operations:

\begin{enumerate}
\item
All active CSA tiles will execute in
parallel to transform their three input numbers into two output numbers
(a CSE number).
\item
Gray tiles teleport their output numbers to the left and to the right to
their black tile neighbors. The exception is the rightmost gray tile will
teleport both of its output numbers to its left black tile nearest neighbor.
\item
Tile colors will change according to the rules above. Approximately
two-thirds of the tiles will
become inactive in the next timestep.
\item
Go back to Step 1 for the next timestep.
\end{enumerate}

These steps and the above tile color rules are best illustrated with a concrete
example. In Figure \ref{fig:mod-mult}, we see the circuit for modular
multiple addition as a series of
snapshots, separated by heavy dotted lines, with the passage of time going
downward. The tiles change color over time, and the arrows indicate the
teleportation of output numbers to neighboring active tiles in each timestep.
In the initial timestep, the tiles are numbered to show how they are assigned
their initial color.
Between Timestep 0 and Timestep 1,
all $\lceil n/3 \rceil$ CSA tiles are active. After each succeeding timestep,
$\lfloor 2/3 \rfloor$
fewer CSA tiles are active until the very end, when only one CSA tile is
active. By the convention established above,
we teleport the rightmost output numbers to the left, so that the
final output is read out from the leftmost CSA tile.

\begin{figure*}[htb!]
\centerline{
\includegraphics[width=5.5in]{figures/mod-mult-add.pdf}
}
\caption{Modular multiple addition of quantum numbers on a CSA tile
architecture for $t=18$ with depth $(\lceil \log_{\frac{3}{2}}(t/3) \rceil + 1) = 6$
timesteps}
\label{fig:mod-mult}
\end{figure*}
%

Now we can analyze the circuit resources for multiplying $n$-bit
quantum numbers, which requires $(t-2)$ modular additions, for $t=n^2$.
The circuit width is the sum of the $O(n^3)$ ancillae
needed for number generation and the ancillae required for $O(n^2)$
modular additions. Each modular addition has width $O(n)$ and depth $O(1)$
from the previous
section. There are
$\lceil \log_{3/2}(n^2 / 3) \rceil +1 $ timesteps of modular addition. Therefore
the entire modular multiplier circuit has depth $O(\log n)$ and width $O(n^3)$.

\section{Quantum Modular Multiplication}
\label{sec:csa-mod-mult}

We can build upon our carry-save adder to implement quantum modular
multiplication in logarithmic depth. We start with a completely classical
problem to illustrate the principle of multiplication by repeated addition.
Then we consider modular multiplication of two quantum integers in a serial
and a parallel fashion in Section
\ref{subsec:csa-mod-mult-qq}. Both of these problems use as a subroutine the
generic problem of \emph{modular multiple addition} which we define and solve
in Section \ref{subsec:mma}.

%%%%%%%%%%%%%%%%%%%%%%%%%%%%%%%%%%%%%%%%%%%%%%%%%%%%%%%%%%%%%%%%%%%%%%%%%%%%%%%
First we consider a completely classical problem:
given three $n$-bit classical numbers $a$, $b$, and $m$,
compute $c = ab \bmod m$, where $c$ is allowed to be in CSE.

We only have to add shifted
multiples of $a$ to itself, ``controlled'' on the bits of $b$. There are
$n$ shifted multiples of $a$, let's call them $z^{(i)}$, one for every bit of $b$:
%%\begin{equation}
$z^{(i)} = 2^i a b_i \bmod m$.
%%\end{equation}
We can parallelize the addition of $n$ numbers in a logarithmic depth
binary tree to get a total depth of $O(\log n)$.

%%%%%%%%%%%%%%%%%%%%%%%%%%%%%%%%%%%%%%%%%%%%%%%%%%%%%%%%%%%%%%%%%%%%%%%%%%%%%%
\subsection{Modular Multiplication of Two Quantum Integers}
\label{subsec:csa-mod-mult-qq}

We now consider the problem of multiplying a classical number controlled
on a quantum bit with a
\emph{quantum integer}\footnote{In this paper, quantum integers often result by entangling a classical number in one register with a quantum control bit.},
which is a
quantum superposition of classical numbers:
%\begin{quote}
Given an $n$-qubit quantum integer $\ket{x}$, a control qubit $\ket{p}$,
and two $n$-bit classical numbers $a$
and $m$,
compute $\ket{c} = \ket{xa[m]}$, where $c$ is allowed to be in CSE.
This problem occurs naturally in modular exponentiation (described in
the next section) and can be considered \emph{serial multiplication},
in that $t$ quantum integers are multiplied in series to a single
quantum register. This is the approach used in serial QPF as mentioned in
Section \ref{sec:related}.
%\end{quote}

We first create $n$ quantum integers $\ket{z^{(i)}}$,
which are shifted multiples of the classical number $a$ controlled on the bits
of $x$:
%\begin{equation}
$\ket{z^{(i)}} \equiv \ket{2^i a[m] \cdot x_i }$.
%\end{equation}
These are typically called \emph{partial products} in a classical multiplier.
How do we create these numbers, and what is the depth of the procedure?
First, note that $\ket{2^i a[m]}$ is a classical number, so we can
precompute them classically and prepare them in parallel using single-qubit
operations
on $n$ registers, each consisting of $n$ ancillae qubits. Each $n$-qubit
register will hold a future $\ket{z^{(i)}}$ value.
We then fan out each of the
$n$ bits of $x$, $n$ times each, using an unbounded fanout operation so that
$n$ copies of each bit $\ket{x_i}$ are next to register $\ket{z^{(i)}}$.
This takes a total of $O(n^2)$ parallel CNOT operations.
We then entangle each $\ket{z^{(i)}}$ with the corresponding $x_i$.
%The schematic for this is shown in Figure \ref{fig:mod-mult-create}.
After this, we interleave these numbers into groups of three using
constant-depth teleportation. This reduces to the task of modular
multiple addition in order to add these numbers down to a single
number modulo $m$, which is described in \ref{subsec:mma}.

%\begin{figure*}[htp!]
%\centerline{
%\includegraphics[width=4.5in]{./znumbers.pdf}
%}
%\fcaption{Creating $n=4$ shifted values $\{z^{(0)},z^{(1)},z^{(2)},z^{(3)}\}$
%for an input number $x$.}
%\label{fig:mod-mult-create}
%\end{figure*}

Finally, we tackle the most interesting problem:
%\begin{quote}
given two $n$-qubit quantum integers $\ket{x}$ and
$\ket{y}$ and a $n$-bit classical number
$m$,
compute $\ket{c} = \ket{xy \bmod m}$,
where $\ket{c}$ is allowed to be in CSE.
This can be considered \emph{parallel multiplication} and is responsible
for our logarithmic speedup in modular exponentiation and parallel QPF.
%\end{quote}

Instead of creating $n$ quantum integers $\ket{z^{(i)}}$, we must create
up to $n^2$ numbers
$\ket{z^{i,j}}$ for all possible pairs of quantum bits $x_i$ and $y_j$,
$i,j \in \{0,\ldots,n-1\}$:
%\begin{equation}
$\ket{z^{i,j}} \equiv \ket{2^i2^j[m]\cdot x_i \cdot y_j}$.
%\end{equation}
We create these numbers using a similar procedure to the previous problem,
but using a more detailed procedure given in Section \ref{subsec:ppc}.
Adding $n^2$ quantum integers of $n$ qubits each takes depth
$O(\log(n^2))$, which is still $O(\log n)$.
Creating $n^2\times n$-bit quantum integers takes width $O(n^3)$.
Numerical constants are given for these resource estimates in
Section \ref{subsec:mod-mult-resources}.

%%%%%%%%%%%%%%%%%%%%%%%%%%%%%%%%%%%%%%%%%%%%%%%%%%%%%%%%%%%%%%%%%%%%%%%%%%%%%%%
\subsection{Partial Product Creation}
\label{subsec:ppc}

This subroutine describes the procedure of creating $t=O(n^2)$ partial products of
the CSE quantum integers $x$ and $y$, each with $2n+3$ bits each. We will now
discuss only the case of parallel multiplication. Although we
will not provide an explicit circuit for this subroutine, we will outline
our particular construction and give a numerical upper bound on the
resources required. We do not argue that our construction is optimal, but
this is an interesting open problem for future work.

Our construction is as follows. First, we need to generate the product bits
$\ket{x_i\cdot y_j}$ for all possible $(2n+3)^2$ pairs of $\ket{x_i}$ and
$\ket{y_j}$.
A particular product bit $\ket{x_i \cdot y_j}$
controls a particular classical number, the
$n$-bit modular residue $2^i 2^j [m]$, to form the partial product
$\ket{z^{(i,j)}}$ defined
in the previous section. However, some of these partial products
consist of only a single qubit, if $2^i 2^j < 2^n$, which is the minimum
value for an $n$-bit modulus $m$. There are at least $2n^2 - 2n + 1$
such single-bit partial products, which can be grouped into at most
$(2n+3)\times n$-bit numbers. Of the $(2n+3)^2$ possible partial products,
this leaves the number of remaining $n$-bit partial products as at most
$2n^2 + 14n +8$. Therefore we have a final maximum number of $n$-bit
partial products, which we will simply refer to as $t'$ from now on.

\begin{equation}
t'=2n^2+16n+11
\label{eqn:tprime}
\end{equation}

The creation of the product bits $\ket{x_i \cdot y_j}$ occurs on a
square lattice of $(3(2n+3))^2$, with the bits $x_i$ and $y_j$ on adjacent edges. The size of the lattice is not just $(2n+3)^2$ in order to allow
the $\ket{x_i}$ and $\ket{y_j}$ bits move past each other, as they are
teleported along axes that are perpendicular to each other.
Product bit creation, and this square lattice, comprise a single module.
In several
rounds, these bits are copied via a CNOT and teleported to the middle of
a recursively halved interval of the grid. The copied bits $\ket{x_i}$ and
$\ket{y_j}$
first form $1$ line, then $3$ lines, then $7$ lines, and so forth,
intersecting at $1$ site, then $9$ sites, then $49$ sites, and so forth.
There are $\lceil \log_2 (2n+3) \rceil$ rounds.

At each intersection, a Toffoli gate is used to create $\ket{x_i \cdot y_j}$
from the given $\ket{x_i}$ and $\ket{y_j}$. These product bits are then
teleported away from this qubit, out of this product bit module, to a different
module where the $\ket{z^{(i,j)}}$ numbers are later generated,
called $z$-sites. There are $t'$ $z$-site modules which each contain 
an $n$-qubit quantum integer. Any
round of partial product generation will produce at most as any product
bits $x_i \cdot y_j$ as in the last round, which is half the total number
of $(2n+3)^2$.
%These product bits are teleported out the two sides of the
%square lattice that are opposite the input numbers $x$ and $y$, which means the
%square lattice has dimension at most $((2(2n+3)-1)(n+2))^2 = O(n^4)$.

%The $z$-sites have total width of $3nt'$ qubits, so the maximum total
%teleportation length of all qubits is $(2n+3)^2$ multiplied by the maximum
%length of any single teleportation length,
%Our construction consists of the following steps:

\begin{enumerate}
\item Initially, the inputs consist of the CSE quantum integers $x$ and $y$,
each with $2n+3$ bits, sitting on adjacent edges of a square lattice that has
a length of $3(2n+3)$ qubits.
\item For each of $\lceil \log_2 (2n+3) \rceil$ rounds:
\begin{enumerate}
\item Of the existing $\{x_i\}$ and $\{y_j\}$ bits, apply a CNOT to create an
entangled copy in an adjacent qubit.
\item Teleport this new copy halfway between its current location and the
new copy.
\item At every site where an $\ket{x_i}$ and an $\ket{y_j}$ meet,
apply a Toffoli gate to create $\ket{x_i \cdot y_j}$.
\item Teleport $\ket{x_i \cdot y_j}$ to the correct $z$-site module.
\end{enumerate}
\item Within each $z$-site module, fanout $\ket{x_i \cdot y_j}$ up to $n$
times, corresponding to each $1$ in the modular residue $2^i 2^j \bmod m$,
to create the $n$-qubit quantum integer $\ket{z^{(i,j)}}$.
\item For each triplet of $z$-site modules, teleport the quantum integers
$\ket{z^{(i,j)}}$ to a CSA tile module, interleaving the three numbers so that
bits of the same significance are adjacent.
\item Perform modular multiple addition (described in Section \ref{subsec:mma})
on $t'$ $n$-qubit quantum integers down to 2 $n$-qubit quantum integers (one CSE number).
\item Uncompute all the previous steps to restore ancillae to $\ket{0}$.
\end{enumerate}

We now present the resources for partial product creation, the first half of
a modular multiplier, including the reverse computation.

The circuit depth is:

\begin{equation}
D_{PPC} = 40\log_2 10 n\text{.}
\end{equation}

The module depth is:

\begin{equation}
\overline{D}_{PPC} = 8\text{.}
\end{equation}

The circuit size is:

\begin{eqnarray}
S_{PPC} & = & (8n^3 + 76n^2 + 178n + 89)\log^2_2 (10n) +\\
        &   & (78n^3 + 597n^2 + 419n + 133)\text{.}
\end{eqnarray}

The module size is:

\begin{eqnarray}
\overline{S}_{PPC} = 6n^2 + 26n + 19\text{.}
\end{eqnarray}

The circuit width is:

\begin{eqnarray}
W_{PPC} = 6n^3 + 48n^2 - 8n + 1\text{.}
\end{eqnarray}

The module width is:

\begin{eqnarray}
\overline{W}_{PPC} = 2n^2 + 14n + 9\text{.}
\end{eqnarray}

%%%%%%%%%%%%%%%%%%%%%%%%%%%%%%%%%%%%%%%%%%%%%%%%%%%%%%%%%%%%%%%%%%%%%%%%%%%%%%%
\subsection{Modular Multiple Addition}
\label{subsec:mma}

As a subroutine to modular multiplication, we define the operation of
repeatedly adding multiple numbers down to a single CSE number, called
\emph{modular multiple addition}.

The modular multiple addition circuit generically adds down $t'\times n$-bit
conventional numbers to an $n$-bit CSE number:
%
\begin{equation}
z^{(1)} + z^{(2)} + \ldots z^{(t')} \equiv (u+v)[m].
\end{equation}
%
It does not matter how the
$t'$ numbers are generated, as long as they are divided into groups of three
and have their bits interleaved to be the inputs of a CSA tile.
From the previous section, serial multiplication results in
$t' \le n$ and parallel multiplication results in $t' \le n^2$. Each CSA tile
is contained in its own module. These modules are arranged in layers within
a logarithmic depth binary tree, where 
the first layer contains $\lceil t'/3 \rceil$ modules. A modular addition
occurs in all the modules of the first layer in parallel. The outputs from this
first layer are then teleported to be the inputs of the next layer of modules,
which have at most two-thirds as many modules. This continues until the
tree terminates in a single module, whose output is a CSE number $u+v$ which
represents the modular product of all the original $t'$ numbers. The resulting
height of the tree is $(\lceil \log_{3/2}(t'/3) \rceil + 1)$ modules.

As the parallel modular additions proceed by layers, all previous layers
must be maintained in a coherent state, since the modular addition leaves
garbage bits behind. Only at the end of modular multiple addition, after
the final answer $u+v$ is obtained, can all the previous layers be
uncomputed in reverse to free up their ancillae.

These steps are best illustrated with a concrete
example in Figure \ref{fig:mod-mult}. The module for each CSA tile is
represented by the symbol from Figure \ref{fig:csa-tile-symbol}.
The arrows indicate the
teleportation of output numbers from the source tile to be input numbers
into a destination tile.

\begin{figure*}[htb!]
\centerline{
\includegraphics[width=5.5in]{./mod-mult-add.pdf}
}
\fcaption{Modular multiple addition of quantum integers on a CSA tile
architecture for $t'=18$ in a logarithmic-depth tree with height $(\lceil \log_{\frac{3}{2}}(t'/3) \rceil + 1) = 6$. Arrows represent teleportation
in between modules.}
\label{fig:mod-mult}
\end{figure*}
%

Now we can analyze the circuit resources for multiplying $n$-bit
quantum integers, which requires $(t'-2)$ modular additions, for $t'$ from
Equation \ref{eqn:tprime}.
The circuit width is the sum of the $O(n^3)$ ancillae
needed for partial product creation and the ancillae required for $O(n^2)$
modular additions. Each modular addition has width $O(n)$ and depth $O(1)$
from the previous
section. There are
$\lceil \log_{3/2}(n^2 / 3) \rceil +1 $ timesteps of modular addition. Therefore
the entire modular multiplier circuit has depth $O(\log n)$ and width $O(n^3)$.

\subsection{Modular Multiplier Resources}
\label{subsec:mod-mult-resources}.

The circuit depth of the entire modular multiplier is:

\begin{equation}
D_{MM} = 3.4 \log_2^3 n + 1295 \log_2 n + 6911\text{.}
\end{equation}

The module depth is:
\begin{equation}
\overline{D}_{MM} = 2\log_2 n + 11\text{.}
\end{equation}

The circuit size is:

\begin{eqnarray}
S_{MM} = & (12n^2 + 644n + 288)\log_2^2 (10n) +\\
        & (74n^3 + 606n^2 + 292n + 41)\log_2 (10n) +\\
        & (1228n^3 + 10151n^2 + 14397n + 4645)\text{.}
\end{eqnarray}

The module size is:

\begin{equation}
\overline{S}_{MM} = 15n^3 + 127n^2 + 178n + 50{.}
\end{equation}

The circuit width is:

\begin{equation}
W_{MM} = 66n^3 + 558n^2 + 870n + 290\text{.}
\end{equation}

The module width is

\begin{equation}
\overline{W}_{MM} = 4n^2 + 28n + 15\text{.}
\end{equation}

%\section{Quantum Modular\\ Exponentiation}
\label{sec:modexp}

We now extend our arithmetic to modular exponentiation, which is repeated
modular multiplication controlled on qubits supplied by a phase estimation
procedure.
If we wish to multiply a $n$-qubit quantum input number $\ket{x}$ by
$t$ classical numbers $a^{(j)}$, we can multiply them in series.
% as shown in
%Figure \ref{fig:modexp-qc-series}.
This requires depth $O(t\log n)$ based on the modular multipliers in previous
sections.

%\begin{figure}[htp!]
%\begin{center}
%\includegraphics[width=5.5in]{figures/modexp-qc-series.pdf}
%\end{center}
%\caption{Multiplying a quantum number $\ket{x}$ by $t$ classical numbers
%$\{a^{0}, a^{1}, \ldots, a^{n-1}\}$ in series.}
%\label{fig:modexp-qc-series}
%\end{figure}

Now consider the same procedure, but this time each classical number $a^{(j)}$
is controlled on a quantum bit $p_j$. This is a special case of
multiplying by $t$ quantum numbers in series, since a classical number
entangled with a quantum number is also quantum.
%This is shown in
%Figure \ref{fig:modexp-qq-series}.
It takes the same depth $O(t\log n)$ as the previous case.
%
%\begin{figure}[htp!]
%\begin{center}
%\includegraphics[width=5.5in]{figures/modexp-qq-series.pdf}
%\end{center}
%\caption{Multiplying a quantum number $\ket{x}$ by $t$ quantum numbers
%$\{\ket{a^{0}p_0}, \ket{a^{1}p_1}, \ldots, \ket{a^{n-1}p_{n-1}}\}$ in series.}
%\label{fig:modexp-qq-series}
%\end{figure}

Finally, we consider multiplying $t$ quantum numbers
$\{x^{(0)}, x^{(1)}, \ldots, x^{(n-1)}\}$ in a parallel,
logarithmic depth, binary tree.
This is shown in Figure \ref{fig:modexp-qq-parallel}.
The tree has depth $\log_2(t)$ in modular multiplier operations. Furthermore,
each
modular multiplier operation has depth $O(\log(n))$ for $n$-qubit
numbers. Therefore, the overall depth of this parallel modular exponentiation
structure is $O(\log(t)\log(n))$. In phase estimation for QPF, it is
sufficient to take $t = O(n)$ \cite{Nielsen2000,Kitaev2002}. Therefore our total depth is
$O(\log^2(n))$ as desired. At this point, combined with the parallel phase
estimation procedure of \cite{Kitaev2002}, we have a complete factoring
implementation in our 2D nearest-neighbor architecture.
%
\begin{figure*}[tb!]
\centerline{
\includegraphics[width=5.5in]{figures/mod-exp-par.pdf}
}
\caption{Parallel modular exponentiation: multiplying $t$ quantum numbers
%$\{\ket{x^{(0)}}, \ket{x^{(1)}}, \ldots, \ket{x^{(t-1)}}\}$ in parallel,
in a $O(\log{(t)}\log{(n)})$-depth binary tree.}
\label{fig:modexp-qq-parallel}
\end{figure*} 

For all known QPF procedures, there are $t=O(n)$ control bits needed, and
also $O(n)$ modular multiplications in a tree of depth $O(\log n)$.
Each modular multiplication has
depth $O(\log n)$ and width $O(n^3)$.
Therefore, the depth of the parallel modular exponentiation circuit above
is $O(\log^2 n)$ and the width is $O(n^4)$.

\section{Quantum Modular Exponentiation}
\label{sec:modexp}

We now extend our arithmetic to modular exponentiation, which is repeated
modular multiplication controlled on qubits supplied by a phase estimation
procedure.
If we wish to multiply an $n$-qubit quantum input number $\ket{x}$ by
$t$ classical numbers $a^{(j)}$, we can multiply them in series.
% as shown in
%Figure \ref{fig:modexp-qc-series}.
This requires depth $O(t\log n)$ based on the modular multipliers in previous
sections.

%\begin{figure}[htp!]
%\begin{center}
%\includegraphics[width=5.5in]{figures/modexp-qc-series.pdf}
%\end{center}
%\caption{Multiplying a quantum number $\ket{x}$ by $t$ classical numbers
%$\{a^{0}, a^{1}, \ldots, a^{n-1}\}$ in series.}
%\label{fig:modexp-qc-series}
%\end{figure}

Now consider the same procedure, but this time each classical number $a^{(j)}$
is controlled on a quantum bit $p_j$. This is a special case of
multiplying by $t$ quantum integers in series, since a classical number
entangled with a quantum integer is also quantum.
%This is shown in
%Figure \ref{fig:modexp-qq-series}.
It takes the same depth $O(t\log n)$ as the previous case.
%
%\begin{figure}[htp!]
%\begin{center}
%\includegraphics[width=5.5in]{figures/modexp-qq-series.pdf}
%\end{center}
%\caption{Multiplying a quantum number $\ket{x}$ by $t$ quantum numbers
%$\{\ket{a^{0}p_0}, \ket{a^{1}p_1}, \ldots, \ket{a^{n-1}p_{n-1}}\}$ in series.}
%\label{fig:modexp-qq-series}
%\end{figure}

Finally, we consider multiplying $t$ quantum integers
$\{x^{(1)}, x^{(2)}, \ldots, x^{(t-1)}, x^{(t)}\}$ in a parallel,
logarithmic-depth binary tree.
This is shown in Figure \ref{fig:modexp-qq-parallel}, where arrows indicate multiplication.
The tree has depth $\log_2(t)$ in modular multiplier operations. Furthermore,
each
modular multiplier operation has depth $O(\log(n))$ for $n$-qubit
numbers. Therefore, the overall depth of this parallel modular exponentiation
structure is $O(\log(t)\log(n))$. In phase estimation for QPF, it is
sufficient to take $t = O(n)$ \cite{Nielsen2000,Kitaev2002}. Therefore our total depth is
$O(\log^2(n))$ as desired. At this point, combined with the parallel phase
estimation procedure of \cite{Kitaev2002}, we have a complete factoring
implementation in our 2D nearest-neighbor architecture.
%
\begin{figure*}[tb!]
\centerline{
\includegraphics[width=5.5in]{./mod-exp-par.pdf}
}
\fcaption{Parallel modular exponentiation: multiplying $t$ quantum integers
%$\{\ket{x^{(0)}}, \ket{x^{(1)}}, \ldots, \ket{x^{(t-1)}}\}$ in parallel,
in a $O(\log{(t)}\log{(n)})$-depth binary tree. Arrows indicate modular
multiplication.}
\label{fig:modexp-qq-parallel}
\end{figure*}

For all known QPF procedures, there are $t=O(n)$ control bits needed, and
also $O(n)$ modular multiplications in a tree of depth $O(\log n)$.
Each modular multiplication has
depth $O(\log n)$ and width $O(n^3)$.
Therefore, the depth of the parallel modular exponentiation circuit above
is $O(\log^2 n)$ in elementary gates and the width is $O(n^4)$. We will now calculate numerical
constants to upper bound circuit resources.

According to the Kitaev-Shen-Vyalyi parallelized phase estimation procedure
\cite{Kitaev2002},
for a constant success probability of $3/4$,
it is sufficient to multiply together $t' = 2867n$ quantum integers,
controlled on the qubits $\ket{p_j}$, in parallel.

In Section \ref{subsec:qcla}, we describe the last step of modular
exponentiation in CSE. In Section \ref{subsec:modexp-resources}, we
state the final circuit resources for the entire modular exponentiation
circuit,
and therefore, our quantum period-finding procedure.

%%%%%%%%%%%%%%%%%%%%%%%%%%%%%%%%%%%%%%%%%%%%%%%%%%%%%%%%%%%%%%%%%%%%%%%%%%%%%%
\subsection{Converting Back to a Unique Conventional Number}
\label{subsec:qcla}

The final product of all $t$ quantum integers is in CSE which is not
unique. As stated in Gossett's original paper \cite{Gossett1998}, this
must be converted back to a conventional number using, for example, the
quantum carry-lookahead adder (QCLA) from \cite{Draper2004}. We can convert
this to a nearest-neighbor architecture by using the qubit reordering
construction of \cite{Rosenbaum2012}. We now compute the resources
needed for this last step.

To add two $(n+2)$-bit numbers in a QCLA, we have a circuit width of
$k = (4(n+2) - 2\log_2 n - 1)$. The depth is at most $4\log_2 n +2$ gates,
and some of them act on qubits that are not nearest-neighbors. Therefore,
we add in between each gate a reordering circuit that takes $k^2$
(reusable) ancillae
qubits and uses two rounds of constant-depth teleportation to rearrange
the qubits into a new order where all the gates are nearest-neighbor.
Adding in the teleportation circuit resources from Table \ref{tab:cd-resources},
we can calculate the following resources.

The circuit depth is:

\begin{equation}
56\log_2 n + 28\text{.}
\end{equation}

The circuit size is:

\begin{eqnarray}
96 \log_2^3 n & - & (384n + 624)\log_2^2 n \nonumber \\
              & + & (384n^2 + 1152n + 840) \log_2 n \nonumber \\
              & + & (192n^2 + 672n + 588)\text{.}
\end{eqnarray}

The circuit width is:

\begin{equation}
4 \log_2^2 n - (16n + 30)\log_2 n + 16n^2 + 60n + 56\text{.}
\end{equation}


%%%%%%%%%%%%%%%%%%%%%%%%%%%%%%%%%%%%%%%%%%%%%%%%%%%%%%%%%%%%%%%%%%%%%%%%%%%%%%
\subsection{Circuit Resources for Modular Exponentiator}
\label{subsec:modexp-resources}

This leads to the following circuit resources for a modular exponentiator.

The circuit depth is:

% From Notebook #16 p. 225
\begin{equation}
D_{ME} = 3.4\log_2^3 n + 1337\log_2^2 n + 23070\log_2 n + 86279\text{.}
\end{equation}

The module depth is:

\begin{equation}
\overline{D}_{ME} = 3\log_2 n + 24
\end{equation}

The circuit size is:

\begin{eqnarray}
S_{ME} & = & 96 \log_2^3 n + \nonumber \\
       &   & (34400 n^3 + 1846079 n^2 + 824937 n - 288)\log_2^2 10n + \nonumber \\
       &   & (212129 n^4 + 1737086 n^3 + 836442 n^2 + 117239 n - 41) \log_2 10n + \nonumber \\
       &   & 3520185 n^4 + 29097629 n^3 + 41260290 n^2 + 13300960 n - 4645\text{.}
\end{eqnarray}

The module size is:

\begin{equation}
\overline{S}_{ME} = 5749n^2 + 8725n +175\text{.}
\end{equation}

The circuit width is:

\begin{equation}
W_{ME} = 94598n^4 + 799749 n^3 + 1246692 n^2 + 415222 n - 145\text{.}
\end{equation}

The module width is:

\begin{equation}
\overline{W}_{ME} = 1434n\text{.}
\end{equation}

%\section{Results}
\label{sec:results}

The resources required for our approach,
as well as other nearest-neighbor approaches,
are listed in Table \ref{tab:results},
where the asymptotic resource bounds assume some fixed constant error
probability for each round of period-finding.
%, say $\epsilon=1/4$.
% and $\delta' = 1/2$ for KSV-QPF.
%Note that the
%number of measurements are included for completeness.
%, since these are
%not counted as gates in our model but may be comparable in terms of
%execution time.
%Some table cells are blank if the entries are not relevant to the current comparison, or if the entires were not %calculated in the prior work.
We achieve an exponential
improvement in nearest-neighbor circuit depth (from quadratic to polylogarithmic)
with our approach at the cost of a polynomial increase in
circuit width. Similar depth improvements at the cost of width increases can be achieved using the modular multipliers
of other factoring implementations
by arranging them in a parallel, KSV-style modular exponentiator.
%
\begin{table*}[htb!]
\begin{center}
\begin{tabular}{|c|c|c|c|}
\hline
Implementation             & Architecture      & Depth             & Width     \\
\hline
Vedral, Barenco \& Ekert \cite{Vedral1996}   & \textsc{AC}       & $O(n^3)$      & $O(n)$ \\
Gossett \cite{Gossett1998}                   & \textsc{AC}       & $O(n \log n)$ & $O(n^2)$  \\
Beauregard \cite{Beauregard2002}                & \textsc{AC}       & $O(n^3)$      & $O(n)$ \\
Zalka \cite{Zalka1998}                     & \textsc{AC}       & $O(n^2)$      & $O(n)$     \\
Takahashi \& Kunihiro \cite{Takahashi2006}     & \textsc{AC}       & $O(n^3)$      & $O(n)$ \\
%Cleve \& Watrous           & \textsc{AC}       & $O(\log L)$   & $O(L)$ \\
\hline
Fowler, Devitt, Hollenberg \cite{Fowler2004} & \textsc{1D NTC}   & $O(n^4)$ & $O(n^3)$\\
% + 40L^3 + 58L^2 + 2L - 2$ & $32L^3 + 80 L^2 - 4L - 2$ \\
Kutin \cite{Kutin2006}                     & \textsc{1D NTC}   & $O(n^2)$ & $O(n)$\\
\hline
%$18L^2 + 12L\log_2^2 L + O(L \log L) $ & $3L + 2\log_2 L + 2$ \\
Current Work               & \textsc{2D NTC}   & $O(\log^2{n})$ & $O(n^4)$   \\
\hline
\end{tabular}
\end{center}
\caption{Asymptotic resource usage for quantum factoring of an $n$-bit number.}
\label{tab:results}
\end{table*}

\section{Asymptotic Results}
\label{sec:results}

The asymptotic resources required for our approach,
as well as the resources for other nearest-neighbor approaches,
are listed in Table \ref{tab:results},
where we assume some fixed constant error
probability for each round of QPF. Not all resources are
provided directly by the referenced source.

Resources in square brackets
are inferred using Equation \ref{eqn:depth-width}.
These upper bounds are correct,
but may not be tight with the upper bounds
calculated by their respective authors.

A more detailed analysis
could give a better upper bound for circuit size. Also note that the
work by Beckman et al. \cite{Beckman1996} is unique in that it uses
efficient multi-qubit gates inherent to linear ion trap technology which at first
seem to
be more powerful than \textsc{1D NTC}. However, use of these gates does not result in an
asymptotic improvement \textsc{1D NTC}.

%, say $\epsilon=1/4$.
% and $\delta' = 1/2$ for KSV-QPF.
%Note that the
%number of measurements are included for completeness.
%, since these are
%not counted as gates in our model but may be comparable in terms of
%execution time.
%Some table cells are blank if the entries are not relevant to the current comparison, or if the entires were not %calculated in the prior work.
We achieve an exponential
improvement in nearest-neighbor circuit depth (from quadratic to polylogarithmic)
with our approach at the cost of a polynomial increase in
circuit size and width. Similar depth improvements at the cost of width increases can be achieved using the modular multipliers
of other factoring implementations
by arranging them in a parallel, KSV-style modular exponentiator.
Our approach is the first \textsc{2D NTC} implementation for factoring, with the
modifications of allowing a classical controller and parallel, communicating
modules (\textsc{2D CCNTCM}).
%
\begin{table}[htb!]
\begin{center}
\begin{tabular}{|c|c|c|c|c|}
\hline
Implementation             & Architecture      & Depth   & Size   & Width     \\
\hline
Vedral, et al. \cite{Vedral1996}   & \textsc{AC}      & $[O(n^3)]$ & $O(n^3)$    & $O(n)$ \\
Gossett \cite{Gossett1998}                   & \textsc{AC}       & $O(n \log n)$  & $[O(n^3\log n)]$  & $O(n^2)$  \\
Beauregard \cite{Beauregard2002}                & \textsc{AC}       & $O(n^3)$      & $O(n^3 \log n)$ & $O(n)$ \\
Zalka \cite{Zalka1998}                     & \textsc{AC}       & $O(n^2)$      & $[O(n^3)]$ & $O(n)$     \\
Takahashi \& Kunihiro \cite{Takahashi2006}     & \textsc{AC}       & $O(n^3)$      & $O(n^3\log n)$ & $O(n)$ \\
Cleve \& Watrous \cite{Cleve2000}           & \textsc{AC}       & $O(\log^3 n)$ & $O(n^3)$ & $[O(n^3 / \log^3n)]$ \\
\hline
Beckman et al. \cite{Beckman1996} & \textsc{Ion trap}   & $O(n^3)$ & $O(n^3)$ & $O(n)$\\
\hline
Fowler, et al. \cite{Fowler2004} & \textsc{1D NTC}   & $O(n^3)$ & $O(n^4)$ & $O(n)$\\
Van Meter \& Itoh \cite{VanMeter2006} & \textsc{1D NTC}   & $O(n^2 \log n)$ & $[O(n^4\log n)]$ & $O(n^2)$\\
Kutin \cite{Kutin2006}                     & \textsc{1D NTC}   & $O(n^2)$ & $O(n^3)$ & $O(n)$\\
\hline
Current Work               & \textsc{2D CCNTCM}   & $O(\log^3{n})$ & $O(n^4)$ & $O(n^4)$   \\
\hline
\end{tabular}
\end{center}
\tcaption{Asymptotic circuit resource usage for quantum factoring of an $n$-bit number.}
\label{tab:results}
\end{table}

%\section{Conclusions and Future Work}
\label{sec:conclude}

In this paper, we have presented a 2D architecture for factoring on a quantum
computer.
%While this is only an intermediate progress report, the preliminary
%results are promising.
Using a combination of algorithmic
improvements (carry-save adders and parallelized phase estimation) 
and architectural improvements (irregular two-dimensional layouts and
constant-depth communication), we conclude
that we can run
the central part of Shor's factoring algorithm (quantum period-finding)
with asymptotically smaller depth than previous implementations.
%, both
%asymptotically and numerically.

%We also provide a constructive upper bound on the overhead
%of nearest-neighbor (NTC) architectures over AC architectures.
%The best-known depth for factoring has constant-depth, but it is an open
%question whether we can satisfy the nearest-neighbor constraint while
%maintaining this constant depth. Furthermore, the cost of decreasing depth
%is often that of increasing width or size, and this time-space tradeoff is
%currently not very well known both for the current work and other
%implementations mentioned in Section \ref{sec:related}.
%Characterizing these tradeoffs is the most obvious extension to
%the current work.

For future work, we would like to determine the exact width, depth, and size of
our proposed factoring circuit, including the constants, as well as
further optimizing our depth to be constant.
%We would also like to determine how our approach compares to the constant-depth
%results in \cite{Browne2009} in terms of circuit size and width.
Along those lines, Rosenbaum recently showed how to convert any $n$-qubit CCAC
circuit to an equivalent CCNTC circuit in constant depth using $n^2$ ancillae
\cite{Rosenbaum2012}.
It is an interesting open question how a
generic conversion of a constant-depth CCAC factoring architecture
\cite{Browne2009,Hoyer2002} to CCNTC compares to our hand-optimized circuit.
The depth of our circuit may also be improved by extending the carry-save adder
from
a $3\rightarrow 2$ circuit to any $2^{n-1}\rightarrow n$ circuit.

The authors wish to thank Aram Harrow, Austin Fowler, and David Rosenbaum for
useful discussions.
P. Pham conducted the factoring part of this work during
an internship at Microsoft Research.
He also acknowledges funding of the architecture and layout portions
of this work from
the Intelligence Advanced Research Projects Activity
(IARPA) via Department of Interior National Business Center contract
number D11PC20167. The U.S. Government is authorized to reproduce and
distribute reprints for Governmental purposes notwithstanding any
copyright annotation thereon. Disclaimer: The views and conclusions
contained herein are those of the authors and should not be
interpreted as necessarily representing the official policies or
endorsements, either expressed or implied, of IARPA, DoI/NBC, or the
U.S. Government.

\section{Conclusions and Future Work}
\label{sec:conclude}

In this paper, we have presented a 2D architecture for factoring on a quantum
computer using a model of nearest-neighbor, concurrent two-qubit
interactions, a classical controller, and communication between
independent modules. We call this new model
\textsc{2D CCNTCM}.
%While this is only an intermediate progress report, the preliminary
%results are promising.
Using a combination of algorithmic
improvements (carry-save adders and parallelized phase estimation)
and architectural improvements (irregular two-dimensional layouts and
constant-depth communication, and parallel modules), we conclude
that we can run
the central part of Shor's factoring algorithm (quantum period-finding)
with asymptotically smaller depth than previous implementations.
%, both
%asymptotically and numerically.

%We also provide a constructive upper bound on the overhead
%of nearest-neighbor (NTC) architectures over AC architectures.
%The best-known depth for factoring has constant-depth, but it is an open
%question whether we can satisfy the nearest-neighbor constraint while
%maintaining this constant depth. Furthermore, the cost of decreasing depth
%is often that of increasing width or size, and this time-space tradeoff is
%currently not very well known both for the current work and other
%implementations mentioned in Section \ref{sec:related}.
%Characterizing these tradeoffs is the most obvious extension to
%the current work.

A natural extension of the current work is to improve its
depth to constant using the approach outlined in
\cite{Hoyer2002,Siu1993}, generalizing the carry-save adder to
a block-save adder using threshold gates. It would also be
beneficial to determine lower bounds for the
time-space tradeoffs involved in Shor's factoring algorithm.
These results would tell us whether we have found an
optimal nearest-neighbor circuit.

\nonumsection{Acknowledgements}
\noindent
The authors wish to thank Aram Harrow, Austin Fowler, and David Rosenbaum for
useful discussions.
P. Pham conducted the factoring part of this work during
an internship at Microsoft Research.
He also acknowledges funding of the architecture and layout portions
of this work from
the Intelligence Advanced Research Projects Activity
(IARPA) via Department of Interior National Business Center contract
number D11PC20167. The U.S. Government is authorized to reproduce and
distribute reprints for Governmental purposes notwithstanding any
copyright annotation thereon. Disclaimer: The views and conclusions
contained herein are those of the authors and should not be
interpreted as necessarily representing the official policies or
endorsements, either expressed or implied, of IARPA, DoI/NBC, or the
U.S. Government.

\nonumsection{References}
\noindent


\bibliography{PhamSvore_QIC}
\bibliographystyle{ieeetr}
%\appendix
%\noindent

\end{document}

