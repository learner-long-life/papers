\newdimen \myunit
\newdimen \myhsize
\newdimen \myvsize
	\myunit 1.5pt
	\myhsize 12\myunit
	\myvsize 10\myunit
\setlength{\unitlength}{\myunit}
\newcommand{\wire}[1]{\raisebox{3\myunit}[0cm][0cm]{\small #1}}
\newcommand{\smwidth}{8}
\newcommand{\smhalf}{4}
\newcommand{\qthicklines}{\linethickness{1.5\myunit}}

\newcommand{\setsize}[1]{\myunit=#1 \myvsize=10\myunit \myhsize=12\myunit \setlength{\unitlength}{#1}}

\newcommand{\qheight}{10}
\newcommand{\qtopheight}{5}
\newcommand{\qtopheightminusthreehalves}{3.5}
\newcommand{\qtopheightplustwo}{7}
\newcommand{\qbottomheight}{5}
\newcommand{\qbottomheightminusthreehalves}{3.5}
\newcommand{\qbottomheightplustwo}{7}

\definecolor{darkgreen}{rgb}{0,0.5,0}

% For sn (skip):  0 = -, 1 = |, 2 = +.
%           3, 4 are ++ (for use with boxes  3 is box, 4 is inverse
% For cn, nn, xn:  0 = +, 1 = bottom of gate, 2 = top of gate.
%           xn:  3 is no control.  4 is box, 5 is inverse
% same for tb, tp

% control narrow
\def\cn#1{
\begin{picture}(4,\qheight)(0,0)
  \put(2,\qtopheight){\circle*{2}}
  \put(0,\qtopheight){\line(1,0){4}}
\ifcase #1
    \put(2,0){\line(0,1){\qheight}}
\or \put(2,\qtopheight){\line(0,1){\qbottomheight}}
\or \put(2,\qtopheight){\line(0,-1){\qtopheight}}
\fi
\end{picture}
}

% color control narrow
\def\ccn#1#2{
\begin{picture}(4,\qheight)(0,0)
  \put(0,\qtopheight){\line(1,0){4}}
\textcolor{#1}{
  \put(2,\qtopheight){\circle*{2}}
\ifcase #2
    \put(2,0){\line(0,1){\qheight}}
\or \put(2,\qtopheight){\line(0,1){\qbottomheight}}
\or \put(2,\qtopheight){\line(0,-1){\qtopheight}}
\fi}
\end{picture}
}

% negated control narrow
\def\nn#1{
\begin{picture}(4,\qheight)(0,0)
  \put(2,\qtopheight){\circle{3}}
  \put(0,\qtopheight){\line(1,0){.5}}
  \put(4,\qtopheight){\line(-1,0){.5}}
\ifcase #1
    \put(2,0){\line(0,1){\qtopheightminusthreehalves}}
    \put(2,\qheight){\line(0,-1){\qbottomheightminusthreehalves}}
\or \put(2,\qheight){\line(0,-1){\qbottomheightminusthreehalves}}
\or \put(2,0){\line(0,1){\qtopheightminusthreehalves}}
\fi
\end{picture}
}

% color negated control narrow
\def\cnn#1#2{
\begin{picture}(4,\qheight)(0,0)
  \textcolor{#1}{\put(2,\qtopheight){\circle{3}}}
  \put(0,\qtopheight){\line(1,0){.5}}
  \put(4,\qtopheight){\line(-1,0){.5}}
\textcolor{#1}{
\ifcase #2
    \put(2,0){\line(0,1){\qtopheightminusthreehalves}}
    \put(2,\qheight){\line(0,-1){\qbottomheightminusthreehalves}}
\or \put(2,\qheight){\line(0,-1){\qbottomheightminusthreehalves}}
\or \put(2,0){\line(0,1){\qtopheightminusthreehalves}}
\fi}
\end{picture}
}

% negation narrow (allows 3 for no control)
\def\xn#1{
\begin{picture}(4,\qheight)(0,0)
  \put(2,\qtopheight){\circle{4}}
  \put(0,\qtopheight){\line(1,0){4}}
\ifcase #1
    \put(2,0){\line(0,1){\qheight}}
\or \put(2,\qheight){\line(0,-1){\qbottomheightplustwo}}
\or \put(2,0){\line(0,1){\qtopheightplustwo}}
\or \put(2,\qtopheightplustwo){\line(0,-1){4}}
\fi
\end{picture}
}

% negation narrow (allows 3 for no control)
\def\cxn#1#2{
\begin{picture}(4,\qheight)(0,0)
  \textcolor{#1}{
  \put(2,\qtopheight){\circle{4}}
  \put(0,\qtopheight){\line(1,0){4}}
\ifcase #2
    \put(2,0){\line(0,1){\qheight}}
\or \put(2,\qheight){\line(0,-1){\qbottomheightplustwo}}
\or \put(2,0){\line(0,1){\qtopheightplustwo}}
\or \put(2,\qtopheightplustwo){\line(0,-1){4}}
\fi}
\end{picture}
}


% skip narrow; also use at start
\newcommand{\sn}[1]{
\begin{picture}(4,\qheight)(0,0)
\ifcase #1
    \put(0,\qtopheight){\line(1,0){4}}
\or \put(2,0){\line(0,1){\qheight}}
\or \put(0,\qtopheight){\line(1,0){4}}
    \put(2,0){\line(0,1){\qheight}}
\or \put(0,\qtopheight){\line(1,0){4}}
    \multiput(2,1)(0,\qtopheight){2}{\line(0,1){\qtopheightminusthreehalves}}
\fi
\end{picture}
}

% color skip narrow; also use at start
\newcommand{\csn}[2]{
\begin{picture}(4,\qheight)(0,0)
\ifcase #2
    \put(0,\qtopheight){\line(1,0){4}}
\or \textcolor{#1}{\put(2,0){\line(0,1){\qheight}}}
\or \put(0,\qtopheight){\line(1,0){4}}
    \textcolor{#1}{\put(2,0){\line(0,1){\qheight}}}
\fi
\end{picture}
}

% skip medium ; use between depths
\def\sm#1{
\begin{picture}(\smwidth,\qheight)(0,0)
\ifcase #1
    \put(0,\qtopheight){\line(1,0){\smwidth}}
\or \put(\smhalf,0){\line(0,1){\qheight}}
\or \put(0,\qtopheight){\line(1,0){\smwidth}}
    \put(\smhalf,0){\line(0,1){\qheight}}
\or \put(0,\qtopheight){\line(1,0){\smwidth}}
    \multiput(\smhalf,1)(0,\qtopheight){2}{\line(0,1){\qtopheightminusthreehalves}}
\fi
\end{picture}
}

% Everything else is wide 
% skip
\def\sx#1{
\begin{picture}(12,\qheight)(0,0)
\ifcase #1
    \put(0,\qtopheight){\line(1,0){12}}
\or \put(6,0){\line(0,1){\qheight}}
\or \put(0,\qtopheight){\line(1,0){12}}
    \put(6,0){\line(0,1){\qheight}}
\or \multiput(0,\qtopheight)(2.6,0){\qtopheight}{\line(1,0){1.6}}
    \put(1,0){\line(0,1){\qheight}}
    \qthicklines
    \put(11,0){\line(0,1){\qheight}}
    \thinlines
\or \multiput(0,\qtopheight)(2.6,0){\qtopheight}{\line(1,0){1.6}}
    \qthicklines
    \put(1,0){\line(0,1){\qheight}}
    \thinlines
    \put(11,0){\line(0,1){\qheight}}
\fi
\end{picture}
}

% control
\def\dx#1{
\begin{picture}(12,\qheight)(0,0)
  \put(6,\qtopheight){\circle*{2}}
  \put(0,\qtopheight){\line(1,0){12}}
\ifcase #1
    \put(6,0){\line(0,1){\qheight}}
\or \put(6,\qtopheight){\line(0,1){\qbottomheight}}
\or \put(6,\qtopheight){\line(0,-1){\qtopheight}}
\or {}
\or \put(1,0){\line(0,1){\qheight}}
    \qthicklines
    \put(11,0){\line(0,1){\qheight}}
    \thinlines
\or \qthicklines
    \put(1,0){\line(0,1){\qheight}}
    \thinlines
    \put(11,0){\line(0,1){\qheight}}
\fi
\end{picture}
}

% open circle
\def\ex#1{
\begin{picture}(12,\qheight)(0,0)
  \put(6,\qtopheight){\circle{3}}
  \put(0,\qtopheight){\line(1,0){4.5}}
  \put(12,\qtopheight){\line(-1,0){4.5}}
\ifcase #1
    \put(6,0){\line(0,1){\qtopheightminusthreehalves}}
    \put(6,\qheight){\line(0,-1){\qbottomheightminusthreehalves}}
\or \put(6,\qheight){\line(0,-1){\qbottomheightminusthreehalves}}
\or \put(6,0){\line(0,1){\qtopheightminusthreehalves}}
\or {}
\or \put(1,0){\line(0,1){\qheight}}
    \qthicklines
    \put(11,0){\line(0,1){\qheight}}
    \thinlines
\or \qthicklines
    \put(1,0){\line(0,1){\qheight}}
    \thinlines
    \put(11,0){\line(0,1){\qheight}}
\fi
\end{picture}
}

% XOR
\def\nt#1{
\begin{picture}(12,\qheight)(0,0)
  \put(6,\qtopheight){\circle{4}}
  \put(0,\qtopheight){\line(1,0){12}}
\ifcase #1
    \put(6,0){\line(0,1){\qheight}}
\or \put(6,\qheight){\line(0,-1){\qbottomheightplustwo}}
\or \put(6,0){\line(0,1){\qtopheightplustwo}}
\or \put(6,\qtopheightplustwo){\line(0,-1){4}}
\or \put(1,0){\line(0,1){\qheight}}
    \qthicklines
    \put(11,0){\line(0,1){\qheight}}
    \thinlines
    \put(6,\qtopheightplustwo){\line(0,-1){4}}
\or \qthicklines
    \put(1,0){\line(0,1){\qheight}}
    \thinlines
    \put(11,0){\line(0,1){\qheight}}
    \put(6,\qtopheightplustwo){\line(0,-1){4}}
\fi
\end{picture}
}

% open circle, but with a line through it
\def\ox#1{
\begin{picture}(12,\qheight)(0,0)
  \put(6,\qtopheight){\circle{3}}
  \put(0,\qtopheight){\line(1,0){12}}
\ifcase #1 
    \put(6,0){\line(0,1){\qheight}}
\or \put(6,\qheight){\line(0,-1){6.5}}
\or \put(6,0){\line(0,1){6.5}}
\fi
\end{picture}
}

\newcommand{\ct}[1]{
\begin{picture}(12,\qheight)(0,0)
  \multiput(0,\qtopheight)(11,0){2}{\line(1,0){1}}
  \put(6,\qtopheight){\circle{\qheight}}
  \put(0,0){\vbox to \myvsize{\vfill
	\hbox to \myhsize{\hfill #1\hfill}\vfill}}
\end{picture}
}

\newcommand{\ti}[1]{
\begin{picture}(12,\qheight)(0,0)
  \multiput(1,0)(\qheight,0){2}{\line(0,1){\qheight}}
  \multiput(1,0)(0,\qheight){2}{\line(1,0){\qheight}}
  \multiput(0,\qtopheight)(11,0){2}{\line(1,0){1}}
  \put(0,0){\vbox to \myvsize{\vfill
	\hbox to \myhsize{\hfill #1\hfill}\vfill}}
\end{picture}
}

\newcommand{\tc}[1]{
\begin{picture}(12,\qheight)(0,0)
 \put(6,\qtopheight){\circle{\qheight}}
 \multiput(0,\qtopheight)(11,0){2}{\line(1,0){1}}
 \put(0,0){\vbox to \myvsize{\vfill
	\hbox to \myhsize{\hfill #1\hfill}\vfill}}
\end{picture}
}

\newcommand{\tb}[2]{
\begin{picture}(12,\qheight)(0,0)
  \put(1,0){\line(0,1){\qheight}}
  \qthicklines
  \put(11,0){\line(0,1){\qheight}}
  \thinlines
  \multiput(1,0)(\qheight,0){2}{\line(0,1){\qheight}}
  \multiput(0,\qtopheight)(11,0){2}{\line(1,0){1}}
  \put(0,0){\vbox to \myvsize{\vfill
	\hbox to \myhsize{\hfill #2\hfill}\vfill}}
\ifcase #1
{}
\or \put(1,0){\line(1,0){\qheight}}
\or \put(1,\qheight){\line(1,0){\qheight}}
\or \multiput(1,0)(0,\qheight){2}{\line(1,0){\qheight}}
\fi
\end{picture}
}

\newcommand{\tp}[2]{
\begin{picture}(12,\qheight)(0,0)
  \qthicklines
  \put(1,0){\line(0,1){\qheight}}
  \thinlines
  \put(11,0){\line(0,1){\qheight}}
  \multiput(0,\qtopheight)(11,0){2}{\line(1,0){1}}
  \put(0,0){\vbox to \myvsize{\vfill
	\hbox to \myhsize{\hfill #2\hfill}\vfill}}
\ifcase #1
{}
\or \put(1,0){\line(1,0){\qheight}}
\or \put(1,\qheight){\line(1,0){\qheight}}
\or \multiput(1,0)(0,\qheight){2}{\line(1,0){\qheight}}
\fi
\end{picture}
}

\newcommand{\place}[1]{\vbox to \myvsize{\vfill
	\hbox to \myhsize{\hfill #1\hfill}\vfill}}
\def\plac#1#2{\vbox to \myvsize{\vfill
	\hbox to #1\myhsize{#2\hfill}\vfill}}
\newcommand{\pv}{\place{\vbox to \myvsize{\vfill\vfill\vfill\vfill\smash{$\vdots$}\vfill}}}
\newcommand{\pcd}{\place{$\cdots$}}


