\documentclass{article}
\usepackage{eprint}
\usepackage[osf]{mathpazo}

\title{Constant-Depth Circuits for Teleportation, Fanout, and Cat State Creation}

\author{Paul Pham}

\newcommand{\targfix}{\qw {\xy {<0em,0em> \ar @{ - } +<.39em,0em>
\ar @{ - } -<.39em,0em> \ar @{ - } +
<0em,.39em> \ar @{ - }
-<0em,.39em>},<0em,0em>*{\rule{.01em}{.01em}}*+<.8em>\frm{o}
\endxy}}

\newcommand{\targless}{\qw {\xy {<0em,0em> \ar @{ - } +<.39em,0em>
\ar @{ - } -<.39em,0em> \ar @{ - } +<0em,.80em> \ar @{ - }
-<0em,.39em>},<0em,0em>*{\rule{.01em}{.01em}}*+<.8em>\frm{o}
\endxy}}

\input{Qcircuit}

\newcommand{\braket}[2]{\langle #1|#2 \rangle}
\newcommand{\normtwo}{\frac{1}{\sqrt{2}}}
\newcommand{\norm}[1]{\parallel #1 \parallel}

\newcommand\coolleftbrace[2]{%
#1\left\{\vphantom{\begin{matrix} #2 \end{matrix}} \right.}

\newcommand\coolrightbrace[2]{%
\left. \vphantom{\begin{matrix} #2 \end{matrix}} \right\} #1}

\begin{document}

\maketitle

\begin{abstract}
We review the techniques for performing teleportation
and unbounded fanout which are behind recent results in performing
Shor's algorithm in constant-depth on a nearest-neighbor architecture
with a classical controller. We give concrete circuits for both techniques,
place them within a historical context, and discuss their limitations.
\end{abstract}

%%%%%%%%%%%%%%%%%%%%%%%%%%%%%%%%%%%%%%%%%%%%%%%%%%%%%%%%%%%%%%%%%%%%%%%%%%%%%%%
\section{Introduction}
Many recent works in quantum circuit construction are concentrated around a
core group of techniques that execute certain primitive gates in constant-depth.
Traditionally, in order to be fault-tolerant,
primitive gates have been assumed to come from the Clifford
group, 
which must be efficiently implementable on any physical technology.
The typical generating set of the Clifford group is usually given as
the following:

\begin{equation}
X =
 \left[
  \begin{array}{cc}
    0 & 1 \\
    1 & 0 \\
  \end{array} \right]
\qquad
Z = \Lambda(e^{i\pi}) =
 \left[
  \begin{array}{cc}
    1 & 0 \\
    0 & -1 \\
  \end{array} \right]
\qquad
K = \Lambda(e^{i\pi/2}) =
 \left[
  \begin{array}{cc}
    1 & 0 \\
    0 & i \\
  \end{array} \right]
\end{equation}
\begin{equation}
H = \normtwo
 \left[
  \begin{array}{cc}
    1 & 1 \\
    1 & -1 \\
  \end{array} \right]
\qquad
CNOT =
 \left[
  \begin{array}{cccc}
    1 & 0 & 0 & 0 \\
    0 & 1 & 0 & 0 \\
    0 & 0 & 0 & 1 \\
    0 & 0 & 1 & 0
  \end{array} \right]
\end{equation}

This is often augmented with the single-qubit rotation $\Lambda(e^{i\pi/4})$
in order to be universal, although this gate by itself may be hard to
implement in many error-correcting codes.

\begin{equation}
\Lambda(e^{i\pi/4}) = 
 \left[
  \begin{array}{cc}
    1 & 0 \\
    0 & e^{i\pi/4} \\
  \end{array} \right]
\end{equation}

Before about 2009, many works measured the circuit resources of
depth, size, and width when implementing Shor's
factoring algorithm \cite{Zalka1998} \cite{Zalka2006} \cite{Kutin2006}
\cite{Fowler2004}
on various architectural models (AC, 1D NTC, and 2D NTC)
\cite{VanMeter2006}
using the set of primitive gates above, or a variant which only counts
two-qubit gates. However, is there a way to further reduce circuit
resources by considering other gates as primitive? And can these new primitive
gates be efficiently implemented using some capability previously
unconsidered, yet is physically realizable?

The answer to both of these questions is yes. However, there are several parts
to this puzzle
of showing that Shor's factoring algorithm could be performed in constant-depth
on an augmented nearest-neighbor architecture. It is now known that this task
in possible, but it remains to be seen what is the optimal size and width of
such a constant-depth circuit.

In 2002, H{\o}yer and {\v S}palek showed that if an unbounded quantum
fanout gate was taken as a primitive multi-qubit gate (as is physically
the case for trapped ions),
factoring could be performed in constant-depth. However, how could an
unbounded fanout gate be implemented in an architecture where only two-qubit
gates, and not multi-qubit gates, are efficient? Can we make use
of a classical controller, which is available to us in experiments
through fast digital computers?

In 2009, Browne, Kashefi, and Perdrix answered
both of these questions in the affirmative by
linking the power of measurement-based (one-way) quantum computing to
H{\o}yer and {\v S}palek's unbounded fanout results, building upon previous
results about the power of the measurement-based model by Broadbent
and Kashefi in 2007 \cite{Broadbent2007}. Using this connection,
the logarithmic-depth factoring circuit of Cleve and Watrous from
2000 \cite{Cleve2000}
can be done in constant-depth on a $k$-dimensional
nearest neighbor architecture with a
classical controller, a model which we call \textsc{kD CCNTC}, for $k \ge 2$.
This construction
relies on two new primitives,
\emph{constant-depth teleportation} and \emph{constant-depth fanout}, which
we will review in these notes.

%%%%%%%%%%%%%%%%%%%%%%%%%%%%%%%%%%%%%%%%%%%%%%%%%%%%%%%%%%%%%%%%%%%%%%%%%%%%%%
\section{Constant-Depth Teleportation (\textsc{CDT})}

In the normal teleportation protocol, a quantum state $\ket{\psi}$ is transported
physically from a source qubit (called $A$ for Alice) to a target qubit
(called $B$ for Bob) using a so-called two-qubit
\emph{Bell pair}, or an entangled state of the following form:

\begin{equation}
\ket{\Phi^{+}} = \normtwo(\ket{00} + \ket{11})
\end{equation}

This is actually one of four possible Bell states, or maximally-entangled
two-qubit states, which forms an alternate basis from the usual computational
basis.

\begin{eqnarray}
\ket{\Phi^{-}} & = & \normtwo(\ket{00} - \ket{11}) \\
\ket{\Psi^{+}} & = & \normtwo(\ket{01} + \ket{10}) \\
\ket{\Psi^{-}} & = & \normtwo(\ket{01} - \ket{10})
\end{eqnarray}

Note that we can form the state $\ket{\Phi^{+}}$
starting from two unentangled qubits
initialized in the $\ket{0}$ state through the following circuit, which
enacts a two-qubit gate that we'll call $U_{00}$.

\begin{center}
\begin{figure}[!h]
\begin{displaymath}
\Qcircuit @C=1em @R=.7em {
\lstick{\ket{0}} & \gate{H} & \ctrl{1} & \qw \\
\lstick{\ket{0}} & \qw      & \targfix & \qw
}
\begin{matrix}% matrix for right braces
\vphantom{a}\\ 
\coolrightbrace{\ket{\Phi^{+}}}{a \\ a }
\end{matrix}%
\end{displaymath}
\caption{Creating a Bell pair}
\label{fig:create-bell-pair}
\end{figure}
\end{center}

The other three Bell states can be created using similar circuits.
Furthermore, we can perform a measurement in the Bell-state basis
(usually simply called the Bell basis) by 
converting back to the computational basis (by applying $U_{00}^\dag$)
and measuring in the computational basis. This is shown in Figure
\ref{fig:measure-bell}, where the classical bits $j$ and $k$ denote
the outcome measurements and tell us which of the four Bell basis states
we have projected into.

\begin{center}
\begin{figure}[!h]
\begin{displaymath}
\begin{array}{ccc}

\begin{array}{c}
\begin{matrix}% matrix for left braces
\vphantom{a}\\ 
\coolleftbrace{\ket{\Phi^{+}}}{a \\ a }
\end{matrix}
\Qcircuit @C=1em @R=.7em {
& \qw & \multimeasure{1}{\text{Bell}} & \cw & j \\
& \qw & \ghost{\text{Bell}}           & \cw & k 
}\\
\end{array}
&
\begin{array}{c}
\equiv
\end{array}
&
\begin{array}{c}
\begin{matrix}% matrix for left braces
\vphantom{a}\\ 
\coolleftbrace{\ket{\Phi^{+}}}{a \\ a }
\end{matrix}
\Qcircuit @C=1em @R=.7em {
& \qw & \ctrl{1} & \gate{H} & \qw & \meter & \cw & j \\
& \qw & \targfix & \qw      & \qw & \meter & \cw & k
}\\
\end{array}
%\Qcircuit @C=1em @R=.7em {
%\qw & \qw      & \ctrl{1} & \qw & \meter & \cw & $j$ \\
%\qw & \gate{H} & \targfix & \qw & \meter & \cw & $k$
%}
\end{array}
\end{displaymath}
\caption{Measuring in the Bell basis}
\label{fig:measure-bell}
\end{figure}
\end{center}

Then the circuit for teleportation from $A$ to $B$ is performed in Figure
\ref{fig:normal-teleport}
using a Bell state $\ket{\Phi^{+}}^{AB}$ that is shared between Alice and
Bob and a source state $\ket{\psi}^A$ to teleport that belongs to Alice
only. Note that the state that is teleported to Bob is $\ket{\psi}$ up to
Pauli corrections, which are measured by Alice and then communicated
classically to Bob.

\begin{center}
\begin{figure}[!h]
\begin{displaymath}
\Qcircuit @C=1em @R=.7em {
\lstick{\ket{\psi}^A} & \qw      & \qw      & \multimeasure{1}{\text{Bell}} & \cw & \rstick{j} \\
\lstick{\ket{0}^A}    & \gate{H} & \ctrl{1} & \ghost{\text{Bell}}           & \cw & \rstick{k} \\
\lstick{\ket{0}^B}    & \qw      & \targfix & \qw                           & \qw & \rstick{X^k Z^j \ket{\psi}} \\
}
\end{displaymath}
\caption{Normal teleportation protocol}
\label{fig:normal-teleport}
\end{figure}
\end{center}

Now then, consider two changes to the above circuit. Instead of Alice and
Bob being separate parties who may be arbitrarily far away from each other,
assume now that Alice and Bob are merged
into an omniscient classical controller which can access all qubits at the
same time. Further, assume that the qubits are on a regular lattice and
can only interact via single-qubit and two-qubit nearest neighbor gates.
This architectural model is called 2D CCNTC.

Our problem now is to
teleport a state $\ket{\psi}$ from one qubit to another arbitrarily far
away (say $n$ qubits away) where all the intervening qubits have been
initialized to $\ket{0}$. These intervening qubits are known as the
\emph{teleportation channel} and intuitively one can think of it as a road
or a railway for transmitting qubits which must be kept clear (reinitialized
back to $\ket{0}$ after use).
By repeating the normal teleportation procedure
$n$ times in sequence, it is easy to see that we can perform this task
in depth $O(n)$. However, because we have a classical controller, we can
actually interleave Bell measurements, measure all in one step, and then
apply Pauli corrections in a second step. This will give us the constant-depth
teleportation (CDT) procedure that we are looking for.

\begin{center}
\begin{figure}[!h]
\begin{displaymath}
\Qcircuit @C=1em @R=.7em {
\lstick{\ket{\psi}}	& \qw      & \qw      & \qw & \qw & \qw & \qw & \qw                                          & \qw & \qw & \multimeasure{1}{\text{Bell}} & \cw & \rstick{j_1} \\
\lstick{\ket{0}}    & \gate{H} & \ctrl{1} & \qw & \qw & \qw & \qw & \qw                                          & \qw & \qw & \ghost{\text{Bell}}           & \cw & \rstick{k_1} \\
\lstick{\ket{0}}    & \qw      & \targfix & \qw & \qw & \qw & \qw & \qw_{X^{k_1}Z^{j_1}\ket{\psi}}               & \qw & \qw & \multimeasure{1}{\text{Bell}} & \cw & \rstick{j_2} \\
\lstick{\ket{0}}    & \gate{H} & \ctrl{1} & \qw & \qw & \qw & \qw & \qw                                          & \qw & \qw & \ghost{\text{Bell}}           & \cw & \rstick{k_2} \\
\lstick{\ket{0}}    & \qw      & \targfix & \qw & \qw & \qw & \qw & \qw_{X^{k_2}Z^{j_2}X^{k_1}Z^{j_1}\ket{\psi}} & \qw & \qw & \multimeasure{1}{\text{Bell}} & \cw & \rstick{j_3} \\
\lstick{\ket{0}}    & \gate{H} & \ctrl{1} & \qw & \qw & \qw & \qw & \qw                                          & \qw & \qw & \ghost{\text{Bell}}           & \cw & \rstick{k_3} \\
\lstick{\ket{0}}    & \qw      & \targfix & \qw & \qw & \qw & \qw & \qw & \qw_{X^{k_3}Z^{j_3}X^{k_2}Z^{j_2}X^{k_1}Z^{j_1}\ket{\psi}} & \qw & \qw & \gate{X^{k_3}} & \gate{Z^{j_3}} & \gate{X^{k_2}} & \gate{Z^{j_2}} & \gate{X^{k_1}} & \gate{Z^{j_1}} & \qw & \rstick{\ket{\psi}} \\
}
\end{displaymath}
\caption{A constant-depth teleportation circuit due to Aram Harrow.}
\label{fig:cd-teleport}
\end{figure}
\end{center}

It is important to note that after a CDT has happened, all the qubits in the
channel have been projectively measured into a product state of classical
bits, including the original source qubit which was in the state $\ket{\psi}$.
Therefore, these can all be simultaneously corrected back to $\ket{0}$ by the
classical controller. In this way, we can keep reusing the teleportation
channel indefinitely. 

If one were to count single-qubit gates,
two-qubit gates, and single-qubit computational basis measurements as a
single step, CDT can be executed in depth 6.

%%%%%%%%%%%%%%%%%%%%%%%%%%%%%%%%%%%%%%%%%%%%%%%%%%%%%%%%%%%%%%%%%%%%%%%%%%%%%%
\section{Constant-Depth Fanout (\textsc{CDF})}

Now we turn to a harder operation, unbounded quantum fanout in constant depth.
Classically, we take it for granted that we can create copies of a bit,
but quantumly we are prevented from doing this by the no-cloning theorem.
Creating an unentangled copy of a generic quantum state is not even a linear
operation, let alone unitary. However, we can create entangled copies using
a CNOT into an initialized qubit.

\begin{equation}
\ket{\psi} \otimes \ket{0} \equiv
(\alpha\ket{0} + \beta\ket{1}) \otimes \ket{0} \rightarrow
\alpha\ket{00} + \beta\ket{11}
\end{equation}

We will denote the unbounded fanout operation as a variation of CNOT with
one control but multiple (say $n$) target qubits. Certainly we can create
$n$ entangled operations in logarithmic depth by apply a tree of CNOTs
as shown in Figure \ref{fig:fanout-tree}. It doesn't matter that the CNOTs
operate on non-adjacent qubits, since we can use CDT from the previous
section to perform a local CNOT and then teleport the target qubit an
arbitrarily far distance.

\begin{center}
\begin{figure}[!h]
\begin{displaymath}
\begin{array}{ccc}
\begin{matrix}% matrix for left braces
\vphantom{a}\\ 
\coolleftbrace{n}{a \\ a \\ a \\ a \\ a \\ a \\ a \\ a }
\end{matrix}
\vphantom{a}
\qquad \qquad \qquad
\underbrace{
\begin{array}{c}
\Qcircuit @C=.7em @R=1.2em {
\lstick{\alpha\ket{0}+\beta\ket{1}} & \qw & \ctrl{4} & \qw & \ctrl{2} & \qw & \ctrl{1} & \qw \\
                                    & \qw  & \qw      & \qw & \qw      & \qw & \targfix & \qw \\
                                    & \qw  & \qw      & \qw & \targfix & \qw & \ctrl{1} & \qw \\
                                    & \qw  & \qw      & \qw & \qw      & \qw & \targfix & \qw \\
                                    & \qw  & \targfix & \qw & \ctrl{2} & \qw & \ctrl{1} & \qw \\
                                    & \qw  & \qw      & \qw & \qw      & \qw & \targfix & \qw \\
                                    & \qw  & \qw      & \qw & \targfix & \qw & \ctrl{1} & \qw \\
                                    & \qw  & \qw      & \qw & \qw      & \qw & \targfix & \qw
}\\
\vphantom{a}
\end{array}
}_{\log_2{n}}
\begin{matrix}% matrix for right braces
\vphantom{a}\\ 
\coolrightbrace{\alpha\ket{0}^{\otimes n} + \beta\ket{1}^{\otimes n}}{a \\ a \\ a \\ a \\ a \\ a \\ a \\ a }
\end{matrix}%

&
\qquad
=
\qquad
&
\begin{array}{c}
\Qcircuit @C=.7em @R=1.2em {
\lstick{\alpha\ket{0}+\beta\ket{1}} & \qw  & \ctrl{1} & \qw \\
                                    & \qw  & \targfix & \qw \\
                                    & \qw  & \targless & \qw \\
                                    & \qw  & \targless & \qw \\
                                    & \qw  & \targless & \qw \\
                                    & \qw  & \targless & \qw \\
                                    & \qw  & \targless & \qw \\
                                    & \qw  & \targless & \qw \\
                                    & \qw  & \targless & \qw
}
\end{array}
\end{array}
\end{displaymath}
\caption{The equivalence of unbounded fanout to a logarithmic depth tree of CNOTs.}
\label{fig:fanout-tree}
\end{figure}
\end{center}

How can we improve the depth? To answer that, consider a special
state called the $n$-qubit \emph{cat state}, which is the maximally-entangled
state on $n$ qubits. We denote this state as $\ket{\Phi_n}$ as shown in
Equation \ref{eqn:cat}. Note that the Bell state $\ket{\Phi^{+}}$ from the
previous section is simply $\ket{\Phi_2}$.

\begin{equation}
\label{eqn:cat}
\ket{\Phi_n} = \frac{1}{\sqrt{2}}
\left(\ket{0}^{\otimes n} + \ket{1}^{\otimes n}\right)
\end{equation}

Assume that we can create $\ket{\Phi_n}$ efficiently, say in
constant-depth. (We will describe how to do this in the next section.)
Then we can perform
unbounded fanout by entangling the source qubit
$\ket{\psi} = \alpha\ket{0} + \beta\ket{1}$ with the cat state as shown in
Equation \ref{eqn:copy}. This accomplishes the operation of constant-depth
fanout (CDF), as shown in Figure \ref{fig:copy}.

\begin{equation}
(\alpha \ket{0} + \beta \ket{1}) \otimes \ket{0}^{\otimes n} \rightarrow
\alpha \ket{0}^{\otimes (n+1)} + \beta \ket{1}^{\otimes (n+1)}
\label{eqn:copy}
\end{equation}

\begin{center}
\begin{figure}[!h]
\begin{displaymath}
\begin{matrix}% matrix for left braces
\vphantom{a}\\ 
\coolleftbrace{\ket{\Phi_n}}{a \\ a \\ a \\ a \\ a \\ a \\ a \\ a }
\end{matrix}
\vphantom{a}
\qquad \qquad \qquad
\underbrace{
\begin{array}{c}
\Qcircuit @C=.7em @R=1.2em {
\lstick{\alpha\ket{0}+\beta\ket{1}} & \qw  & \multimeasure{1}{\text{Bell}} & \cw & \rstick{j} \\
                                    & \qw  & \ghost{\text{Bell}}           & \cw & \rstick{k} \\
                                    & \qw  & \qw                           & \qw & \qw \\
                                    & \qw  & \qw                           & \qw & \qw \\
                                    & \qw  & \qw                           & \qw & \qw \\
                                    & \qw  & \qw                           & \qw & \qw \\
                                    & \qw  & \qw                           & \qw & \qw \\
                                    & \qw  & \qw                           & \qw & \qw
}\\
\vphantom{a}
\end{array}
}_{\log_2{n}}
\begin{matrix}% matrix for right braces
\vphantom{a}\\ 
\vphantom{a}\\ 
\coolrightbrace{X^{k}Z^{j}\left(\alpha\ket{0}^{\otimes {n-1}} + \beta\ket{1}^{\otimes {n-1}}\right)}{ a \\ a \\ a \\ a \\ a \\ a \\ a }
\end{matrix}%
\end{displaymath}
\caption{Entangling a source state with an $n$-qubit cat state to perform unbounded fanout.}
\label{fig:copy}
\end{figure}
\end{center}

When we refer to \textsc{CDF} in the rest of this note, we mean the
above constant-depth implementation of unbounded quantum fanout on a
nearest-neighbor
architecture with a classical controller. This is in contrast to the
notion of a primitive, unbounded fanout operation which may be available
in a particular physical technology (e.g. trapped ions).

%%%%%%%%%%%%%%%%%%%%%%%%%%%%%%%%%%%%%%%%%%%%%%%%%%%%%%%%%%%%%%%%%%%%%%%%%%%%%%
\section{Constant-Depth Cat State Creation \textsc{CDCSC}}

Now we discuss how to create the $n$-qubit cat state $\ket{\Phi_n}$ in
constant depth. Note that for some fixed-size, small cat states,
say $\ket{\Phi_3}$, we can create these simply using a ladder of CNOTs
as shown in Figure \ref{fig:cat3}.

\begin{center}
\begin{figure}[!h]
\begin{displaymath}
\Qcircuit @C=.7em @R=1.2em {
\lstick{\ket{0}} & \qw  & \ctrl{1} & \qw & \qw      & \qw \\
\lstick{\ket{0}} & \qw  & \targfix & \qw & \ctrl{1} & \qw \\
\lstick{\ket{0}} & \qw  & \qw      & \qw & \targfix & \qw
}
\begin{matrix}% matrix for right braces
\vphantom{a}\\ 
\coolrightbrace{\ket{\Phi_3}}{a \\ a \\ a  }
\end{matrix}%
\end{displaymath}
\caption{Circuit for creating the 3-qubit cat state.}
\label{fig:cat3}
\end{figure}
\end{center}

However, this is not scalable for $\ket{\Phi_n}$. We can build an arbitrary-sized
cat state by using a sequence of fixed-size cat states and Bell basis measurements
as shown in Figure \ref{fig:cd-cat}. There are several details to note in this
figure. The first is that the majority of the qubits are used in the
``scaffolding'' to create the cat state (in this case, the six qubits which
are measured to obtain a classical outcome $j_i$ and $k_i$). Just like for
\textsc{CDT}, these classical outcomes are used for simultaneous
Pauli corrections on the qubits which are actually part of the usable cat
state (in this case, the four qubits which are labelled with $\ket{\ell}$.

The second detail to note is that the first and last qubit in this chain is
$\ket{\Phi_2}$, while the intermediate qubits are $\ket{\Phi_3}$. This is
because the first and last qubit in each $\ket{\Phi_3}$ is involved in a
Bell measurement that entangles this triplet of qubits with the preceding
and the succeeding one. Only the middle qubit in each triplet is usable as
a qubit of the usable cat state. However, after the simultaneous round of
Bell measurements, the scaffolding qubits are in a classical product state
to the cat state, and can be corrected back to $\ket{0}$ and reused again in
the future.

\begin{center}
\begin{figure}[!h]
\begin{displaymath}
\begin{matrix}% matrix for left braces
\vphantom{a}\\ 
\coolleftbrace{\ket{\Phi_2}}{a \\ a }\\
\vphantom{a}\\
\coolleftbrace{\ket{\Phi_3}}{a \\ a \\ a}\\
\vphantom{a}\\
\coolleftbrace{\ket{\Phi_3}}{a \\ a \\ a}\\
\vphantom{a}\\
\coolleftbrace{\ket{\Phi_2}}{a \\ a }
\end{matrix}
\vphantom{a}
\begin{array}{c}
\Qcircuit @C=1em @R=1em {
 & \qw & \qw                           & \qw & \qw & \qw & \qw & \rstick{\ket{\ell}} \\
 & \qw & \multimeasure{1}{\text{Bell}} & \cw & \cw & \cw & \cw & \rstick{j_1} \\
 & \qw & \ghost{\text{Bell}}           & \cw & \cw & \cw & \cw & \rstick{k_1} \\
 & \qw & \qw                           & \qw & \qw & \qw & \qw & \rstick{X^{k_1}Z^{j_1}\ket{\ell}} \\
 & \qw & \multimeasure{1}{\text{Bell}} & \cw & \cw & \cw & \cw & \rstick{j_2} \\
 & \qw & \ghost{\text{Bell}}           & \cw & \cw & \cw & \cw & \rstick{k_2} \\
 & \qw & \qw                           & \qw & \qw & \qw & \qw & \rstick{X^{k_2}Z^{j_2}X^{k_1}Z^{j_1}\ket{\ell}}  \\
 & \qw & \multimeasure{1}{\text{Bell}} & \cw & \cw & \cw & \cw & \rstick{j_3} \\
 & \qw & \ghost{\text{Bell}}           & \cw & \cw & \cw & \cw & \rstick{k_3} \\
 & \qw & \qw                           & \qw & \qw & \qw & \qw & \rstick{X^{k_3}Z^{j_3}X^{k_2}Z^{j_2}X^{k_1}Z^{j_1}\ket{\ell}}
}
\end{array}
\end{displaymath}
\caption{A constant-depth creation circuit for a cat state (in this example, $\ket{\Phi_4}$) due to Aram Harrow.}
\label{fig:cd-cat}
\end{figure}
\end{center}

%%%%%%%%%%%%%%%%%%%%%%%%%%%%%%%%%%%%%%%%%%%%%%%%%%%%%%%%%%%%%%%%%%%%%%%%%%%%%%
\section{Limitation to \textsc{CDF} and \textsc{CDCSC}}

In contrast to \textsc{CDT}, there is no way to disentangle $\ket{\Phi_n}$ from
$\ket{\psi}$. Therefore, we say that CDF \emph{consumes the cat state}.
In order to keep a similar channel clear for unbounded-fanout, we must
repeatedly shuffle a used cat state off to one side and shuffle in fresh
ancillae initialized to $\ket{0}$.
While CDF enables us to compress many circuits to constant-depth, it comes
as this cost of maintaining these ``garbage'' cat states, which are still
entangled with our computation state. If any of them
decoheres, it will corrupt our output. Therefore, \textsc{CDF}
may not be as efficient
in practice for physical technologies without efficient multi-qubit gates.
This is a major limitation of \textsc{CDF} in practice, and indicates that
minimizing depth at all costs may not be desirable, and that some other
circuit resource or combination of resources is a better measure of
architectural efficiency. This is an open problem. See \cite{Pham2012c}
for one proposed circuit resource.

%%%%%%%%%%%%%%%%%%%%%%%%%%%%%%%%%%%%%%%%%%%%%%%%%%%%%%%%%%%%%%%%%%%%%%%%%%%%%%
\section{Further Progress}

In the specific case of factoring, the Browne-Kashefi-Perdrix result
showed how to combine \textsc{CDF} with the logarithmic-depth implementation
of H{\o}yer-{\v S}palek to get a constant-depth factoring implementation in
a \textsc{CCAC} architecture, that is an \textsc{AC} architecture augmented
with a classical controller. In this section, we examine two additional
questions.

How does the efficiency of a factoring circuit change if we impose the
restriction of nearest-neighbor interactions? In particular, for purposes
of realistic fabrication, what is the efficiency of a factoring circuit in
\textsc{2D CCNTC}? Pham and Svore showed in 2011 that factoring an $n$-bit
number can be
performed in this setting in poly-logarithmic depth and with size
and width $O(n^6)$. \cite{Pham2012b}. It is an open question whether the
size and width can be improved, and how the size and width scale if the depth
is brought down to a constant.

What other kinds of circuits can we efficiently convert
from \textsc{CCAC} to \textsc{kD CCNTC}? That is, can this conversion
be done in constant depth and
polynomial overhead in size and width? In 2012,
Rosenbaum \cite{Rosenbaum2012} showed this was possible for any $n$-qubit gate
by using a two-dimensional array of $n^2$ qubits. By using \textsc{CDT} to
transport each of the qubits to a different column, and then down to the
same row again, arbitrary pairs of the original qubits can be made
nearest neighbors. Ordinary \textsc{kD NTC} operations can then be
performed on these ``projected'' qubits, and then they can be teleported back
to their original positions if desired. All ancillae qubits can be reused,
and this entire procedure occurs in constant depth.

We hope these notes provide a useful summary of recent
constant-depth techniques.

%%%%%%%%%%%%%%%%%%%%%%%%%%%%%%%%%%%%%%%%%%%%%%%%%%%%%%%%%%%%%%%%%%%%%%%%%%%%%%
\section{Acknowledgments}

Thanks to Aram Harrow and David Rosenbaum for useful discussions and
providing many of the circuits that appear in these notes.

\bibliography{constant-depth}
\bibliographystyle{tocplain}

\end{document}